%!TEX TS-program = xelatex
%!TEX encoding = UTF-8 Unicode
% !TEX root = ../../2017-GS-COME01-INVITO-ASCOLTO.tex

\clearpage

\thispagestyle{empty}

\includepdf[offset=-10 0,
			scale=2.15,
		    pagecommand={
		    	\begin{tikzpicture}[
					remember picture,
					overlay]
		    	\node [xshift=4cm,yshift=1cm] at (current page.south west) {\color{white}{\emph{\textbf{Massimo Cacciari} e \textbf{Luigi Nono}, Venezia 1983}}};
				\end{tikzpicture}}
		    ]{images/nono/luigi-nono-massimo-cacciari.pdf}

%Massimo Cacciari e Luigi Nono, Venezia 1983.

\clearpage

%-------------------------------------------------------------
%------------------- LUIGI NONO - POST-PRAE LUDIUM PER DONAU -
%-------------------------------------------------------------

\chapter*{1987. Luigi NONO.\\\emph{Post-Prae Ludium per Donau}.}
\addcontentsline{toc}{chapter}{1987. Luigi NONO. \emph{Post-Prae Ludium per Donau}.}

\begin{quote}
	Il percorso della composizione è fissato nei suoi dettagli; la creazione è invece pensata come un appunto per l'esecutore. Nuove possibilità di tecnica dell'esecuzione di una tuba a sei cilindri danno all'interprete la continua libertà di superare questi appunti e creare eventi sonori casuali.

	La trasformazione elettronica del suono è intessuta nella composizione in maniera differenziata.

	La tuba deve captare, elaborare e rispondere ai processi di espansione del suono.

	La notazione data, la nuova tecnica dell'esecuzione e l'elettronica dal vivo, insieme sostituiscono l'effetto di una mia interpretazione\footnote{Luigi Nono, ottobre 1987}.
\end{quote}
