https://www.izotope.com/en/learn/are-you-listening-metering-in-mastering.html

i due sistemi di analisi e misurazione di ampiezza più diffusi, che consideriamo
con la maggior parte dei misuratori riguardano il livello di picco ed il livello
RMS.

Livello di picco

Il livello di picco è il livello di segnale più alto raggiunto dal segnale audio.
È importante prestare attenzione al livello di picco perché se supera il livello
massimo consentito (0 dBFS nella maggior parte dei casi digitali o +24 dBu nel
mondo analogico), il sistema non sarà in grado di riprodurre il segnale con
questa ampiezza e audio taglierà e distorcerà.

Il segnale audio può raggiunge il livello di picco durante i transitori di attacco.

Livello RMS

Tuttavia, il livello di picco in realtà non ci dice nulla sul volume percepito,
perché il nostro cervello ha bisogno di un po' più di tempo per valutare
l'intonazione o il volume. Questo è il motivo per cui misuriamo il livello
RMS (radice quadrata media), che misura il volume medio su una finestra di circa
300 millisecondi. Di conseguenza, il livello RMS mostrerà in generale quanto
apparirà forte un brano rispetto ad altri. Questo corrisponde più da vicino al
modo in cui il nostro cervello percepisce il volume.

Il livello di picco sarà inevitabilmente più alto del livello RMS perché il
segnale trascorre di solito solo una breve quantità di tempo a livello di picco.
Il livello RMS fa una media di questi picchi con il resto dell'audio. Tra queste
due misurazioni, puoi capire rapidamente la presenza e la persistenza dell'audio
(RMS) e la sua relazione con la distorsione (picco).

Fattore di cresta: la differenza tra il livello di picco e il livello RMS

La differenza tra il livello di picco e il livello di RMS è chiamato fattore di
cresta. Misurando la differenza tra il livello massimo e il livello medio,
abbiamo una buona idea di quanto sia dinamica una traccia.

LUFS / LKFS

Il nostro cervello e le nostre orecchie non sono ugualmente sensibili in tutto
lo spettro delle frequenze. Se dovessi riprodurre un tono di 1 kHz a -20 dBFS,
e quindi riprodurre un tono di 50 Hz a -20 dBFS, noteresti che il tono di 1 kHz
suonerebbe più forte di circa 10-12 dB. In entrambi i casi, tuttavia, il
misuratore dovrebbe leggere -20 dBFS, il che non rispecchia realmente la nostra
percezione dell'audio.

Pertanto, utilizziamo spesso la misurazione LUFS (unità di volume a fondo scala)
(a volte chiamata LKFS, ma sono essenzialmente intercambiabili). Questa
misurazione prende in considerazione il fatto che prendiamo una misurazione media
su circa 300 ms per registrare il volume e che siamo più sensibili a determinate
frequenze. Di conseguenza, LUFS è un sistema utile per misurare il volume percepito.

Puoi utilizzare LUFS per quantificare il motivo per cui alcune tracce suonano
più forti di altre, anche se sono mixate allo stesso livello. Tuttavia, poiché
il sistema LUFS tiene conto della percezione, non viene utilizzato per misurazioni
di livello puramente tecniche, come nell'impostazione della soglia di un compressore.

Tipi di misure LUFS

Esistono tre diversi tipi di misurazioni LUFS. I primi due, a breve e breve
termine, sono ragionevolmente simili nella definizione e nell'uso. Il terzo
tipo si chiama integrato. Diamo un'occhiata a come interagiscono queste diverse
misurazioni.

Rumorosità momentanea

Il volume momentaneo (misurato a destra in LUFS in Insight 2) è in media su
circa 300 ms. Il volume a breve termine a sinistra viene misurato per un periodo di
circa 3 secondi. Di conseguenza, i due misurano spesso abbastanza da vicino
l'uno all'altro, ma il volume a breve termine misurerà un po 'più in basso
perché è in media su una finestra più lunga.

Questa finestra di tre secondi per il volume a breve termine ci dice qualcosa in
più sulla persistenza del suono o su un'impressione più generale del volume di
una traccia. Le misurazioni del volume sia temporanee che a breve termine sono
modi perfettamente praticabili per misurare il volume medio in una traccia.

Per capire il volume integrato, tuttavia, immergiamoci un po 'nella storia dietro di esso.

Livello sonoro integrato

Questa misura di livello in un programma completo è chiamata volume integrato.
Se osservi il valore di volume integrato su Insight 2, cambia molto più lentamente.
Questo perché sta calcolando la media dell'intera lunghezza della registrazione.
Negli Stati Uniti, The Calm Act è stato messo in legge per fissare un volume
integrato target per l'audio di trasmissione. Negli Stati Uniti, è -24 LKFS.
Nell'UE, i legislatori hanno scelto di istituire specifiche e regolamentazione
del settore con le specifiche EBU R128 con un volume target di -23 LUFS.
Se un programma non è a questi livelli, il mix verrà alzato o abbassato per
conformarsi.

Quindi, perché dovremmo preoccuparci del volume integrato nella musica?
La maggior parte dei servizi di streaming musicale ora implementa la
normalizzazione del volume, in cui viene misurato il livello integrato e un brano
è conforme a uno standard di riproduzione.

Puoi utilizzare una misurazione LUFS integrata per misurare il tuo mix e capire
come potrebbe essere influenzato su diversi servizi di streaming, a seconda dei
loro standard di riproduzione.

considerazioni

Quando stai padroneggiando, LUFS integrato dovrebbe essere l'ultima cosa a cui
stai pensando. È molto più importante mescolare e padroneggiare la musica in modo
che suoni e si senta bene, non così che colpisca un bersaglio.

Se desideri controllare il volume integrato della traccia, puoi utilizzare uno
strumento come Insight 2 e riprodurre la traccia fino in fondo. Oppure potresti
usare qualcosa come RX, che ha una finestra chiamata statistica della forma d'onda,
che fornirà informazioni sul livello per una registrazione completa o una selezione
di una registrazione. Ma di nuovo, questo è qualcosa di cui dovresti preoccuparti
solo quando hai finito con il tuo padrone.

Lavorare con un metro

Durante la misurazione, generalmente lavorerai con il livello di picco e medio.
Questo dà un'idea del margine rimanente e del volume generale della traccia. Se
il volume medio non è abbastanza alto, puoi sperimentare cose come la limitazione
per aumentare il livello medio.

Il mix nelle schermate sottostanti non è ancora stato masterizzato. Puoi vedere
che c'è un po 'di margine (il livello di picco non è al massimo) e quel livello
medio è compreso tra -13 e -16 dB. Poiché la traccia qui è una traccia rock,
avremo bisogno di qualche altro dB di volume per soddisfare gli standard di genere
e l'artista.

Aumentare il livello porterebbe i nostri picchi sopra 0 dB, quindi usiamo il
modulo Maximizer di Ozone per ottenere questo volume extra. Possiamo vedere su
Ozone che stiamo solo limitando un po 'i picchi (calcio e rullante), con solo
un dB o due in riduzione del guadagno.
Guarda i contatori di Insight in basso: il livello medio è ora salito. Questo
sarà più vicino al volume che l'artista vuole dal genere (rock), e sappiamo che
non stiamo distorcendo perché il nostro livello di picco è inferiore a 0 dB.

Questo ovviamente non copre l'intera gamma delle decisioni di mastering, ma il
rapporto picco-medio - fattore di cresta - è un aspetto davvero importante del
processo di mastering.

Altre visualizzazioni importanti per il mastering

La maggior parte delle visualizzazioni che abbiamo discusso finora riguardano
l'ampiezza, ma ci sono alcune visualizzazioni che ci aiutano a indirizzare la
frequenza più direttamente.

Misuratore del fattore di cresta del controllo del bilanciamento tonale

Poiché siamo meno sensibili alle frequenze più basse, i bassi devono essere a un
livello più alto in un mix, quindi spesso occupano più spazio rispetto a qualsiasi
altra area dello spettro delle frequenze. Il contenuto di picco nella fascia bassa
spesso appartiene al kick drum, mentre il basso guiderà più del livello medio di
fascia bassa.

Un forte drum kick consumerà molta headroom in un master e renderà difficile per
noi ascoltare il basso, o per noi avere la headroom per spingere la traccia più
forte senza distorcere. D'altra parte, se il kick drum è sepolto sotto il basso,
perdiamo un senso di spinta che otteniamo dal basso transitorio del kick.

Il misuratore del fattore di cresta nell'angolo in alto a sinistra del controllo
del bilanciamento tonale mostra il fattore di cresta della fascia bassa per
aiutarci a capire meglio questa relazione kick-bass. Nello screenshot qui sotto,
noterai che abbiamo un fattore di cresta abbastanza alto. Questo ci dice che i
picchi sono un po 'rumorosi rispetto al livello medio nei minimi, il che potrebbe
significare che il calcio è un po' caldo. Potremmo voler trovare un modo per
controllare il livello del kick drum, usando EQ, compressione o anche Low End
Focus.

spettrogrammi

Possiamo anche usare uno strumento spettrogramma come quello in RX per vedere
parti del segnale che non riusciamo a sentire molto bene. Vedi nello spettrogramma
sotto che ci sono alcuni impulsi forti tra 10–20 Hz. Anche se potremmo non
sentirlo molto bene, sta occupando spazio per la testa. Potremmo rotolare via un
po 'di quell'inudibile fascia bassa per liberare lo spazio per la testa e
permetterci di portare su tutta la pista, senza una notevole differenza nei minimi

Potrebbe essere allettante utilizzare un filtro passa alto per eliminarlo, ma
anche i filtri passa alto creano i propri problemi. Uno scaffale basso può essere
uno strumento più delicato e meno problematico per l'audio dell'EQing. Possiamo
anche utilizzare uno spettrogramma per studiare il contenuto ad alta frequenza,
un altro posto in cui siamo meno sensibili. Questo è un ottimo modo per trovare
rumore, clic, pop e altri artefatti che non possiamo ascoltare.

Conclusione

Le visualizzazioni corrette sono una grande protezione contro il nostro imperfetto
senso di percezione soggettiva. Se non puoi fidarti delle tue orecchie tutto il tempo,
puoi almeno fidarti dei tuoi misuratori e visualizzazioni per essere un controllo
istintivo per l'ascolto. Sia che tu stia lavorando con il livello di picco, il
livello di RMS, il fattore di cresta o uno dei tre tipi di misurazioni LUFS,
conoscere il tuo modo di aggirare un metro e perché la misurazione può effettuare o
interrompere una sessione di mastering ti preparerà per il successo con la prossima progetto.


https://www.izotope.com/en/learn/psychoacoustics-how-perception-influences-music-production.html


dBu

The term dBu is used to describe the r.m.s. voltage of a signal relative to 0.775 V (r.m.s.). Thus 0 dBu = 0.775 V r.m.s. [ITU-R BS.645-2]
