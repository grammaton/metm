%!TEX TS-program = xelatex
%!TEX encoding = UTF-8 Unicode
% !TEX root = ../metm.tex

\chapter{SPETTROMORFOLOGIA}
\startcontents[chapters]
\printcontents[chapters]{}{1}{}

%\begin{adjustwidth}{.19\textwidth}{0mm}

Dannis Smalley conia il termine Spettromorfologia (data) per rappresentare l'idea di come le
componenti dello spettro sonoro si modificano nel tempo.

Tutti i suoni, siano essi musicali o no, posseggono un'identità spettromorfologica.

Quando decodifichiamo un'oggetto sonoro la particolare configurazione di energia spettrale
ci permette di riconoscerlo, ne riconosciamo l'impronta spettromorfologica.

Il termine spettromorfologia non è inteso come una qualità acustica discreta di un suono,
ma identifica come l'energia spettrale viene ascoltata.

L'esperienza di spazialità riguarda molti aspetti di un suono. Riguarda la sua energia e come questa viene rilasciata, come si muove attraverso lo spazio, come innesca meccanismi di ritardo. Noi osserviamo e partecipiamo come osservatori e parte del fenomeno.

Nella composizione acusmatica la spettromorfologia di un suono osserva gli stessi principi. I suoni vengono introdotti nello spazio e questi svelano lo spazio, generano ed articolano lo spazio.

Spazialità e tempo sono indissolubilmente legate.

La spazialità dell'immagine acusmatica può essere investigata attraverso l'indagine di tre concetti chiave:

\begin{description}
  \item[Lo spazio prospettico] è la relazione tra posizione spaziale, movimento e rapporto tra le spettromorfologie viste dal punto di osservazione dell'ascoltatore
  \item[Spazio legato alla sorgente] riguarda le zone spaziali e le immagini mentali prodotte da, o dedotte da, sorgenti sonore e le loro cause (se ce ne sono)
  \item[spazialità spettrale]è l'impressione di spazio e spaziosità prodotta dalla presenza e dal movimento all'interno del continuum delle frequenze udibili.
  \end{description}

Esperienza sensoriale inter-modale dell'immagine acusmatica.

Come ascoltatore percettivo, io sono il centro dell'esperienza sensoriale.

Sono in relazione con la mia corporeità, la qualità di essere un corpo materiale; con il mio spazio egocentrico, quello che raggiungo e copro con le braccia, attorno alla mia posizione di ascoltatore; ho un punto di osservazione, unico, la mia posizione di ascoltatore osservatore con il mio spazio prospettico, e da questa percepisco e ricevo l'immagine acusmatica.

%\end{adjustwidth}

\clearpage
