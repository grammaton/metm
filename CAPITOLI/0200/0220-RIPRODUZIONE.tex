%!TEX TS-program = xelatex
%!TEX encoding = UTF-8 Unicode
% !TEX root = ../../metp.tex

\section{Riproduzione}

JEAN CLAUDE RISSET SUONO TEMPO SUONO SPAZIO

L'esperienza sonora è viva nella produzione acustica del suono. Culturalmente
siamo portati a pensare che i nostri sistemi audio siano in grado di ricostruire
quella esperienze, senza mai minimamente pensare che questo è semplicemente e
praticamente impossibile. Ci sono diversi motivi culturali per cui lo pensiamo.
Ci sono innumerevoli questioni e parole consumate dalla pratica quotidiana.
Parole private del loro significato originale e consumante nella concezione di
una immotivata sicurezza operativa, come per esempio quella applicata al suono
stereo a due canali. Nell'esperienza quotidiana siamo divertiti e soddisfatti
dalla stereofonia, senza necessariamente mettere a confronnto o relazione con
quella che dovrebbe essere la vera esperienza. La semplice differenza tra i due
canali, destro e sinistro, è proposta come stereofonia. Altoparlante sinistro,
altoparlante destro. Centro? Fantasma. Raramente si sente descrivere questa
tipologia di ascolto come monofonica, mono o doppio mono, o polifonica. Una
soprgente di questo tipo ascoltato attraverso le cuffie diventa orecchio
sinistro, orecchio destro e centro della testa. Certamente può esserci
un'essenza di musica in tutto ciò, la persistenza i un segnale culturale. Manca
però in una riproduzione di questo tipo qualsiasi parvenza di spazio acustico e
atmosfera di realismo o verosomiglianza. La sapiente configurazione nella
ripresa microfonica e l'elaborazione del segnale operato  durante la
registrazione possono migliorare le cose, ma nella migliore delle ipotesi la
stereofonia così basata su due sorgenti di riproduzione rimane un formato
direzionalmente e spazialmente debole, privato, e antisociale, che può soddisfare
un solo punto di ascolto.

% Abbiamo bisogno di più canali per acquisire, archiviare e riprodurre anche le
% percezioni essenziali dei campi sonori tridimensionali. Questo è ciò che il
% mondo del cinema conosce da decenni, e ora i cinema hanno ben 62 canali nei
% formati audio coinvolgenti. Questo è eccessivo per le esigenze musicali, ma più
% di due sarebbe bello. Fortunatamente ci sono esempi di eccellente musica
% multicanale e indicazioni che una versione binaurale di essa farà parte dei
% sistemi di realtà virtuale. Rimanete sintonizzati. Dal punto di vista
% scientifico, le origini dell'acustica moderna si trovano in gran parte nel
% dominio delle sale per l'esecuzione di musica classica. Che questa musica
% piaccia o meno a una persona, le percezioni di base generate da queste
% esibizioni dal vivo sono generosamente condivise in tutta la musica registrata,
% qualunque sia il genere. Il riverbero, l'ampiezza, l'involucro e così via sono
% semplicemente esperienze percettive semplicemente piacevoli, e agli ingegneri
% della registrazione sono stati forniti elaboratori elettronici elaborati che
% consentono loro di essere incorporati in qualsiasi tipo di musica, aggiungendo
% alla tavolozza artistica. Il futuro suona bene.


In questo ampio ragionamento introduttivo non abbiamo considerato che l'oggetto
altoparlante, il diffusore incaricato di ingannare i nostri sensi, è forse
l'oggetto più brutto ed antiquato tecnologicamente della catena elettroacustica.

% altoparlante da esposizioni elettroacustiche

Si va quindi verso l'affermazione che il suono prodotto, l'evento acustico, non
può essere completamente riprodotto. La rappresentazione, in quanto tale,
riproduzione, necessariamente contempla una perdita di informazioni.

Se ora rileggessimo il periodo precedente con la consapevolezza che non abbiamo
ancora minimamente accennato allo spazio acustico che avvolge l'evento sonoro,
alla inevitabile forza narratrice che questo imprime al segnale acustico, forse
immaginiamo meglio il livello di perdita nella riproduzione.

È interessante notare che, anche in diverse sale, i timbri essenziali di voci
strumenti musicali rimangono notevolmente costanti. Abbiamo una notevole
capacità di separare il suono della sorgente dal suono della sala. In altre
parole, sembriamo adattarci alla stanza in cui ci troviamo e “ascoltarla” per
ascoltare le fonti sonore. Una variante di questa interpretazione è che ci
impegniamo in ciò che Bregman (1999) chiama "analisi della scena uditiva" e
"trasmettiamo" il suono delle voci e degli strumenti come separato in modo
significativo dal suono della stanza. Facciamo questo a tal punto che ci si può
concentrare sul suono di una sezione dell'orchestra, sopprimendone altre. Due
orecchie e un cervello sono notevoli. Se una performance in una sala da concerto
sembra non avere bassi, come alcuni fanno, l'inclinazione è quella di incolpare
la sala, non i musicisti o i loro strumenti. Istintivamente sappiamo dove sta
la colpa. Per ora, è sufficiente notare che riprodurre un'esperienza da sala da
concerto significa consegnare sia i componenti timbrici che quelli spaziali.
Questo non è facile.

% blumlein sulla percezione della voce
