%%%%%%%%%%%%%%%%%%%%%%%%%%%%%%%%%%%%%%%%%%%%%%%%%%%%%%%%%%%%%%%%%%%%% DECODIFICA
%%%%%%%%%%%%%%%%%%%%%%%%%%%%%%%%%%%%%%%%%%%%%%%%%%%%%%%%%%%%%%%%%%%%%%%%%%%%%%%%
\section{Decodifica}

La decodifica di un segnale \amb~del primo ordine in un segnale stereofonico
\emph{LR} è una delle operazioni più importanti per comprendere la simmetrica
semplicità della tecnologia ideata da \mg.

L'esercizio più semplice consiste nel considerare i quattro segnali \emph{WXYZ}
per quello che sono, segnali microfonici (anche se generati da un \emph{encoder}
o da una matrice \emph{AB module}). Un segnale microfonico può essere messo in
relazione con un altro segnale microfonico per generare un terzo segnale che
abbia caratteristiche derivanti dai primi due. Nella tabella \ref{tab:polarcoef}
sono illustrate relazioni tra le figure polari \emph{non-direzionale} e
\emph{bidirezionale} al fine di generare figure polari intermedie e variabili
(anche nel tempo). Considerato ciò, osservando la quaterna di figure polari
descriventi uno segnale \emph{B-Format}, possiamo a colpo d'occhio intuire che
attraverso la relazione della componente \emph{W} e una qualsiasi delle
componenti \emph{XYZ} possiamo generare figure polari cardioide lungo gli assi
cartesiani del sistema tridimensionale. Allo stesso modo, mettendo in relazione
due tra \emph{XYZ} possiamo generare figure bidirezionali mediane ad esse. Ancora,
mettendo in relazione le figure polari mediane così ottenute con la componente
\emph{W} otteniamo lungo gli assi mediani una variazione polare da bidirezionale
a cardioide a omnidirezionale. È presto intuito inoltre che mettendo quindi in
relazione l'intera quaterna di segnali si possono puntare microfoni virtuali
a figura polare variabile verso ogni angolo interno alla sfera.

Vista in questo modo la decodifica di un segnale \amb~del primo ordine perde
quell'aura di magia oscura alla quale si può credere osservando un sotware
produrre un segnale stereofonico, riacquisendo unn'abilità pratica a generare
suoni da esso mediante operazioni di mix, perché no, direttamente su un mixer.

L'attività di decodifica stereofonica attraverso un mixer permette di ascoltare
la decodifica, cercare soluzioni adatte sia al contenuto \amb~che alla sala ed
alla situazione di decodifica.
