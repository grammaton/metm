%!TEX TS-program = xelatex
%!TEX encoding = UTF-8 Unicode
% !TEX root = ../../metp.tex
\begin{refsection}

\chapter{AMBISONIC}
\startcontents[chapters]
\printcontents[chapters]{}{1}{}

\vfill\null

Il primo passo necessario verso la comprensione del concetto di \emph{Stereofonia},
per di arrivare alle tecniche ed alle tecnologie elettroacustiche che la
rendono possibile, è stabilire una base concettuale solida attraverso
l'etimologia del termine e dei termini ad esso collegati. \emph{Stereo}, dal greco
\emph{Stere\'os}, significa \emph{solido}. Non un numero, non una configurazione
ma un aggettivo qualitativo. Nel dizionario inglese Oxford: \emph{Solid, firm
and stable in shape. Having three dimensions}. Solido, \emph{solid}, dalla radice
latina di \emph{Solidus, Sollus}, intero.

Con la parola \emph{Stereofonia} dovremmo quindi descrivere una condizione
nella quale \emph{phon\={e}}, sempre dal greco, \emph{suono}, la \emph{voce},
arrivi all'ascoltatore solida, integra, ferma e stabile nella sua forma (sonora)
multi dimensionale, intera.

\include{CAPITOLI/0400/0401-RADICI}

%%%%%%%%%%%%%%%%%%%%%%%%%%%%%%%%%%%%%%%%%%%%%%%%%%%%%%%%%%%%%%%%%%%%%% LE RADICI
%%%%%%%%%%%%%%%%%%%%%%%%%%%%%%%%%%%%%%%%%%%%%%%%%%%%%%%%%%%%%%%%%%%%%%%%%%%%%%%%
\section{Codifica}

L'idea di riprodurre con il suono anche lo spazio che lo caratterizza risale,
come altre idee ed invenzioni legate al mondo della diffusione e riproduzione
dei suoni, alla fine dell'ottocento, nell'era elettrica della rivoluzione
industriale. In quei decenni si svolgevano grandi eventi e mostre di divulgazione
tecnologica, come l'\emph{International Exposition of Electricity} tenutasi nel 1881.
Insieme a bulbi luminosi, incandescenze e batterie per l'accumulazione
dell'energia, in quel periodo ci fu spazio anche per l'idea di \emph{telefono stereo},
alla base del \emph{Théâtrophone} di Clément Ader. Per tutta la durata
dell'esposizione, ogni sera nelle sale del \emph{Grande Opera} di Parigi, veniva
suonata musica che il pubblico dell'esposizione poteva ascoltare, per alcuni
minuti, al telefono, a circa due chilometri di distanza presso il
\emph{Palais de Industrie}.


%%%%%%%%%%%%%%%%%%%%%%%%%%%%%%%%%%%%%%%%%%%%%%%%%%%%%%%%%%%%%%%%%%%%% DECODIFICA
%%%%%%%%%%%%%%%%%%%%%%%%%%%%%%%%%%%%%%%%%%%%%%%%%%%%%%%%%%%%%%%%%%%%%%%%%%%%%%%%
\section{Decodifica}

La decodifica di un segnale \amb~del primo ordine in un segnale stereofonico
\emph{LR} è una delle operazioni più importanti per comprendere la simmetrica
semplicità della tecnologia ideata da \mg.

L'esercizio più semplice consiste nel considerare i quattro segnali \emph{WXYZ}
per quello che sono, segnali microfonici (anche se generati da un \emph{encoder}
o da una matrice \emph{AB module}). Un segnale microfonico può essere messo in
relazione con un altro segnale microfonico per generare un terzo segnale che
abbia caratteristiche derivanti dai primi due. Nella tabella \ref{tab:polarcoef}
sono illustrate relazioni tra le figure polari \emph{non-direzionale} e
\emph{bidirezionale} al fine di generare figure polari intermedie e variabili
(anche nel tempo). Considerato ciò, osservando la quaterna di figure polari
descriventi uno segnale \emph{B-Format}, possiamo a colpo d'occhio intuire che
attraverso la relazione della componente \emph{W} e una qualsiasi delle
componenti \emph{XYZ} possiamo generare figure polari cardioide lungo gli assi
cartesiani del sistema tridimensionale. Allo stesso modo, mettendo in relazione
due tra \emph{XYZ} possiamo generare figure bidirezionali mediane ad esse. Ancora,
mettendo in relazione le figure polari mediane così ottenute con la componente
\emph{W} otteniamo lungo gli assi mediani una variazione polare da bidirezionale
a cardioide a omnidirezionale. È presto intuito inoltre che mettendo quindi in
relazione l'intera quaterna di segnali si possono puntare microfoni virtuali
a figura polare variabile verso ogni angolo interno alla sfera.

Vista in questo modo la decodifica di un segnale \amb~del primo ordine perde
quell'aura di magia oscura alla quale si può credere osservando un sotware
produrre un segnale stereofonico, riacquisendo unn'abilità pratica a generare
suoni da esso mediante operazioni di mix, perché no, direttamente su un mixer.

L'attività di decodifica stereofonica attraverso un mixer permette di ascoltare
la decodifica, cercare soluzioni adatte sia al contenuto \amb~che alla sala ed
alla situazione di decodifica.


\include{CAPITOLI/0400/0404-ELABORAZIONE}

\printbibliography
\end{refsection}
