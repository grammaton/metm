%!TEX TS-program = xelatex
%!TEX encoding = UTF-8 Unicode
% !TEX root = ../metm.tex

\chapter{PAESAGGIO SONORO}
\startcontents[chapters]
\printcontents[chapters]{}{1}{}

%\begin{adjustwidth}{.19\textwidth}{0mm}

\begin{quote}
  In \emph{The Tuning og the World}\footnote{\emph{Il Paesaggio Sonoro}}, un saggio
  che tratta della storia del suono nelle nostre vite, ho spiegato come rovesciare
  la prosepttiva, da negativa - l'inquinamento acustico - a positiva: la ricerca di
  un design del paesaggio sonoro. Per me il design del paesaggio sonoro non deve
  provenire da qualcosa di estraneo, ma \emph{dal di dentro}, stimolando un numero
  sempre maggiore di persone ad ascoltare i suoni che ci circondano con una
  maggiore attenzione critica. Quali sono i suoni che desideriamo conservare?
  Come sostenerli in modo che le caratteristiche fondamentali del nostro ambiente
  possano essere conservate e anzi diventare più attraenti?
\end{quote}
%\end{adjustwidth}
