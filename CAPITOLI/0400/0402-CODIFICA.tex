%%%%%%%%%%%%%%%%%%%%%%%%%%%%%%%%%%%%%%%%%%%%%%%%%%%%%%%%%%%%%%%%%%%%%% LE RADICI
%%%%%%%%%%%%%%%%%%%%%%%%%%%%%%%%%%%%%%%%%%%%%%%%%%%%%%%%%%%%%%%%%%%%%%%%%%%%%%%%
\section{Codifica}

L'idea di riprodurre con il suono anche lo spazio che lo caratterizza risale,
come altre idee ed invenzioni legate al mondo della diffusione e riproduzione
dei suoni, alla fine dell'ottocento, nell'era elettrica della rivoluzione
industriale. In quei decenni si svolgevano grandi eventi e mostre di divulgazione
tecnologica, come l'\emph{International Exposition of Electricity} tenutasi nel 1881.
Insieme a bulbi luminosi, incandescenze e batterie per l'accumulazione
dell'energia, in quel periodo ci fu spazio anche per l'idea di \emph{telefono stereo},
alla base del \emph{Théâtrophone} di Clément Ader. Per tutta la durata
dell'esposizione, ogni sera nelle sale del \emph{Grande Opera} di Parigi, veniva
suonata musica che il pubblico dell'esposizione poteva ascoltare, per alcuni
minuti, al telefono, a circa due chilometri di distanza presso il
\emph{Palais de Industrie}.
