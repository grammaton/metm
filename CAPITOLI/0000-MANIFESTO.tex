%!TEX TS-program = xelatex
%!TEX encoding = UTF-8 Unicode
% !TEX root = ../metp.tex

\chapter*{MANIFESTO}
\addcontentsline{toc}{chapter}{MANIFESTO}

%\begin{adjustwidth}{23mm}{0mm}

Questo lavoro deriva dall'esigenza di organizzare un testo che possa
ristabilire un equilibrio scolastico e, con un po' di presunzione, artistico,
nelle attività didattiche inerenti la Musica Elettronica e le Tecnologie
Musicali. Non è propriamente un manuale quanto una raccolta organizzata di
appunti che seguano un percorso chiaro e progressivo attraverso gli anni di
studio della materia.

L'esigenza di mettere in un contenitore questi materiali viene direttamente
dal percorso d'insegnamento in diversi gradi del percorso formativo elettroacustico.
La bibliografia specifica è ricca ed in continua evoluzione data la materia
viva che la alimenta. Nonostante questo, in lingua italiana, i testi attualmente di
riferimento hanno un assetto ludico, d'intrattenimento. Organizzati ad
unità come fossero un corso di lingua, a livelli, trasformano argomenti musicali
in attitudini ricreative, rafforzando l'idea di didattica applicata alla base
di troppe classi di musica elettronica in Italia.

Un riferimento letterario, ad indirizzo opposto, è il testo di Walter Branchi
\emph{Tecnologia della Musica Elettronica} che dal lontano 1975 ancora riesce
a suggerire un'idea di didattica organica e progressiva ineguagliata, con i
difetti di un'opera prima in tutti i sensi, ma con la prospettiva dello studio
scolastico organizzato. A questo testo si ispira questa raccolta, \emph{METP},
all'idea di un percorso di comprensione che viene da un'idea di scuola,
senza la pretesa di creare un abile manovratore di pomelli in ogni possibile lettore.
Le problematiche essenziali sono altre e sono, sempre prendendo spunto dal lavoro di Branchi,
ancora egregiamente espresse dalle parole di Domenico Guaccero nella prefazione:

\begin{quote}
  \ldots val la pena ricordare e mantenere fermi gli apporti che la “musica
  elettronica” ha introdotto nell'opera musicale\ldots E cioè l'andare alla
  radice del suono, come fatto fisico e (ache) da qui come fatto musicale, la
  possibilità di liberarsi definitivamente dal “dover costruire” sitemi
  intervallari, il porre il problema del rapporto musica esecutore e musica
  spettacolo in termini fino allora inusitati, di riproporre, ma allargandone
  permamentemente i confini, il rapporto tra tecnologia e composizione.
\end{quote}

In questa ottica di indagine, alla \emph{radice del suono}, ci si pone
trasversalmente, lungo il periodo d'apprendimento di uno studente, a qualsiasi
livello. Non è dunque la quantità di nozioni (o unità) che si collezionano con
la lettura, quanto la qualità dell'esperienza formativa a tracciare il percorso.

\begin{quote}
  Cominciare con uno studio del “tempo” e delle "onde"\ldots È li che comincia
  il moto, quindi il suono. È li che musicista e tecnico cominciano ad
  incontrarsi. Prima di tutte le cognizioni di acustica ed elettroacustica o di
  tecnologia elettronica, è li che si può misurare quello che diventa il "nuovo"
  tecnico-musicale itrodotto tramite i mezzi elettronici: com'è stato detto, si
  tratta di "comporre il suono, non comporre col suono".
\end{quote}

La composizione come materia di studio sembra non appartenere più alle classi di
tecnologia. Il mezzo tecnico ha completamente inebriato le menti e distolto
l'attenzione dall'\emph{osservazione mediante composizione} di ciò che percepiamo,
ovvero, ancora: il suono. Uno studente agli inizi del proprio percorso potrebbe
affrontare un esercizio compositivo se gli si fornissero le regole di
comprensione del proprio operato e della relazione di questo con il mondo percepito, per

\begin{quote}
  \ldots andare alla radice del suono. In quest'ambito possono avere più senso
  i vari problemi, come quello di mirare a costruire timbri nuovi\ldots studiare
  per analisi, gli strumenti naturali, in maniera da usare gli strumenti naturali
  come termini di confronto cui l'orecchio è abituato, cioè scritti in codice
  forte.

  Dalla radice del suono, come fatto fisico, si può passare alla radice della
  percezione e della realizzazione sonora.
\end{quote}

Negli anni di scuola superiore, del Liceo, questa metodologia è adottata
\emph{naturalmente}. Sarebbe troppo semplice parlare di materie scientifiche
quali la fisica, la biologia, la chimica per creare paralleli tra percepito,
analizzato e compreso. Girandoci più alla larga possibile potremmo parlare di
latino e italiano, di letteratura. Di imparare a scrivere attraverso la storia
della scrittura, dei meccanismi che hanno resistito alla storia per finire nel
linguaggio comune. Nella musica elettronica, nelle tecnologie musicali, nei
libri ad unità, si procede per atemporalità, per meccanismi e tecnicismi avulsi
dall'essere nel tempo delle cose. Si spiegano le tecniche senza su queste fare
quello che si fa in italiano, scrivere, comporre temi. Come se la composizione
fosse una questione esclusiva dei compositori.

Insegnare a comporre, a scrivere, attraverso il ragionameto, è un esattamente
una questione tecnica

\begin{quote}
  faranno bene ad essere quanto più rigorosi teoriocamente e quanto più
  preparati tecnologicamente, in maniera da poter maneggiare tutte le
  apparecchiature possibili\ldots È la via della didattica, quella a cui
  intelligentemente si rifà questo libro di Branchi\ldots

  Esso non è un testo pensato a priori per insegnare le tecniche elettroniche,
  ma è \emph{il risultato} d'una attività didattica già effettuata\ldots Non
  un'aggiunta all'erudizione o un contributo all'indagine teorica, ma, più
  semplicemente e pertinemtemente, uno strumento di lavoro, da utilizzare nella
  pratica.
\end{quote}

Questo lavoro ha la presunzione di non servire a niente se lo scolpo della
lettura è un'apprendimento collezionistico di espedienti e meccanismi.

Questo lavoro vuole e può fornire meccanismi mentali di indagine,
sperimentazione, speculazione, comprensione e composizione che \emph{una scuola}
dovrebbe innestare in alcuni studenti, in alcuni individui. Non in tutti, come nei
corsi di lingua di base, una unità alla volta, ma solo quelli che da questi
ragionamenti possano alzare gli occhi per scoprire di aver
cambiato il proprio modo di percepire le relazioni tra le cose.

***

Questo testo, pur prendendo prezioso esempio dalle pagine di Branchi, vuole fare
diversi passi in molteplici altre direzioni. La distanza temporale dal testo di
Branchi ci permette di guardare a nuovi modi di fare didattica. Il più
importante ed efficae è quello del libero accesso al codice, alla libera
partecipazione al progetto per un'informazione aperta alla discussione.

Questo testo è quindi open source, il progetto risiede all'indirizzo
\url{https://github.grammaton.io/metp}. Atteaverso questo indirizzo si può
partecipare alla scrittura, all'integrazione, alla discussione al fine di
rendere il testo vivo, in continua composizione.

Please note that this project is released with a Contributor Code of Conduct.
By participating in this project you agree to abide by its terms.

%\end{adjustwidth}

\clearpage
