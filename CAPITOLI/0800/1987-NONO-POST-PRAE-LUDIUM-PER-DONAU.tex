%-------------------------------------------------------------
%------------------- LUIGI NONO - POST-PRAE LUDIUM PER DONAU -
%-------------------------------------------------------------

\subsection*{1987. Luigi NONO.\\\emph{Post-Prae Ludium per Donau}.}

\begin{quote}
	Il percorso della composizione è fissato nei suoi dettagli; la creazione è invece pensata come un appunto per l'esecutore. Nuove possibilità di tecnica dell'esecuzione di una tuba a sei cilindri danno all'interprete la continua libertà di superare questi appunti e creare eventi sonori casuali.

	La trasformazione elettronica del suono è intessuta nella composizione in maniera differenziata.

	La tuba deve captare, elaborare e rispondere ai processi di espansione del suono.

	La notazione data, la nuova tecnica dell'esecuzione e l'elettronica dal vivo, insieme sostituiscono l'effetto di una mia interpretazione\footnote{Luigi Nono, ottobre 1987}.
\end{quote}
