%-------------------------------------------------------------
%--------------------------- KARLHEINZ STOCKHAUSEN STUDIE II -
%-------------------------------------------------------------

\subsection*{1956.\\ Karlheinz STOCKHAUSEN. \\ \emph{Studie II}.}

Lo \emph{StudieII} nel repertorio elettroacustico rappresenta il reperto archeologico, il grado zero della conoscenza elettronica, pensiero cristallizzato ed articolato in testo musicale tuttora leggibile, interpretabile, attraverso un processo compositivo ri-eseguibile.

Il materiale sonoro è quanto di più antico in termini di produzione umana elettronica in musica. Per ricostruire lo spazio tecnico di lavoro che ha plasmato quelle idee musicali, le possibilità tecniche ed il lessico specifico ci serviamo delle parole di alcuni personaggi di quella storia. Così Prieberg scrive dei suoni sinusoidali in \emph{Musica Ex Machina}:

\begin{quote}
	[\ldots] Il suono sinusoidale è un fenomeno completamente nuovo nella musica. Si intende un suono senza armonici, cioè il suono nudo, che forma una vibrazione sinusoidale. È un prodotto dei vibratori graduati elettronici, con i quali di solito vengono prodotti nelle stazioni radio il segnale orario, il diapason per i musicisti e – prima e dopo l’orario giornaliero di trasmissione – le frequenze pilota.
	[\ldots] è veramente senza luogo, una vibrazione isolata, che nasce ed è amplificata nella valvola elettronica e diventa udibile solo nell’altoparlante. Il suono sinusoidale si sottrae a ogni definizione all’infuori di quella con il numero di \emph{Hertz} delle sue vibrazioni. Girando l’interruttore del condensatore del generatore elettronico lo si può spostare in qualsiasi punto dell’intera sfera uditiva\footcite[pag. 170--179]{prieberg:mexm}.
\end{quote}

Lo stesso Prieberg si serve poi delle parole di Herbert Eimert\footnote{
	Herbert Eimert, tra le due guerre lavorava alla Radio di Colonia dal 1927 al 1933. Nel 1945, sotto l'amministrazione britannica, fu il primo tecnico assunto dalla nuova Radio di Colonia (\emph{NWDR}). Nel 1951 insieme a Werner Meyer-Eppler persuade Hanns Hartmann, direttore di NWDR, a creare il primo studio per la musica elettronica che diresse fino al 1962, quando gli successe Karlheinz Stockhausen. Insieme a Hans Ulrich Humpert, scrisse il \emph{Lexikon der elektronischen Musik}, (dizionario della musica elettronica).
} il quale descrive il suono sinusoidale come

\begin{quote}
  neutrale e già abbastanza forte a un’intensità media, perciò non è affatto smorto e insussistente. Siccome manca degli armonici caratteristici, non ha alcun timbro ben definito. Il suo contrassegno principale è la diretta immediatezza del suono. Dipende dalla sua natura elettrica il fatto che esso risuona come una corrente uniforme, rigido e non modulato\footcite[\emph{idem}]{prieberg:mexm}.
\end{quote}

Prieberg descrive alcune pratiche e tecniche di laboratorio di quegli anni

\begin{quote}
	[\ldots] Suoni sinusoidali possono essere sovrapposti in qualsiasi numero e frequenza, sia armonicamente, di modo che si ottiene un suono con un corteggio di armonici naturale, sia non armonicamente, di modo che nasce una cosiddetta « mescolanza di suoni ». [\ldots] L’inizio del suono coincide con l’innesto, e con il disinnesto s’interrompe immediatamente. Perciò il carattere del suono deve essere aggiunto alla composizione e per fare questo si offrono le più varie possibilità. Con le forbici il compositore taglia via dal nastro inciso non solo la lunghezza ma anche le cosiddette « curve di inviluppo », le caratteristiche forme dei suoi elementi acustici. [. . . ] Inoltre egli può servirsi di un apparecchio di risonanza o della riverberazione che si trova in ogni stazione radio. Se la formazione lo soddisfa, inizia allora il montaggio dei molti piccoli ritagli di nastro nella disposizione ritmica desiderata.
	Complicatissimi passaggi dinamici nascono sotto un costante controllo con il centimetro. Siccome il compositore conosce la velocità del nastro, di solito $ 76,2 $ o $ 38,1 $ centimetri al secondo, misurando la lunghezza di pause e suoni egli può produrre ritmi « irrazionali » precisissimi, mentre ogni compositore di musica strumentale fallirebbe senza speranza in quest’impresa.
\end{quote}

Su Studio 2

\begin{quote}
	[\ldots] Esistono diverse possibilità di fissare la musica elettronica in qualcosa di simile a uno « spartito ». Una « partitura » elettronica, lo Studio II di Karlheinz Stockhausen, è già stampata. Un progetto di costruzione, un disegno tecnico per cosìdire, documento unico di una musica dell’avvenire. Al primo sguardo si presenta come un disegno affascinante di vari rettangoli e triangoli.
	L’idea della notazione, cioè la distribuzione del tempo in senso orizzontale da sinistra a destra, e la collocazione delle note e dei suoni in senso verticale, dal basso in alto, rimane inalterata. Invece è nuova l’annotazione assoluta. Le note musicali si limitano a dare un sistema di riferimento. Esse sono relative. Il loro valore assoluto dipende da una convenzione non obbligatoria, dall’altezza relativa del diapason. Un’esecuzione più o meno autentica – anche per quel che riguarda l’altezza di suono – riesce unicamente se si ha una precisa conoscenza di questa convenzione. Invece la musica elettronica è incisa su nastro conformemente alle idee del compositore. L’interpretazione non è nénecessaria népossibile. Perciò la rappresentazione grafica dell’opera elettronica deve essere vera e assolutamente precisa: il tecnico dello studio lavora su di essa. Molte novità saltano agli occhi. L’altezza del suono si misura in Hertz, cioè in vibrazioni al secondo; l’intensità si esprime in Decibel, cioè nei gradi di un aumento del 26 per cento circa dell’energia sonora, che vengono ancora percepiti dall’orecchio – non più con indicazioni così vaghe come forte e pianissimo; per la velocità si evitano le arbitrarie indicazioni andante, presto o largo, si calcola secondo i centimetri del nastro che scorre. Al posto dell’approssimazione interpretativa è subentrata matematica esattezza. La « partitura » elettronica di Stockhausen corrisponde alle premesse di un caso compositivo speciale: le centonovantatre mescolanze di suoni, di cinque suoni sinusoidali ciascuno, che egli usa, non sono montati singolarmente ma risultano dalla relazione meccanica dei suoni sinusoidali sonati uno dopo l’altro nella riverberazione. Siccome si tratta esclusivamente di suoni di uguale intensità e con intervalli costanti, ne consegue subito che una simile annotazione semplificata può valere soltanto per questa composizione elettronica.

	Se si esamina ora nei particolari, si vede che la parte superiore della « partitura » – un po’ più della metà – è costituita da un sistema di linee che comprende lo spazio da 100 fino a $ 17200 $ \emph{Hertz} sfruttato da Stockhausen nello Studio II. Rettangoli di varie forme simbolizzano blocchi di cinque suoni sinusoidali ciascuno, cioè mescolanze di suoni. Sotto questa parte la lunghezza dei blocchi è riportata ancora una volta su di una scala in centimetri di nastro. $ 76,2 $ centimetri corrispondono alla velocità allora usuale del nastro. Più sotto ancora vi è un secondo più sottile sistema di linee per la dinamica da 402 fino a 0 decibel. L’altezza del grafico in questo sistema indica l’intensità corrispondente. Le sue forme determinano i contorni, cioè le curve di inviluppo delle mescolanze dei suoni. In alcuni punti dei due sistemi mancano figure, il che significa pausa. Anche la durata è visibile dal numero nella scala delle lunghezze. Una pagina della partitura rappresenta all’incirca sei secondi di musica.
	I primi sette pezzi elettronici dello studio di Colonia [\ldots] Sono come uno scenario acustico, che provoca senza dubbio eccitamenti molto violenti i quali, pur non causando uno shock, opprimono tuttavia in modo quasi insopportabile il sistema nervoso.
\end{quote}

da La Musica Elettronica3

\begin{quote}
	[\ldots] L’articolazione e l’esatto metodo di produzione del materiale sono stati interamente esposti nella prefazione alla “partitura”, una delle rare rappresentazioni di musica elettronica pubblicate: si potrà rimandare ad essa. In effetti, in essa si ottiene una fusione molto maggiore degli “elementi” all’interno dei “suoni complessi”. Essa è dovuta, in primo luogo, alla loro “uguaglianza dinamica” ed alla complessità molto più grande dei loro rapporti armonici (ricordiamo che nel Primo Studio non si trattava di spettri “armonici” propriamente detti, centrati su un fondamentale unico; almeno tutti gli intervalli facenti parte di un blocco erano degli intervalli giusti, degli intervalli semplici e “trasparenti”). Essa è anche dovuta, per alcuni di essi (per i più “agglomeranti”), al serrarsi di questi intervalli costitutivi che provoca (non soltanto per la percezione, ma, si potrebbe dire, oggettivamente) la distruzione reciproca delle periodicità individuali e (come già il cromatismo simultaneo nelle musiche strumentali di cui abbiamo parlato nel primo capitolo) genera dei veri rumori (molto controllati, tuttavia). Inoltre, l’utilizzazione della camera a eco (“naturale”) come uno degli stessi mezzi della produzione sonora (metodo che reintroduce fin da allora un elemento non elettronico e per questo non strettamente controllato) rinforza ancora la fusione delle componenti (valorizzando i legami interferenziali) e conferisce ai risultati un’unità supplementare, dovuta al colore proprio di cui riveste in un certo senso tutto ciò che passa attraverso di essa.

	Infine, l’attacco simultaneo di diversi blocchi di questa specie da cui si ritagliano eventualmente le regioni armoniche (e la cui ulteriore evoluzione divergente dà, per contrasto, conferma), ed il fatto che Stockhausen abbia superato su questi attacchi di un poco le soglie di registrazione del nastro magnetico (di passare nella zona “rossa” dei potenziometri: fino a + 6dB), producendo, con la distorsione che ne deriva, dei veri transitori, paragonabili agli attacchi di certi strumenti (a percussione), contribuiva ad ottenere dei fenomeni sonori molto più unitari, la cui unità era già caratterizzata da un certo tasso di evoluzione interna, e li rendeva capaci di sopportare meglio il paragone con il carattere “organico” dei fenomeni naturali. Certo, a questo si univa una (molto relativa, ma innegabile nel rigore della prospettiva iniziale) perdita di controllo. Tuttavia le immediate conclusioni che se ne sarebbero potute tirare, le conseguenze che questo avrebbe avuto, nel senso di un certo ammorbidimento dei principi di realizzazione (e in primo luogo di concezione) non erano il solo insegnamento che se ne potesse trarre: Stockhausen (e coloro che seguivano da vicino le sue esperienze cominciando eventualmente a dedicarsi ad esperienze parallele) aveva appreso qui ogni sorta di nozioni precise sulla struttura dei fatti sonori, sulla possibilità di continuare a ricercarne il controllo integrale, anche se questo dovesse durare per un periodo abbastanza lungo e passare attraverso tappe apparentemente in contraddizione.
	In effetti, fin dal compimento del Primo Studio di Stockhausen, altri compositori erano stati invitati a lavorare temporeneemente allo studio di Colonia ed a realizzarvi una composizione per forza di cose modesta. Cosi, alla fine del 1954, una prima esecuzione dei lavori (che occupavano la metà di un concerto) potévenir proposta al pubblico (l’altra parte comprendeva esecuzioni di nuova musica americana da parte di John Cage e di David Tudor). Ad eccezione di qualche dettaglio, tuttavia, nessuno di questi lavori apportava nulla di fondamentalmente nuovo rispetto alle realizzazioni di Stockhausen.

	[\ldots] fenomeni paragonabili fino a un certo punto agli aggregati più indivisibili del Secondo Studio di Stockhausen, potevano essere ottenuti filtrando un fenomeno elettronico il meno periodico, il più disordinato possibile, la cui applicazione acustica si chiama “rumore bianco”. Ed infine alcuni momenti dell’una o dell’altra composizione particolarmente movimentati, particolarmente “microstrutturati” provavano (soprattutto se accelerati ancora, cosa di cui si poteva fare esperienza quotidiana durante il lavoro di studio) che si potevano raggiungere unità di una nuova specie, la cui instabilità, la cui mobilità, sarebbe una delle caratteristiche principali, che le contrapporrebbe dunque in maniera molto netta alla maggior parte dei suoni strumentali (analoghe solo alle misture a un tempo molto dense e molto rapide, come ne esistevano già, per esempio nella seconda delle prime Structures per due pianoforti di Boulez, o nei Kontrapunkte di Stockhausen). Prescindendo dalla perdita di controllo che il tentativo di sistematizzare dei fenomeni di questo tipo poteva rappresentare (e che potrebbe, per lo meno provvisoriamente, essere compensato da criteri di determinazione statistica) era proprio questo l’effetto di una di quelle “immaginazioni concrete” di cui dicevamo, che furono all’origine della nascita della musica seriale generalizzata, immaginazione che ci sforziamo dunque, intravedendola, di rendere attuale con un massimo di efficacia, ad un tempo “espressiva” (cioè piuttosto “qualitativa”) e strutturale (o più “quantitativa”). Sembrava che ci fossero delle vie per condurre a questo senza passare per il missaggio ed il montaggio di elementi sinusoidali (operazione naturalmente fastidiosa nel caso in cui si vogliano realizzare dei fenomeni formalmente antinomici rispetto alla sinusoide – e del resto poco efficaci a causa dei “rumori di fondo,” nel senso molto generale della parola, che la molteplicità delle operazioni introduce ed accumula). Le sinusoidi non rappresentarono dunque più che uno degli estremi di un campo di cui gli altri due “poli” simbolici sembrano essere: l’onda periodica “angolare” cioè il meno sinusoidale possibile, definibile in termini di “impulso” (dente di sega, “ago” o rettangolo) ed il “rumore bianco” o processo vibratorio il meno periodico possibile. Questo allargamento delte “riserve” materiali a cui poteva attingere il compositore apriva ad un tratto alla musica elettronica uno spazio figurativo molto pid ricco, molto più duttile. Se i compositori sapessero rivelarsi sufficientemente attenti, la luce momentanea (ed in alcuni casi del tutto relativa) del principio del controllo integrale, potrebbe non essere che una specie di astuzia per permettere l’appropriazione e la progressiva realizzazione di questi materiali sonori in diverse tappe. Tuttavia la prima di esse sarebbe una tappa in cui l’accento sarebbe posto più sovente, perlomeno per una parte dei compositori (e non necessariamente per i meno “prospettici”), sull’aspetto qualitativo, spontaneo, e quindi sulla generazione relativamente empirica, talvolta davvero improvvisata, delle nuove sonorità).
	[\ldots]
\end{quote}

da Musica Espansa4

\begin{quote}
	Gran parte dell’esperienza seriale elettronica maturata a Colonia è sorretta dall’operare teorico a compositivo di Stockhausen, dalla sua estrema fiducia nell’analisi cognitiva e da una metodologia scientifica di lavoro. Nel periodo dal 1953 al 1954, egli realizza alla WDR i pezzi elettronici Studie I e II, che rimangono tra i progetti musicali più rappresentativi di quella stagione musicale. Un condensato di riflessione teorica e di tecnica realizzativa, ma anche di scontro compositivo con il materiale e con i problemi della percezione musicale\footnote{(nota presente nel testo originale) Lo Studie II è inoltre il primo esempio di musica elettronica rappresentata da una partitura, un tentativo riuscito di creare una mediazione tra gesto formale e ascolto, mediante una descrizione grafica puntuale dello spazio sonoro: ampiezze, durate, misture frequenziali, inviluppi, densità verticale e movimento delle sequenze nel tempo. Dal punto di vista della grafia musicale nella musica elettronica degli anni Cinquanta, esistono altri esempi importanti, tra cui Incontri di fasce sonore (1956) di F. Evangelisti, e Artikulation (1958) di G. Ligeti. Entrambe hanno avuto una stesura che si avvicina a delle partiture di ascolto. Altre musiche concrete o elettroniche dispongono di partiture che vanno intese come partiture di lavoro, tra queste: Timbre-Durèes (1952) di O. Messiaen, Essay (1957) di M. Koenig. Il problema di una grafia della musica elettronica è anche contraddittorio per il significato stesso di partitura, cioè di mezzo simbolico non astratto ma finalizzato per l’esecuzione musicale. Il nastro magnetico è di per séla partitura.}

	Se lo Studie I può essere considerato una prova generale importante di progettazione musicale nel laboratorio della serialità elettronica, lo Studie II rivela un approfondimento compositivo indirizzato a ricercare una maggiore caratterizzazione interna dei materiali elettronici, a un maggiore contrasto tra le diverse tipologie delle misture sonore. In particolare, caratterizzando il pezzo con un uso delle strutture frequenziali molto più ricco e complesso, caricato da una bassa armonicità, in cui prevale alla fine una risultante timbrica assimilabile a quella ottenuta con rumori variamente filtrati e inviluppati. La contrapposizione tra i diversi transitori di attacco e tra le dinamiche è molto più netta; il fortissimo si avvale anche di una controllata saturazione dinamica del magnetofono, che conferisce di conseguenza una maggiore “coloratura” che, come sottolinea Henri Pousseur, “ contribuiva a ottenere dei fenomeni sonori molto più unitari, la cui unità era già caratterizzata da un certo tasso di evoluzione interna, che li rendeva capaci di sopportare meglio il paragone con il carattere organico dei fenomeni naturali”6.
	è utile portare l’attenzione sulla definizione di “organico” in contrapposizione al non organico che Pousseur utilizza nella descrizione del modo di suonare di certi blocchi sonori dello Studie II. Tale osservazione mette in luce uno degli aspetti di maggiore criticità della musica elettronica seriale, dovuta all’immobilità interna del suono. alla mancanza di microarticolazione della materia che conferisce viceversa una stimolazione percettiva e un interesse nell’ascolto e che ci permette di parlare di timbro e di agglomerati sinusoidali, cosa che Stockhausen aveva rilevato.
	Da un lato, il permutare e il combinare continuo dei parametri acustici delle “misture sonore” non facilita l’articolazione del tessuto sonoro, a causa del prodursi di una continua indifferenziazione, dovuta anche all’esiguo numero di parziali che compongono in genere le misture; dall’altra le difficoltà intrinseche nella realizzazione di una vera e propria sintesi additiva del timbro. Quindi il “comporre il suono” con i mezzi tecnologici dell’epoca, conduce ad accentuare volutamente soluzioni più dirette. Da qui l’introduzione di accorgimenti elaborativi estranei, come il passaggio delle misture nella “camera di riverberazione”, con lo scopo di conferire al risultato contorni “di disturbo” non prevedibili e quindi aleatori, in grado di dare respiro e circolazione sanguigna ai blocchi di suono.
	[...] In generale l’ascolto delle opere realizzate nello Studio di Colonia rivela una difficoltà nell’uscire da un’omologazione timbrica e da una economia dei materiali che non permette di ottenere una maggiore flessibilità linguistica.
	è individuabile un interesse primario rivolto soprattutto al sistema delle frequenze, che guida il procedere compositivo anche in un simile contesto sperimentale; Stockhausen, per esempio, utilizza in Studie II un rapporto di 5:1 suddiviso in 25 parti. Probabilmente ha ragione Franco Evangelisti quando sottolinea la necessità di ricorrere a rapporti di distanza, nello spazio frequenziale, espressi da numeri non interi, come condizione per uscire da certi vincoli di periodicità nella generazione spettrale dei suoni e conquistare realmente un diverso mondo sonoro. La sua composizione Incontri di fasce sonore (1956) è l’ultimissimo esempio di una musica elettronica che parte da un’idea costruttiva del timbro secondo il modello delle misture sonore sinusoidali, anche se esistono nel pezzo altre soluzioni a sostegno di una maggiore ricchezza di materiale disponibile. Scrive Evangelisti a proposito dell’organizzazione delle misture sonore presenti nella sua opera:

	\begin{quote}
		La suddivisione dello spazio sonoro e delle sue durate è stata concepita in base ai rapporti psicofisici e in funzione dei parametri degli stessi. La scala delle grequenze è stata suddivisa in 91 gradi a partire dalla frequenza più bassa di 87 Hz alla più alta di 11 ̇950 Hz. Il problema fondamentale è stato qello di non stabilire rapporti di armonia di nessun genere, espressi mediante numeri razionali interi, come per esempio nella scala temperata il rapporto di 2:1, o come nello studio di k. Stockhausen il rapporto di 5:1. [...]
	\end{quote}

La questione non risolta della mobilità interna del suono e della diversificazione dei materiali è un problema di tale complessità che gli assiomi iniziali vengono ben presto rivisitati e frantumata così l’estetica dell’elettronica pura.

\end{quote}
