%-------------------------------------------------------------
%---------------- PIERRE BOULEZ - DIALOGUE DE L'OMBRE DOUBLE -
%-------------------------------------------------------------

\subsection*{1985. Pierre BOULEZ.\\\emph{Dialogue de l'ombre double}.}

\subsubsection*{Scheda dell'opera}

\begin{table}[ht]
%\caption{default}
\begin{center}
\begin{tabular}{>{\sffamily}r>{\normalsize }p{7.5cm}}

\hline
\hline
											& \\
autore:										& Pierre Boulez (1925) \\
titolo:										& \textbf{Dialogue de l'ombre double} \\
											& \textsf{per clarinetto/primo sul palco} \\
											& \textsf{e clarinetto/doppio registrato} \\
											& \\
dedica:										& \emph{A Luciano Berio per il suo sessantesimo compleanno} \\
data di composizione:						& 1984-85 \\
durata:										& 20' \\
											%& \\
Editore:									& Universal Edition \\
Codice di catalogo							& UE 18407 \\
produzione elettronica:						& IRCAM, Parigi \\
realizzazione:								& Andrew Gerzso \\
dispositivi:								& Suono su supporto \\
											%& \\
prima Esecuzione:							& Alain Damiens, 28 Ottobre 1985, Firenze \\
											& \\
\hline
\hline

\end{tabular}
\end{center}
%\label{default}
\end{table}%

%----------------------------------------------------------------------------------------
%	BEGIN MULTICOLS
%----------------------------------------------------------------------------------------

%\begin{multicols}{2} % Two-column layout throughout the main article text

\subsubsection*{Introduzione}

%\begin{description}
%{\small
%\item[Dialogo] - \textgreek{διάλογος}, da \textgreek{διαλέγομαι} “converso”.}
%\end{description}

Il \emph{Dialogue de l'ombre double}, composto e dedicato a Luciano Berio nel 1985, rappresenta, per diversi motivi, un brano di particolare interesse nel repertorio della musica elettroacustica. È un brano che si avvale del supporto elettronico per la sola registrazione del clarinetto/doppio, ma richiede una messa in scena sonora articolata nella diffusione in sala mediante una complessa regia del suono. Il brano, prodotto all'interno dell'IRCAM si avvale delle conoscenze e delle tecniche sviluppate intorno al live electronics, nel contesto di speculazione sullo spazio sonoro da cui è nato poco prima anche \emph{Répons}.%\footnote{\emph{Répons, per sei solisti, orchestra da camera e live electonics}}.

\clearpage

%%------------------------------------------------- IMMAGINE
%
%\begin{figure}[H]
%\begin{center}
%\includegraphics[width=.47\textwidth]{img/ED01_tetrarec_lucier13degrees_CG.jpg}
%\caption{Dettaglio della disposizione dei microfoni durante la registrazione Tetraedrica Spaziata. }
%\label{default}
%\end{center}
%\end{figure}
%
%%------------------------------------------------- IMMAGINE

\begin{flushright}
{\small
\textit{
L'ordine è il piacere della ragione:\\ma il disordine\\ è la delizia dell'immaginazione.\\ Però i calzini usati mettili in lavatrice.\\
 }\textbf{Paul Claudel}}%, La scarpina di raso}
\end{flushright}

%Quote rosso1 Il segreto della creatività è il saper nascondere le proprie fonti. Quote rosso2
%~ Albert Einstein

\subsubsection*{Strutture}

%Dialogo doppia ombra, dedicato a Luciano Berio e scritto per il suo sessantesimo anniversario nel 1985, è stato creato 28 Ottobre 1985 a Firenze da Alain Damiens. Il lavoro viene svolto presso l'IRCAM di Andrew Gerzo, assistente musicale. Su suggerimento di Pascal Gallois, Pierre Boulez nel 1995 trascritto per fagotto, che aveva fatto in precedenza per i campi, un altro dei suoi lavori per clarinetto. In dialogo, la musica rimane la stessa, ma i diversi registri di entrambi gli strumenti richiedono trasposizioni.

Tra il 1946 e il 1958 Pierre Boulez fu direttore musicale della compagnia teatrale \emph{Renaud-Barrault}. In quel contesto incontrò Paul Claudel. \emph{L’ombra doppia} è il titolo di una scena della \emph{Scarpina di raso} di Claudel alla quale il \emph{Dialogue de l'ombre double} è ispirato, dove un gioco di luci proietta una doppia ombra sullo sfondo della scena. Qui, nell'opera di Boulez, è il clarinetto a sdoppiarsi in un intreccio dialogico tra le due parti.

\begin{quote}
{\small
\textbf{Dialogo} - La parte di uno scritto e, più spesso, di un’opera scenica, narrativa, o di un film, in cui sono introdotti a parlare due o più personaggi. -- \textbf{letteratura} Prescindendo dalle opere sceniche, dove è nel suo proprio luogo, e senza tener conto degli elementi dialogici contenuti nelle liriche, nei poemi, nella prosa narrativa, il d. come genere letterario si può dire nasca con Platone, per l’esigenza di presentare drammaticamente il processo di scoprimento e di conquista della verità, attraverso il contrasto di opposte opinioni. [\ldots] -- \textbf{musica} Composizione vocale (per due o più voci accompagnate) del 16º-17º sec., su testo religioso o profano, in forme dialogica. Il termine indicava anche un componimento per due o più strumenti, in stile concertante, in forme varie, coltivato nel 17º secolo.\footnote{http://www.treccani.it/enciclopedia/dialogo/}}
\end{quote}

La forma del dialogo si sviluppa tra il \emph{clarinette/première} e \emph{clarinette/double} nell'alternanza di \emph{strophe} (il primo) e \emph{transition} (il secondo). Entrambe sono eseguite dallo stesso musicista: il discorso si svolge dal vivo tra il clarinetto sul palco e il suo doppio pre-registrato e diffuso dagli altoparlanti. Alla presenza localizzata del primo, si oppone l'immagine diffusa dell'altra. Nello sviluppo dialogico, le strofe sono focalizzate su singole idee compositive mentre le transizioni si spostano gradualmente da un modello ad un altro.

Non c'è sovrapposizione tra l'esecuzione dal vivo ed il nastro pre-registrato, tranne nei periodi di passaggio, ma  senza creare polifonia, in un gioco di ombre che si succedono proseguendo la dimensione orizzontale del testo del discorso. Ora, un vero dialogo tra due soggetti comporta un corso irreversibile del tempo, mentre nella duplicazione della personalità lo sviluppo temporale, come in una riflessione interiore, il tempo non è lineare ma circolare. Formalmente il sistema è reso ancora più ampio e multi-direzionale dalla forma a due percorsi, identificati rispettivamente con numeri arabi o numeri romani.

%Acronimo finale: l'ombra sussurrò alterna passaggi (full circle) e improvvise interiezioni. Dal momento che il suono iniziale, prima filtrato si sviluppa a poco a poco ad essere trasmesso da tutti e tre gli altoparlanti alti.

In ciascuno dei percorsi l'alternanza tra strumento e ombra sviluppa il discorso musicale, a volte condiviso a volte smembrato fino alla fine, trovandosi d'accordo in un unisono.

Nella stesura formale del brano, il doppio, lo sdoppiamento, si gioca anche sul piano della scelta: l'esecutore può scegliere due percorsi tra strofe e transizioni, indicati il primo con \emph{cifre romane} ed il secondo con \emph{cifre arabe}. %La \emph{Sigla iniziale} e la \emph{Sigla finale} sono comuni ad entrambi

%\thispagestyle{fancy} % All pages have headers and footers

\subsubsection*{Sigla Iniziale}

Entrambe le sigle sono suonate dal \emph{clarinetto/doppio}. La \emph{Sigla iniziale} è giocata interamente sull'articolazione della terzina come nocciolo dell'incedere cellulare. Cardine di questi brevi sviluppi è il \emph{Re3}\footnote{Espresso in suoni reali, un tono sotto alla notazione in partitura \emph{Mi3}} sul quale Boulez sospende ogni periodo, chiudendo le legature e posizionando la sorgente in punti doversi dello spazio. In un gioco di doppi, un omaggio a Berio non si può ritenere concluso senza intricare il racconto con riferimenti intarsiati nel discorso musicale. A sottolineare il legame tra i due compositori ci sono numerosi riferimenti. Uno di questi è proprio il \emph{Re3}, nota polarizzante sia dell'inizio che della fine della \emph{Sequenza IXa} di Berio.
Volendo indugiare nella ricerca di riferimenti, la stessa figura iniziale della terzina è cardinale nel rigo iniziale della \emph{Sequenza}.

L'intero progetto delle \emph{Sequenze} di Berio descrive accuratamente il tentativo di spingere la pratica strumentale verso un'evoluzione si tecnica ma anche mentale, di approccio e ricerca con lo strumentista ed il suo strumento tradizionale. In un modo simile l'amico Boulez sfruttò il rapporto con Alain Damiens, clarinetto dell'Ensemble Inter-Contemporain, impostando il lavoro

\begin{quote}
{\small
in modo molto didattico, spingendo l'interprete fino ai limiti, affinché egli stesso ne verifichi le possibilità.
}
\end{quote}

%%%%%%%%%%%%%%%%%%%%%%%%%%%%%%%%%%%%%%%%%%%%%%%%%%
% STROFE
%%%%%%%%%%%%%%%%%%%%%%%%%%%%%%%%%%%%%%%%%%%%%%%%%%

\subsubsection*{Strofe}

La partitura comprende quindi sei strofe ordinate in modo diverso nel caso che si scelga la numerazione romana o araba.

\begin{minipage}[t]{0.89\columnwidth}%

\begin{table}[H]
%\caption{default}
%\begin{center}
\begin{tabular}{c c c}

I 	& = & 2 \\
II	& = & 4 \\
III	& = & 1 \\
IV	& = & 6 \\
V	& = & 3 \\
VI	& = & 5 \\

\end{tabular}
%\end{center}
%\label{default}
\end{table}%

\end{minipage}%
\bigskip

La \emph{Strofa I(2)} si basa su un processo di scrittura che verrà ampiamente sviluppata durante il lavoro. %Clarinetto Monody è una giustapposizione di cellule alla maniera di un filo di perle. Queste cellule sono polarizzati (+/-), vale a dire che sono fatte da alci e finali (tempo / venduto). Le cellule regolarmente si alternano a cellule mezzo piano Mezzo Forte.

La \emph{Strofa II (4)} inizia in modo molto dolce, quando uno shock nervoso, un'eruzione vulcanica, appena traumatizzare il discorso. Queste pause appaiono sporadicamente, meno prominente. Tra questi, troviamo la sequenza di celle con valori ritmici meno tesa.

La \emph{Strofa III(1)} gioco di scrittura tra fili di suoni, tenuta e attacchi con terminazioni sforzando fuggitive.

La \emph{Strofa IV(6)} la scrittura, tritato sciolsero, opponendosi alla grande aumento nella quinta strofa.

La \emph{Strofa V(3)} la sequenza porta ad un clima nella parte centrale, dove le cellule sono notevolmente allungati nel tempo e suono, esaltata dalla riverberazione del pianoforte e diffusione di altoparlanti. Alla fine di questa stanza, le cellule riprendono le normali proporzioni e riverbero diminuisce.

La \emph{Strofa VI (5)} è la liberazione. Questa stanza si sviluppa alta vocalizzo, estensione vocale flessibile su una raffica. Le oscillazioni di tempo, in fuga instabile. Questa frase culmina verso acuto e stridente tenuta prima di cadere nella seconda parte del versetto, nel registro inferiore ancorato D, porto di partenza del clarinetto in questo pezzo. Fraseggio alterna linea continua e ornamenti flatterzunge.

\subsubsection*{Transizioni}

Attraverso il racconto di Alain Damiens si può ricostruire come l'idea stessa di evoluzione dialogica del brano sia passata negli anni attraverso più fasi compositive.

\begin{quote}
{\small
Il \emph{Dialogue}, nella storia del compositore, deriva da un brano che si chiama \emph{Domaines} che compose nel 1967, quando era professore di composizione al conservatorio. Chiese per esercizio agli allievi di comporre un pezzo per un strumento solista, deliberatamente asettico, con una simbologia numerica. Un brano contemporaneamente aperto e chiuso. In tutto questo si mise lui stesso ad eseguire l'esercizio e ne ricavò una piccola raccolta, il quaderno A, un “domaine” di circa 40 secondi. Poi sviluppò l'idea in 12 raccolte, 6 originali e 6 speculari e poi compose i \emph{Domaines} per gruppi di strumenti sempre da 1 a 6. Poi cambiò di nuovo idea e riscrisse tutto. Prese la prima raccolta ed i sei moduli per ensemble e sviluppò tutto per doppio clarinetto, elaborando la logica con cui in \emph{Domaines} faceva dialogare il solista con il gruppo. Il tutto passando dai 40 secondi iniziali ai 18 minuti.
}
\end{quote}

Il processo di sviluppo graduale del brano riguarda anche l'introduzione dell'elettronica, in quanto
\begin{quote}
{\small
all'inizio prospettava due persone fisiche sulla scena ma questo poneva molti problemi tecnici e quindi ha optato per la registrazione: si suona sul proprio suono.
}
\end{quote}

In termini pratici, il racconto di Damiens ci fornisce i numeri per capire perché le strofe sono identiche per entrambi i percorsi (romano, arabo) ma con ordine diverso, mentre per le transizioni si hanno quattro identità e due particolarità:

\begin{minipage}[t]{0.89\columnwidth}%

\begin{table}[H]
%\caption{default}
%\begin{center}
\begin{tabular}{c c c}

I a II 		& = & 2 a 3 \\
II a III	& = & 4 a 5 \\
III a IV	& = & 1 a 2 \\
IV a V		& = &  \\
V a VI		& = & 3 a 4\\
			& = & 5 a 6 \\

\end{tabular}
%\end{center}
%\label{default}
\end{table}%

\end{minipage}%
\bigskip

Il numero delle transizioni è uguale al numero delle strofe meno uno quindi per ognuno dei due percorsi B. necessitava di cinque transizioni mentre, almeno secondo la testimonianza di Damiens, il processo compositivo di B. forniva blocchi di sei oggetti musicali, esattamente quelli che si hanno se si contano le righe della nostra tabella.

\subsubsection*{Sigla Finale}

Come accade per la \emph{Sigla Iniziale}, anche nella \emph{Sigla Finale} l'ombra, il \emph{clarinetto/doppio}, inizia il suo discorso mormorando frasi in \emph{\textbf{pp}} nella gamma inferiore dello strumento, con improvvise  interpunzioni in \emph{\textbf{ff}} e progressive aperture nel registro acuto. Il tutto avviene in un crescendo dato dall'entrata progressiva di diffusori per poi scendere di nuovo all'inverso, in un processo di graduale chiusura di ciascun altoparlante fino alla fine del brano. Il \emph{clarinetto/primo} inizia a suonare un \emph{Do6} pianissimo  nel momento in cui tutti i diffusori sono aperti e lo porta fino alla fine del brano dove incontra il \emph{doppio} che tocca la stessa nota ma viene diffuso nell'altoparlante esterno al cerchio dei sei, quello dedicato all'effetto di distanza. L'ultimo contatto tra i due quindi segna l'allontanamento dell'ombra sullo stesso suono, ad indicareche c'è ancora un ultimo gesto da compiere, ma non è dialogico bensì teatrale: una uscita di scena, una uscita dallo spazio.


%\section{Citazioni}



\begin{quote}
{\small
[\ldots] si ha la sensazione che si riduca enormemente, con una sensazione di claustrofobia e poi si allarghi e si riempia. C'è un bel rapporto tra il pubblico e l'esecutore, molto differente ma entrambi completamente dentro, immersi, non soltanto assorbiti dalla musica ma anche dalla visione teatrale scenica e poi questa ultima posa di suono finale, questa nota che va tenuta indefinitamente davanti al pubblico, concetto incomprensibile di infinito, la musica continua verso la fine.
}
\end{quote}

\subsubsection*{Spazio Sonoro}

La spazializzazione del nastro pre-registrato può essere eseguita sia manualmente che in maniera automatizzata. Per al procedura automatizzata si può utilizzare il timecode presente sul secondo canale del tape.

Le istruzioni per la spazializzazione tra metodo manuale e automatizzato sono minimamente diversificate, in modo che l'esecuzione automatizzata possa eseguire virtuosismi che manualmente sarebbero impossibili o estremamente complessi.

%----------------------------------------------------------------------------------------
%	INTRODUZIONE
%----------------------------------------------------------------------------------------

\subsubsection{dalle note tecniche, Introduzione}

Dialogue è una composizione per clarinetto dal vivo (\emph{premiere}) e clarinetto pre-registrato (\emph{double}). Il brano inizia con una sezione di \emph{tape} (\emph{Sigle Initial}) e poi prosegue in un'alternarsi di sezioni dal vivo (\emph{strophes}) e sezioni pre-registrate (\emph{transitions}) per concludere con una sezione di \emph{tape} (\emph{Sigle Final}).

Idealmente l'esecutore si dovrebbe posizionare al centro della sala, circondato dal pubblico. Attorno al pubblico è disposto un cerchio di sei altoparlanti attraverso il quale è riprodotto il \emph{tape} con lo strumento pre-registrato.

FIG 1a

Se non fosse possibile posizionare l'esecutore al centro della sala, si può procedere con il posizionamento tradizionale sul palcoscenico.

FIG 1b

Per ottenere un contrasto maggiore tra le sezioni dal vivo e quelle pre-registrate è necessario utilizzare un'illuminazione dedicata all'esecutore durante le \emph{Strophes}.

Il \emph{tape} dovrebbe essere pre-registrato dallo stesso esecutore. Esiste comunque una versione disponibile presso l'IRCAM.

La composizione è disponibile in due versioni chiamate rispettivamente \emph{Version aux chiffres romains} e \emph{Version aux chiffres arabes}

%----------------------------------------------------------------------------------------
%	STROPHES
%----------------------------------------------------------------------------------------

Durante le \emph{Strophes} il clarinetto dal vivo può essere amplificato attraverso un sistema composto da un microfono e due altoparlanti posizionati accanto al musicista. Nel caso di un'acustica troppo asciutta, lo strumento può essere leggermente riverberato. Durante le \emph{Strophes II, III, V} della versione \emph{aux chiffres romains} e le \emph{Strophes 1, 3, 4} della versione \emph{aux chiffres arabes}, il suono del clarinetto dal vivo è trasformato attraverso un pianoforte nel quale il pedale destro è tenuto abbassato così che le corde possano vibrare liberamente.

FIG 2

%----------------------------------------------------------------------------------------
%	TRANSITIONS
%----------------------------------------------------------------------------------------

La registrazione monofonica utilizzata per le \emph{transitions} viene riprodotta attraverso sei altoparlanti equidistanti tra loro (1--6) che circondano il pubblico e un altoparlante fuori dal cerchio (7).

\begin{quote}
{\small
Questo ultimo altoparlante, utilizzato solo alla fine della \emph{Sigle Final}, deve suonare essere distante e può essere anche posizionato fuori dalla sala da concerto.


Seguendo i punti di riferimento scritti in partitura, il clarinetto pre-registrato viene inviato tra un altoparlante e un altro dando all'ascoltatore l'impressione che il suono si muova nello spazio fisico.

D'ora ci riferiremo a questo col termine di “spazializzazione”
}

\end{quote}

Nel \emph{Dialogue} esistono due tipi di spazializzazione:

\begin{description}
\item[continuo] il suono si muove morbidamente tra gli altoparlanti
\item[discreto] il suono si muove bruscamente da un altoparlante all'altro
\end{description}

La spazializzazione può avvenire in due modi:

\begin{description}
\item[manualmente] il livello di ogni altoparlante è controllabile individualmente dai potenziometri di una piccola superficie di mixaggio
\item[automatizzata] la seconda traccia del \emph{tape} può contenere un \emph{timecode SMPTE} inviabile ad un decoder che controlli un sistema automatico in sincronismo con i riferimenti in partitura.

\end{description}

FIG 3

Confrontando le istruzioni manuali e automatizzate per la spazializzazione si può notare che sono praticamente le stesse (anche se spesso annotate in modi leggermente diversi) per quasi tutte le transizioni, fatta eccezione della \emph{Sigle Initial} (sia per la versione \emph{aux chiffres romains} che per la \emph{aux chiffres arabes} -- vedi tabella \ref{tab:diffsiginit}) e della \emph{Transition de IV à V} (quindi solo per la versione \emph{aux chiffres romains} -- vedi tabella \ref{tab:diffiv-v}). Nella \emph{Transition de IV à V} si hanno due procedure di spazializzazione diversificate, una più semplice realizzabile manualmente, una più complessa e virtuosistica automatizzata.

Le indicazioni dinamiche per gli altoparlanti vengono fornite in termini musicali: quando un altoparlante è acceso deve suonare tra il \emph{mezzoforte} e il \emph{forte}. Nella \emph{Transition de I à II} per esempio gli altoparlanti vengono modulati due volte consecutivamente, prima in \emph{forte} poi in \emph{mezzo-piano}. Questi due differenti livelli dinamici vengono determinati in prova e devono descrivere un chiaro effetto di gioco tra primo piano e sfondo. I tempi di crescita verso il \emph{forte} “il più veloce possibile”, mentre il decadimento al \emph{mezzo-piano} deve durare $ 0.5 sec $. Anche le indicazioni di modulazione tra un valore dinamico e l'altro sono espresse nei termini musicali di \emph{crescendo} e \emph{diminuendo}.




\begin{table*}[h]

\caption{confronto tra spazializzazione manuale ed automatizzata SIGLE INITIAL, uguale sia per le cifre romane che per le cifre arabe}
\begin{center}
%\begin{sf}{\footnotesize
\begin{tabular}{c c c c c}

\hline
\multicolumn{5}{c}{\emph{\textbf{Sigle Initial}}} \\
\hline

	&
\multicolumn{2}{c}{\textbf{Spazializzazione Manuale}} &
\multicolumn{2}{c}{\textbf{Spazializzazione Automatica}} \\

\hline

\textbf{cue number} 	& \textbf{speaker(s) ON}	& \textbf{speaker(s) OFF} 		& \textbf{speaker(s) ON} 		& \textbf{speaker(s) OFF} 		\\
1			&	1			&	-					&	1					&	2, 3, 4, 5, 6 		\\
2			&	3			&	1					&	3					&	1, 2, 4, 5, 6 		\\
3			&	5			&	3					&	5					&	1, 2, 3, 4, 6 		\\
4			&	2			&	5					&	2					&	1, 3, 4, 5, 6 		\\
5			&	5			&	2					&	5					&	1, 2, 3, 4, 6 		\\
6			& 	4			&	5					&	4					&	1, 2, 3, 5, 6 		\\
7			&	6			&	6					&	6					&	1, 2, 3, 4, 5		\\
8			&	3			&	6					&	3					&	1, 2, 4, 5, 6 		\\
9			&	6			&	-					&	3, 6				&	1, 2, 4, 5 			\\
10			&	2, 5		&	3, 6				&	2, 5				&	1, 3, 4, 6 			\\
11			&	4		 	&	5					&	2, 4				&	1, 3, 5, 6 			\\
12			&	1, 6		&	2, 4				&	1, 6				&	2, 3, 4, 5 			\\
13			&	2			&	6					&	1, 2				&	3, 4, 5, 6 			\\
14			&	4, 5		&	1, 2				&	4, 5				&	1, 2, 3, 6 			\\
15			&	3, 6		&	4, 5				&	3, 6				&	1, 2, 4, 5 			\\
16			&	2			&	6					&	2, 3				&	1, 4, 5, 6 			\\
17			&	4			&	-					&	2, 3, 4 			&	1, 5, 6 			\\
18			&	5			&	3					&	2, 4, 5 			&	1, 3, 6 			\\
19			&	1			&	4					&	1, 2, 5 			&	3, 4, 6 			\\
20			&	6			&	2					&	1, 5, 6 			&	2, 3, 4 			\\
21			&	-			&	1, 5				&	6 					&	1, 2, 3, 4, 5 		\\
22			&	4			&	-					&	4, 6 				&	1, 2, 3, 5			\\
23			&	1			&	-					&	1, 4, 6 			&	2, 3, 5 			\\
24			&	3			&	-					&	1, 3, 4, 6 			&	2, 5 				\\
25			&	2			&	-					&	1, 2, 3, 4, 6 		&	5 					\\
26			&	5			&	-					&	1, 2, 3, 4, 5, 6	&	- 					\\
27			&	-			&	1, 2, 3, 4, 5, 6	&	-					&	1, 2, 3, 4, 5, 6	\\

\end{tabular}
%}\end{sf}
\end{center}
\label{tab:diffsiginit}

\end{table*}

\bigskip

\begin{table*}[h]

\caption{confronto tra spazializzazione manuale ed automatizzata TRANSITION DE IV A V}
\begin{center}
%\begin{sf}{\footnotesize
\begin{tabular}{c c c c c}

\hline
\multicolumn{5}{c}{\emph{\textbf{Transition de IV à V}}} \\
\hline

	&
\multicolumn{1}{c}{\textbf{Spazializzazione Manuale}} &
\multicolumn{3}{c}{\textbf{Spazializzazione Automatica}} \\

\hline

\textbf{cue number}		&
\textbf{speaker(s) ON }	&
\textbf{speaker(s) ON}	&
\textbf{speaker(s) ON}	&
\textbf{speaker(s) OFF}	\\

&
&
\emph{forte} &
\emph{mezzo-piano} &
\\

% cue %	manuale ON			% auto on forte			% auto on mp
1	&	3					&	3					&	-	&	-	\\
2	&	6					&	1					&	3	&	-	\\
3	&	3					&	6					&	1	&	3	\\
4	&	5					&	3					&	6	&	1	\\
5	&	1					&	5					&	3	&	6	\\
6	& 	4					&	1					&	5	&	3	\\
7	&	5					&	5					&	1	&	-	\\
8	&	2					&	4					&	5	&	1	\\
9	&	1					&	3					&	4	&	5	\\
10	&	6					&	5					&	3	&	4	\\
11	&	3				 	&	2					&	5	&	3	\\
12	&	4					&	6					&	2	&	5	\\
13	&	2					&	1					&	6	&	2	\\
14	&	5					&	3					&	1	&	6	\\
15	&	3					&	2					&	3	&	1	\\
16	&	6					&	6					&	2	&	3	\\
17	&	1					&	3					&	6	&	2	\\
18	&	3					&	4					&	3	&	6	\\
19	&	3, 1				&	5					&	4	&	3	\\
20	&	3, 1, 2				&	4					&	5	&	-	\\
21	&	3, 1, 2, 4			&	1					&	4	&	5	\\
22	&	3, 1, 2, 4, 6		&	2					&	1	&	4	\\
23	&	3, 1, 2, 4, 6, 5	&	5					&	2	&	1	\\
24	&	-					&	4					&	5	&	2	\\
25	&						&	3					&	4	&	5	\\
26	&						&	6					&	3	&	4	\\
27	&						&	5					&	6	&	3	\\
28	&						&	3					&	5	&	6	\\
29	&						&	6					&	3	&	5	\\
30	&						&	1					&	6	&	3	\\
31	&						&	2					&	1	&	6	\\
32	&						&	1					&	2	&	-	\\
33	&						&	5					&	1	&	2	\\
34	&						&	6					&	5	&	1	\\
35	&						&	4					&	6	&	5	\\
36	&						&	3					&	4	&	6	\\
37	&						&	1, 3				&	-	&	4	\\
38	&						&	1, 2, 3				&	-	&	-	\\
39	&						&	1, 2, 3, 4			&	-	&	-	\\
40	&						&	1, 2, 3, 4, 6		&	-	&	-	\\
41	&						&	1, 2, 3, 4, 5, 6	&	-	&	-	\\
42	&						&	-					&	-	&	1, 2, 3, 4, 5, 6	\\

\end{tabular}
%}\end{sf}
\end{center}
\label{tab:diffiv-v}

\end{table*}

Nella \emph{transition de III à IV} ci sono descritti movimenti circolari in accelerando prima di fermarsi immobile in un solo altoparlante. In riferimento a questo, nel manuale tecnico si legge una nota importante:

\begin{quote}
{\small la massima velocità di rotazione non deve mai dare l'impressione all'ascoltatore di immobilità o suono proveniente da tutti gli altoparlanti contemporaneamente.
}
\end{quote}

Nella \emph{Transition de I à II} si hanno due distinti livelli dinamici, il \emph{mezzo-piano} e il \emph{forte}. L'alternanza tra questi due livelli avviene tenendo un gruppo di cinque altoparlanti sul livello di \emph{mezzo-piano} e facendo uscire dal gruppo un solo diffusore con dinamica \emph{forte}. Questo movimento dinamico-spaziale crea delle emergenze del singolo altoparlante dallo sfondo creato dagli altri.

Anche nella \emph{Transition de V à VI} ci sono due livelli dinamici che si rapportano dialogicamente nel gioco figura/sfondo e sono il \emph{mezzo-piano} ed il \emph{forte}. In questo caso però il \emph{mezzo-piano} è ottenuto da un solo altoparlante, mentre il \emph{forte} avviene con un \emph{tutti}. L'alternanza qui produce un effetto contrario a quello precedentemente descritto.

\subsubsection{Luce e ombra}

L'illuminazione, oltre ad essere un retaggio dell'ispirazione teatrale, ha lo scopo di accentuare il contrasto tra lo strumento dal vivo e quello pre-registrato. In generale, durante le sezioni pre-registrate, il pubblico e l'esecutore sono al buio, mentre durante l'esecuzione dal vivo del \emph{clarinette/premiere} il clarinettista viene illuminato.

FIG. SCHEMA ILLUMINAZIONE.

\subsubsection{Stage}

Sono previste due disposizioni, una con il clarinetto al centro della sala, in mezzo al pubblico circondato dagli altoparlanti. L'altro con il clarinetto sul palco, separato dal pubblico.

FIG. DESCRITTIVE.

%\section{Related Names bla bla}
%
%Incisione utilizzata: Alain Damiens, Ensemble intercontemporain, 1 cd Deutsche Grammophon, nº 451 603-2.
%
%\subsection{Andrew Gerzso}
%Born in Mexico City, Andrew GERZSO studied flute and composition at the New England Conservatory in Boston, California Institute of the Arts in Los Angeles and the Royal Conservatory in The Hague.
%
%As a member of IRCAM's permanent staff since 1977 he has held over the years a number of positions: researcher, Technical Director, Director of Musical Research, Director of the Production Department and Manager of the IRCAM Forum, the institute's software user group. Since 2002 he is the director of the pedagogical department, coordinator of the Pôle Spectacle and is in charge of organizing the interaction between the artistic and scientific sectors of the institute. He has published articles on computer music in journals such as La Recherche, Pour la Science, Scientific American and Leonardo.
%
%Since 1980 he has been a close collaborator of Pierre BOULEZ at IRCAM (for whom he did the electro-acoustic realization for Répons in 1981, Dialogue de l'Ombre Double in 1985, Explosante-fixe in 1991 and Anthèmes 2 in 1997) and at the College de France (for the annual seminars until 1995). The Deutsche Grammophon recordings of Explosante-fixe and Répons received Grammy awards in 1996 and 1999 respectively.
%
%Andrew Gerzso has been a member of IRCAM’s permanent staff since 1977 holding a number of positions beginning as researcher and leading up to his current position as Director of the Department for the Coordination of Scientific and Musical Research. The department manages musical research, the IRCAM Forum (the institute’s software user group) and several documentation projects. Since 1980 he has been a close collaborator of Pierre Boulez at IRCAM (for whom he did the electro-acoustic realization for “Répons” in 1981, “Dialogue de l’Ombre Double” in 1985, “Explosante-fixe” in 1991 and “Anthèmes 2” in 1997) and at the College de France (for the annual seminars until 1995). He is the coordinator of the European project CO-ME-DI-A that explores the use of high speed networks in music.
%
%\subsection{Pascal Gallois}
%Pascal Gallois studied with Maurice Allard and won unanimously the First Prize of bassoon at the Conservatoire de Paris (Paris Conservatoire), where he teaches in turn from 1994 to 2000. Appointed professor of bassoon at the Hochschule für Musik und Theater in Zurich, he also taught at Darmstadt since 2002 (International Musikinstitut Darmstadt). Pedagogy and repertoire development are two fundamental axes of the outreach work he is conducting at the Ensemble Intercontemporain, which he joined in 1981. In 1984, he gave the French premiere of In Freundschaft for solo bassoon, Karlheinz Stockhausen. Many composers have written for him, including Gyorgy Kurtag, Olga Neuwirth, Philippe Fenelon, Brice Pauset and Luciano Berio, who dedicated his Sequenza XII in 1995. The same year he created the version for bassoon of “Dialogue de l’ Ombre double’ by Pierre Boulez. His record Pascal Gallois Dialogues (Stradivarius, 2003) received the “Coup-de-coeur de l’ Académie Charles Cros” and “Choc du Monde de la Musique”

%\bibliography{references}

%\end{multicols}

%\end{document}
