%!TEX TS-program = xelatex
%!TEX encoding = UTF-8 Unicode
% !TEX root = ../metm.tex

\chapter{IL REPERTORIO}
\startcontents[chapters]
\printcontents[chapters]{}{1}{}

ciao\footcite[\emph{idem}]{prieberg:mexm}

Agamben\index{Agamben, Giorgio}

\section{Luigi Nono}

NONO, Luigi (Venezia, 29.1.1924 – Venezia, 8.5.1990)

Nato il 29 gennaio 1924 a Venezia, primogenito di Mario Nono e Maria Manetti. Nono ebbe già nell’ambito familiare i primi stimoli per la sua formazione artistica e culturale: il nonno paterno, Luigi Nono, era un noto pittore della scuola veneziana di fine Ottocento; il prozio Urbano (fratello del pittore) uno scultore; la nonna materna, discendente dell’antica famiglia veneziana Priuli Bon, si dilettava col pianoforte e col canto spaziando dalla musica del passato alla produzione liederistica più recente (Nono ricorderà con stupore di aver rinvenuto tra i suoi spartiti una delle prime edizioni degli Italienische Lieder di Hugo Wolf accanto al Montezuma di Sacchini; Nono 1987, p. 480). La madre e il padre, di professione ingegnere, erano pianisti dilettanti che amavano cimentarsi con brani di repertorio (tra questi, il Boris Godunov di Modest Musorsgkij, spesso ricordato tra i primi ascolti nella sua fanciullezza; ibid.). Frequentatori assidui del Teatro La Fenice e di varie rassegne concertistiche, Mario e Maria Nono erano inseriti nei circoli culturali e musicali della migliore società veneziana. Grazie alla fornita discoteca del padre, Nono poté conoscere presto la musica di Beethoven, Wagner, Mahler nelle prime incisioni di direttori quali Toscanini o Mengelberg. Di non minore importanza furono le prime erratiche letture condotte sui libri della imponente biblioteca paterna (oggi in parte conservati nel lascito del compositore), dalle prime traduzioni italiane di poeti e scrittori russi, ai romanzieri americani pubblicati all’epoca da Einaudi, a Pavese, Gogol’, Rilke e vari altri autori che riaffioreranno nel corso degli anni tra le selezioni testuali delle proprie opere. In questa sfaccettata e privilegiata realtà domestica è possibile ravvisare l’origine di quella che diventerà una cifra caratteristica dell’universo artistico noniano, proiettato verso un’idea (e una pratica) della musica intesa come un’arte senza frontiere che può trovare ispirazione e fondamento in varie manifestazioni artistiche e scientifiche (pittura, architettura, letteratura, poesia, filosofia ecc.) delle diverse epoche storiche.

Intorno ai dodici anni Nono intraprese lo studio del pianoforte con una docente privata amica della madre (la signora Alessandri) e, fin da ragazzo, cominciò ad assistere agli spettacoli della Fenice e del Festival internazionale di musica contemporanea della Biennale di Venezia. Svolse i suoi studi presso il Ginnasio-Liceo classico «Marco Polo» di Venezia dove conseguì la maturità nel 1942. Nello stesso anno, per assecondare i desideri del padre, preoccupato delle incertezze della professione musicale, Nono si iscrisse alla Facoltà di Giurisprudenza dell’Università di Padova. Sempre nel 1942 egli conobbe il giovane pittore Emilio Vedova e intrecciò con lui un’amicizia che, rafforzatasi nel tempo anche grazie a varie collaborazioni artistiche, durò a fasi alterne fino alla morte del compositore.

La formazione scolastica e musicale di Nono si svolse negli anni cruciali del secondo conflitto mondiale e dell’immediato dopoguerra, in un clima familiare e intellettuale di stampo tradizionale e borghese ma perlopiù ostile al fascismo. Per motivi di salute egli fu dispensato dal servizio militare e non partecipò attivamente né alla guerra né alle successive fasi della Resistenza. Nono privilegiò nondimeno in quegli anni il contatto con giovani socialisti veneziani e personalità dell’opposizione locale, coltivando ideali politici e culturali non allineati con il regime.

Nel 1941, all’età di 17 anni, il padre gli procurò un incontro con una delle maggiori personalità musicali del tempo, Gian Francesco Malipiero, compositore che gli «aprì tutti gli orizzonti della musica» (Nono 1961, p. 3). Per qualche anno Nono seguì come studente esterno i suoi corsi di Composizione presso il Conservatorio di musica Benedetto Marcello di Venezia; dal 1943, dopo il ritiro di Malipiero dall’insegnamento di composizione, continuò a frequentare sempre da esterno la classe di contrappunto e fuga di Raffaele Cumar (ex allievo di Malipiero e Dallapiccola), approfondendo contemporaneamente in privato lo studio del pianoforte con Gino Gorini. In un periodo segnato culturalmente da una progressiva chiusura verso le esperienze d’avanguardia sviluppatesi in Europa dagli inizi del XX secolo, gli orizzonti aperti da Malipiero riguardavano lo studio di Monteverdi e della grande tradizione rinascimentale italiana (polifonica e madrigalistica), dei trattati teorici di Zarlino, Gaffurio e Vicentino, nonché la scoperta della musica di Arnold Schönberg, Anton Webern e Béla Bartók. Per la sua indole curiosa e irrequieta, Nono non riuscì mai ad adattarsi a piani di studio imposti e, ben presto, cominciò a nutrire una vera avversione per vincoli formativi legati a programmi ministeriali reputati noiosi e spesso inutili. Dopo il conseguimento del compimento inferiore e medio di composizione – sostenuti rispettivamente nel 1947 e 1949 presso il Conservatorio «Benedetto Marcello» – egli non reputò necessario coronare la sua formazione musicale con il diploma.

Nel 1946, grazie a Malipiero, Nono entrò in contatto con il giovane compositore e direttore d’orchestra veneziano Bruno Maderna, di soli quattro anni più anziano, già noto per il suo passato di fanciullo prodigio. All’incirca nello stesso periodo conobbe inoltre Luigi Dallapiccola, tra i musicisti più stimati e punto di riferimento per molti giovani della sua generazione. Delle prove compositive portate a termine prima del fondamentale incontro con Maderna resta solo la labile traccia di un ricordo (Nono 1979-80): nel 1945, durante la frequentazione di Malipiero – e influenzato anche dai suoi lunghi discorsi sulla musica del XV e XVI secolo –, Nono compose La discesa di Cristo agli inferi, brano (in seguito smarrito) forgiato sul modello delle sacre rappresentazioni e ispirato a un linguaggio di stampo monteverdiano (ivi, p. 242). Lo stimolo per il primo decisivo mutamento di rotta, dopo circa otto anni di studi musicali reputati parziali e insufficienti, sembrerebbe esser stato fornito proprio da un parere critico di Dallapiccola sulla partitura de La discesa di Cristo agli inferi, speditagli dal giovane compositore su suggerimento di Malipiero: «capisco che lei ha qui dentro nel cuore molto da esprimere, soltanto deve studiare ancora molto per poterlo esprimere» (ibid.). Queste parole costituirono per Nono una spinta a ricominciare gli studi musicali pressoché da zero in un nuovo periodo di apprendistato con Maderna. Conclusi gli studi universitari – coronati nel 1947 da una laurea in giurisprudenza con una tesi sul concetto giuridico dell’exceptio veritatis – egli poté dedicarsi solo alla musica e indirizzare il suo apprendistato verso un sentiero di conoscenza autonomo e responsabile, condotto al di fuori di ogni istituzione scolastica o accademica. (Di questo periodo 1946-47 non sono note composizioni al di fuori di un progetto vocale su liriche tratte da L’allegria di Giuseppe Ungaretti, poeta da tempo amato e ammirato.)

L’incontro con Maderna segnò in modo indelebile lo sviluppo musicale e umano di Nono, che riconobbe fino alla fine all’amico il ruolo di primo vero e grande maestro di vita.

Durante lunghe giornate trascorse tra l’abitazione di Maderna e la Biblioteca Marciana di Venezia, Nono approfondì – insieme ad altri giovani raccolti intorno al giovane maestro (Romolo Grano, Gastone Fabris e Renzo Dall’Oglio tra questi) – lo studio della musica del XV-XVI secolo, dall’ars antiqua all’ars nova francese, dai fiamminghi alla polifonia rinascimentale italiana, in un continuo confronto tra teoria e pratica. Tra gli autori più amati e analizzati: Guillaume de Machaut, John Dunstable, Johannes Ockeghem, Josquin Desprès, Adrian Willaert, Andrea e Giovanni Gabrieli, spesso trascritti a partire dall’Odhecaton di Ottaviano Petrucci. L’analisi di questo periodo storico divenne – tra la fine degli anni Quaranta e l’inizio del decennio successivo – lo stimolo per un’indagine comparata dei vari processi compositivi nella storia: data la comprensione di una tecnica musicale, il fine era scoprirne la funzione in relazione al momento storico, ricercandone nuove possibili trasformazioni o applicazioni nella musica delle epoche successive fino alla contemporaneità. Fondamentale importanza assunse per Nono l’analisi del rapporto tra lo stato del materiale, la sua elaborazione e l’epoca di produzione: fu grazie a questa peculiare ricerca che il compositore maturò la convinzione – in seguito mai abbandonata – che il linguaggio artistico deve svilupparsi di pari passo con i grandi movimenti politici e sociali del tempo, arrivando a essere una possibilità (o un mezzo) per poter intervenire all’interno di essi. Conoscere la musica del passato divenne, nel cenacolo maderniano, un mezzo per conoscere e impegnarsi responsabilmente nel proprio presente.


Fu sempre Maderna a far scoprire a Nono il manuale di tecnica compositiva di Hindemith (Unterweisung im Tonsatz, 1a ed. 1937) e, con esso, a suggergli – prima del comune approdo alla scrittura seriale – l’esistenza di soluzioni concrete e alternative nei confronti di un linguaggio armonico in crisi ormai da decenni. Nel 1948, ancora su suggerimento di Malipiero, Nono frequentò insieme a Maderna un corso internazionale di direzione d’orchestra tenuto a Venezia da Hermann Scherchen. Questo nuovo importante incontro segnò l’inizio di un lungo sodalizio (culturale e umano) tra i due giovani compositori e l’anziano direttore che, per circa cinque anni, divenne il loro fondamentale punto di riferimento. Nono cominciò a seguire Scherchen durante i suoi concerti, a soggiornare per lunghi periodi presso di lui (a Zurigo, Rapallo, Gravesano), e a collaborare dapprima come copista, poi come autore, con la sua casa editrice Ars Viva Verlag (acquisita negli anni Cinquanta dalla B. Schott’s Söhne di Magonza). Grazie a Scherchen, attraverso le sue esecuzioni e i suoi racconti, Nono entrò idealmente in stretto contatto con le esperienze – musicali e non – vissute dal direttore in Germania fin dal 1912, dalle prime esecuzioni assolute degli amati Schönberg e Webern alla realtà sociale e culturale tedesca precedente all’avvento del nazismo. Tra il 1948 e il 1949, su impulso di Scherchen Nono compose le Due liriche greche (inedite, formate da La stella mattutina e Ai Dioscuri, su testi rispettivamente di Ione di Ceo e Alceo nella traduzione italiana di Salvatore Quasimodo), dichiaratamente ispirate ai Canti di prigionia di Dallapiccola. Sempre dall’anziano direttore ricevette ulteriori stimoli allo studio della musica di Bach, Beethoven, Schumann e, soprattutto, del metodo dodecafonico, approfondito grazie all’analisi dei differenti approcci di Schönberg, Webern e Dallapiccola.

Nel 1950, su segnalazione di Scherchen e Maderna, Nono frequentò per la prima volta gli Internationale Ferienkurse für Neue Musik di Darmstadt – corsi estivi di musica contemporanea, fondamentale punto di incontro e di confronto per i giovani musicisti della generazione postbellica – debuttando sulla scena internazionale con il suo primo brano per orchestra, le Variazioni canoniche sulla serie dell’op. 41 di Arnold Schönberg (1949-50). Diretta da Scherchen, l’opera fu accolta in modo non unanime ma, al contempo, rivelò la centralità del giovane compositore nel contesto delle problematiche e delle discussioni dell’avanguardia musicale coeva. L’esperienza di Darmstadt – luogo frequentato ininterrottamente per dieci anni (dal 1957 come docente) – costituì un momento di fondamentale importanza nella sua evoluzione artistica, umana e politica. Fu qui che Nono ebbe modo di approfondire ulteriormente la musica dodecafonica e, in particolare, quella di Schönberg; di conoscere Edgard Varèse (tra gli estimatori delle Variazioni canoniche in occasione della contrastata prima esecuzione) e approcciarsi alla sua musica visionaria che tanta importanza ebbe nelle successive evoluzioni del pensiero sonoro noniano. In questa sede egli instaurò importanti rapporti – alimentati dalla consentaneità così come dal dissenso – con vari musicisti europei ed extraeuropei tra cui Karlheinz Stockhausen, Pierre Boulez, Henri Pousseur, John Cage. Soprattutto, fu a Darmstadt che si ebbero le prime esecuzioni assolute di alcune tra le più importanti pagine degli anni Cinquanta: alle già citate Variazioni canoniche seguirono infatti Polifonica-Monodia-Ritmica (1951), l’Epitaffio per Federico García Lorca I. España en el corazón (1952), La victoire de Guernica (1954), Incontri (1955), Cori di Didone (1958) e, ancora, Composizione per orchestra n. 2: Diario Polacco ’58 (1959).

Sono, queste, le opere che rivelarono Nono come uno dei maggiori rappresentanti dell’avanguardia europea e del linguaggio seriale, insieme a Stockhausen e Boulez. Da questi stessi compositori, e dall’ambiente dei Ferienkurse, egli prese nondimeno le distanze in modo dichiarato nel 1959, allorché con la conferenza Geschichte und Gegenwart in der Musik von heute (reso in italiano come Presenza storica nella musica d’oggi) polemizzò apertamente contro alcuni rappresentati dell’avanguardia e con la cosiddetta «Scuola di Darmstadt» (espressione da lui stesso coniata, cfr. Nono 1957, p. 34), denunciandone incoerenze e aporie. Dopo anni in cui gli stimoli si erano intrecciati a costruttivi conflitti, questo intervento sancì una prima frattura con il luogo e con alcuni dei suoi rappresentanti (primo tra tutti Stockhausen). L’inconciliabilità delle posizioni riguardava tanto alcune derive iperdeterministiche dei seguaci della serialità integrale e il loro rifarsi a modelli ricavati dalle scienze naturali, quanto esperienze di segno opposto riconducibili all’alea e all’indeterminazione (Cage fu apertamente nominato come pars pro toto). Entrambe le tendenze furono additate da Nono come fuga dalla storia e sintomo di una irresponsabile volontà di evitare chiare prese di posizione nei confronti di problematiche artistiche del proprio presente. La rottura, definitiva, fu quindi sancita nel 1960 con la nuova conferenza Text – Musik – Gesang (pubblicata in italiano come Testo – musica – canto), dove la distanza dalle posizioni di Stockhausen e da un’ortodossia seriale ormai vissuta come gabbia assunse infine i toni dell’aperto attacco.

Nel 1954, in occasione della prima rappresentazione del Moses und Aron di Schönberg, Nono conobbe ad Amburgo la figlia del grande compositore austriaco, Nuria, che sposò l’anno successivo. Dal matrimonio nacquero due figlie, Silvia (1959) e Serena Bastiana (1964). Tra la fine del 1958 e gli inizi del 1960 Nono compì inoltre la sua unica (nonché atipica) esperienza di docente con Helmut Lachenmann che, per lunghi periodi, soggiornò a Venezia collaborando con Nono alla formulazione tedesca delle due menzionate conferenze tenute ai Ferienkurse di Darmstadt.

Negli stessi anni Cinquanta furono di grande importanza la scoperta o l’approfondimento delle esperienze politiche e culturali d’oltralpe, della Rivoluzione sovietica e della cultura della Repubblica di Weimar, delle avanguardie storiche russe e tedesche e delle innovazioni in campo teatrale di Vsevolod Mejerchol’d, Vladimir Majakovskij, Erwin Piscator. Sollecitazioni, queste, che vennero ad aggiungersi al parallelo entusiasmo per la rivelazione dell’insegnamento di Antonio Gramsci, del pensiero filosofico di Jean-Paul Sartre, di una produzione poetica ancorata alle problematiche del proprio tempo e rappresentata da autori quali Federico García Lorca, Pablo Neruda, Paul Éluard, Cesare Pavese, Giuseppe Ungaretti.

È soprattutto da questi scrittori che Nono seleziona i testi per le proprie opere vocali degli anni Cinquanta, decennio al cui centro campeggia uno dei suoi capolavori, Il canto sospeso (1955-56), basato su lettere di condannati a morte della Resistenza europea. Abbandonato l’uso di materiali ritmici preesistenti – proprio alle sue pagine composte tra il 1950 e il 1953 – e approfondita quella personale elaborazione della tecnica seriale già avviata in campo strumentale con Canti per 13 (1955), Nono perfezionò le proprie conquiste compositive nel campo della vocalità: con Il canto sospeso approdò a una nuova pratica basata su una peculiare tecnica di frammentazione del testo, enunciato nelle sue componenti vocaliche o fonetiche in successione tra le singole voci o simultaneamente, come blocco o aggregato sonoro.

Come testimoniano anche vari scritti teorici redatti nel corso degli anni Cinquanta, è in questo periodo che si rafforzarono in Nono le proprie idee sulla capacità comunicativa della musica e sulla necessità di poter e dover esprimere attraverso la propria arte le sfaccettate contraddizioni del proprio tempo. Gradualmente la selezione dei testi fu sempre più orientata verso temi politicamente impegnati tratti dall’attualità storico-sociale del presente o dell’immediato passato. Questo aspetto si rivelò in tutta la sua evidenza a partire dall’azione scenica Intolleranza 1960 (1960-61) – opera in cui si concretizzarono per la prima volta alcune idee su un “nuovo teatro musicale” maturate nel corso degli anni Cinquanta – e giunse al culmine nella prima metà degli anni Settanta con la seconda azione composta per le scene, Al gran sole carico d’amore (1972-74, rev. 1977).

L’esperienza teatrale di Nono si era inizialmente nutrita di un sentimento di rifiuto – comune a diversi giovani compositori dell’avanguardia post-bellica – nei confronti dei modelli operistici fin de siècle, negazione in cui si rispecchiava un aperto dissenso verso l’organizzazione della società borghese. Fin dai primi progetti drammaturgici tracciati nel corso degli anni Cinquanta – tra i quali vari soggetti mai realizzati su testi di John Steinbeck, Anna Seghers e Anna Frank –, il teatro fu inteso come un luogo in cui temi attuali avrebbero dovuto essere rappresentati con mezzi espressivi e scenotecnici altrettanto originali. La ricerca di Nono in ambito scenico si concentrò in quegli anni sulle esperienze teatrali (soprattutto non musicali) del primo ventennio del Novecento. Come era avvenuto con lo studio della musica del passato condotto sotto la guida di Maderna, anche l’approdo al teatro musicale fu edificato sulla base di un’indagine storica atta ad approfondire tutte quelle esperienze artistiche che, proprio per essere state condannate o represse da regimi diversi, apparivano come modelli ancora potenzialmente attuali. Le letture condotte da Nono nel corso degli anni Cinquanta sul tema «teatro» furono numerose e discontinue: dalle esperienze di Gropius e del Bauhaus commentate da Giulio Carlo Argan (Einaudi 1951) alla storia del teatro di Baty-Chavance (Einaudi 1951), dai libri sul teatro russo e tedesco dei primi anni del Novecento (tra cui il fondamentale testo sul teatro politico di Piscator, 1929) a volumi su e di Majakovskij, Brecht, ecc. L’analisi parallela dei suoi approfondimenti teorici e dei suoi primi progetti drammaturgici sembra suggerire che già dai primi mesi del 1952 il compositore avesse posto le basi di quella personale riflessione sulle implicazioni tecniche e sulle possibilità di una funzione politica del teatro che, nel 1960-61, condusse alla prima azione scenica, Intolleranza 1960 (su testo proprio, rielaborato dal compositore a partire da un’idea di Angelo Maria Ripellino, con passi tratti da Alleg, Brecht, Éluard, Fučik, Majakovskij, Sartre e dallo stesso Ripellino; realizzato scenicamente con la collaborazione di Josef Svoboda ed Emilio Vedova). Soprattutto a causa dei suoi contenuti politici, in occasione della sua prima messa in scena (Venezia, 1961) l’opera scatenò violente contestazioni in sala e, nel panorama musicale dell’epoca, costituì un vero evento che divise il giudizio di pubblico e critica. Pressoché ignorata nella sua prima ricezione fu invece la portata innovativa dei suoi contenuti musicali: Intolleranza 1960 si poneva infatti come un’opera spartiacque, al contempo momento di sintesi e laboratorio sperimentale in cui tecniche ormai acquisite si affiancavano a nuove procedure compositive proiettate verso il futuro. In essa convergevano tutte le maggiori conquiste tecnico-linguistiche che avevano inaugurato gli anni Sessanta. Tra queste, la nuova tecnica vocale della “linea unica” sperimentata nei brani a cappella Sarà dolce tacere e «Ha venido». Canciones para Silvia (entrambe del 1960) e, tra le conquiste più importanti in prospettiva futura, l’uso del nastro magnetico e degli strumenti di produzione elettroacustica del suono. Sempre al 1960 data infatti la prima composizione elettronica di Nono, Omaggio a Emilio Vedova, realizzata presso lo Studio di Fonologia della RAI di Milano, laboratorio elettronico dove per diciannove anni, fino al 1979, egli produsse tutte le sue opere per/con nastro magnetico. A partire da questo momento, l’esperienza elettronica divenne una costante dell’itinerario creativo di Nono, che usò il mezzo tecnologico come nuova frontiera per esprimersi artisticamente in modo sempre più libero e immediato, sperimentando di volta in volta soluzioni sonore e spaziali non ottenibili con una liuteria tradizionale e, soprattutto, non più classificabili in generi musicali codificati. Fu anche grazie all’elettronica che Nono dismise gradualmente sistemi compositivi rigidamente organizzati, quali griglie seriali e di regolazione statistica dei parametri musicali (proprie alle composizioni degli anni Cinquanta), privilegiando sempre più l’organizzazione degli eventi sonori in strutture locali e l’elemento intervallare in luogo di quello ritmico. Svincolata da serie e da un preordinato controllo delle altezze, la scelta degli intervalli divenne per Nono sempre più “intuitiva”, definita localmente e proiettata verso la giustapposizione o sovrapposizione di superfici di suono complesse (blocchi, fasce, linee, ecc.).

L’impegno politico, i temi di conflittualità e denuncia sociale, il rifiuto della psicologia individuale a favore dell’amplificazione collettiva del dramma – tutti elementi propri all’orizzonte testuale di Intolleranza 1960 – divennero una costante nel corso degli anni Sessanta e Settanta, nel corso dei quali il concetto di impegno acquisì per Nono il valore di «imperativo morale» (J.P. Sartre) da affiancare a quello estetico. Il rapporto tra arte e attualità divenne sempre più intrecciato e profondo: ogni brano, realizzato o solo progettato, era concepito come un mezzo per partecipare attivamente, e con i propri strumenti specifici, a un più ampio processo di trasformazione della realtà sociale.

Spogliato il termine “ideologia” dell’accezione negativa propria della concettualizzazione marxiana, Nono ricondusse questa categoria di pensiero al significato gramsciano di “idea del mondo” esaltandone le caratteristiche di “strumento di verità” e di “funzione sociale” inscindibili dal messaggio artistico. A queste funzioni l’autore collegava senza mediazione lo sviluppo di un proprio peculiare linguaggio musicale, di una tecnica compositiva intesa anche come mezzo per arrivare a una testimonianza eticamente consapevole del proprio presente. Questo bisogno di attualità coinvolgeva il doppio piano del contenuto e della forma, binomio che caratterizzò, fin dagli esordi di Nono, il rapporto tra creazione e impegno. Ne La fabbrica illuminata (1964), per esempio, una voce dal vivo interagisce con se stessa preregistrata su nastro magnetico e vari materiali sonori (rumori, voci di operai ecc.), registrati dal vivo nella fabbrica dell’Italsider di Genova Cornigliano quindi elaborati elettronicamente in Studio. La denuncia, implicita nei testi documentari rielaborati da Giuliano Scabia sulla condizione operaia, è bilanciata in chiusura da una ferma fiducia nell’amore e nel futuro, in un chiaroscuro dramma-speranza caratteristico di molte opere vocali di Nono. Impegnate in una dimensione internazionale e terzomondista sono le successive A floresta é jovem e cheja de vida (1965-66), su testi documentari curati da Giovanni Pirelli, e Y entonces comprendió (1969-70), su testi di Carlos Franqui ed Ernesto “Che” Guevara. In queste opere l’alternanza tra voci dal vivo e preregistrate è ulteriormente sperimentata ed elaborata; si radicalizzano inoltre alcuni aspetti di prassi compositiva (intimamente legati a problematiche di prassi esecutiva) determinanti nella poetica musicale noniana degli anni Settanta-Ottanta. A partire da opere come La fabbrica illuminata o A floresta é jovem e cheja de vida il processo creativo di Nono venne infatti progressivamente a definirsi sempre a più stretto contatto con interpreti specifici, selezionati per le loro peculiarità timbriche ed espressive (in questa fase si ricordano, tra gli altri, i nomi di Carla Henius, Kadigia Bove, Elena Vicini). Grazie al lavoro condotto con gli interpreti in fase compositiva, Nono cominciò a non sentire più l’esigenza di fissare in un modo unico e definitivo la sua volontà autoriale in un’edizione a stampa (come nel caso di A floresta, ricostruita ed edita post mortem nel 1998). Soggetta a variabili ambientali, di proiezione spaziale del suono in sala, microfoniche ecc., l’opera cominciò ad essere intesa come prodotto di un processo in continuo divenire, spesso lasciata allo stadio di istruzione, appunto, progetto o schizzo e definita in forma conchiusa nelle sole direttive date all’interprete (la cui memoria spesso poteva coincidere in parte o in toto con il testo dell’opera). Anche laddove edite, come nel caso di Y entonces o della successiva Como una ola de fuerza y luz (1971-72, su testo di Julio Huasi), le creazioni venivano comunque definite gradualmente grazie a un lavoro condotto anche insieme agli interpreti (vocali, strumentali o addetti alla regia del suono) e orientato sempre più verso una pratica esecutiva del tutto atipica rispetto alle consuetudini tradizionali.

Ogni scelta testuale e/o performativa operata in questi anni testimonia della incessante volontà di Nono di intendere la musica come un mezzo di lotta, politica e sociale, per arrivare a denunciare ingiustizie e assurdità del presente. Nella biografia artistica e umana di Nono, la problematica relativa al concetto di impegno è stata (ed è ancora oggi) uno degli aspetti più complessi, dibattuti e, a seconda delle letture più o meno di parte, soggetto ad equivoci. Iscritto al Partito Comunista Italiano dal 1952 (quindi dal marzo del 1975 membro del Comitato Centrale), amico di esponenti e vertici del partito o critici marxisti militanti (Luigi Pestalozza tra questi), Nono non venne mai meno a un’ideale di artista d’avanguardia engagé, e questo anche quando – nelle ultime fasi della sua vita – l’impegno e la denuncia assunsero forme meno dirette. Il periodo più fecondo sul piano politico fu quello degli anni Sessanta-Settanta, in cui spesso i dati artistici giunsero a coincidere con quelli biografici (si pensi ai vari viaggi condotti nei paesi dell’Est a partire dal 1958, nell’URSS nel 1963 e negli anni Settanta, negli USA nel 1965 e nel 1979, nei paesi dell’America Latina – dal Cile al Perù a Cuba – a partire dal 1967; e ancora al confronto con la teoria e con la prassi del marxismo internazionale, la partecipazione alle lotte operaie degli anni Sessanta e ai movimenti studenteschi del 1968 ecc.). La militanza politica – esplicitamente dichiarata nelle scelte di carattere etico, sociale e artistico – divenne in questa fase inscindibile da quella di musicista alla continua ricerca di nuove soluzioni sonore. Proprio sul doppio terreno della politicizzazione delle opere e di un uso “rivoluzionario” dell’elettronica si consumò, nel 1964, la rottura con il suo primo editore storico, l’Ars Viva Verlag (Schott), e il conseguente passaggio alla Ricordi. Nel corso degli anni Sessanta-Settanta più volte Nono ebbe a parlare della propria musica come del prodotto di un’unione tra tecnica e ideologia, affermando a chiare lettere la propria “necessità” di declinare la musica al presente: «Sicuramente una partitura può causare una rivoluzione così poco come un quadro, una poesia o un libro; ma una musica può esattamente come un quadro, una poesia o un libro dare nota dello stato desolato della società, può contribuire, può fondare consapevolezza se le sue qualità tecniche si mantengono allo stesso livello di quelle ideologiche» (Nono 1969, p. 26). Questa e simili testimonianze contenute in altri testi redatti negli anni Sessanta (quali Il musicista nella fabbrica, 1966, o il testo di presentazione per Contrappunto dialettico alla mente, 1968) si sono rivelate spesso fuorvianti in sede critica, dove si è spesso dato più peso alle «qualità ideologiche» che alla portata innovativa e al valore artistico della sua musica.

Leggendo il dato in retrospettiva, la relazione tra arte e ideologia affonda le radici nello stesso apprendistato condotto con l’amico-mentore Maderna alla fine degli anni Quaranta e costituisce una tra le premesse creative della sua opera di esordio, di quelle Variazioni canoniche sulla serie dell’op. 41 di Arnold Schönberg intese come «conseguenza dei miei primi studi dei canoni enigmatici ma […] anche una scelta ideologica» (Nono 1979-80, p. 242). Nella fase più dichiaratamente engagée dell’evoluzione di Nono, il coinvolgimento politico procurò indiscutibili impulsi alla creazione: «sempre – scrisse presentando la Composizione per orchestra n. 2 – Diario polacco ’58, composta a seguito del suo primo viaggio in Polonia – la genesi di un mio lavoro si basa su una provocazione umana: un avvenimento, una esperienza, un testo della nostra vita provoca il mio istinto e la mia conoscenza a dare la testimonianza di me musicista-uomo» (Nono 1960, p. 433), parole applicabili anche alle ultime fasi del suo percorso artistico.

Queste «provocazioni», o stimoli, sono spesso palesi e verificabili nelle scelte dei materiali e dei collaboratori. Per le fonti testuali, gli anni Sessanta-Settanta sono segnati dalla ricerca di testi tratti da autori o soggetti storici che fossero al contempo simboli di lotta, di forza, di speranza e di sacrificio per la collettività (Karl Marx, “Che” Guevara, Rosa Luxemburg, Bertolt Brecht, Fidel Castro, Tania Bunke, ecc.). Per le fonti sonore, Nono fece spesso ricorso in questi anni a suoni concreti delle realtà operaie o di rivolta, fissati nelle sue opere con/per nastro magnetico, in cui violenza umana e sonora si intrecciano talora indelebilmente (come ne La fabbrica illuminata, in A floresta o in Non consumiamo Marx, seconda parte del dittico Musica-Manifesto n. 1, 1969). Sul fronte delle collaborazioni, si pensi a quella con Piscator nel 1965, con la realizzazione delle musiche elettroniche per lo spettacolo teatrale Die Ermittlung [L’istruttoria] di Peter Weiss; al sodalizio umano e artistico che lo legò a Claudio Abbado e Maurizio Pollini, conosciuti rispettivamente nel 1965 e nel 1966, interpreti legati indissolubilmente alla genesi di alcune tra le sue pagine più importanti (“su” e “per” Pollini fu ideata la parte pianistica di Como una ola de fuerza y luz, 1971-72 e …..sofferte onde serene…, 1976, laddove Abbado sarà sul podio delle prime assolute di Come una ola, Al gran sole carico d’amore, Prometeo); si pensi ancora al lavoro con il Living Theatre nel 1966 per il nastro di A floresta o, ancora, al lavoro collettivo con il regista e lo scenografo del teatro Taganka di Mosca (Jurij Ljubimov e David Borovskij) per Al gran sole carico d’amore. Questa seconda azione scenica, dedicata alle lotte di liberazione di tutto il mondo, segnò l’acme del periodo apertamente politico di Nono: dalla Comune di Parigi alla rivoluzione russa del 1905 a quella cubana ecc., varie sommosse per la libertà furono lette da Nono attraverso il ruolo assunto al loro interno dalle donne, simbolo di speranza forza e amore. Comporre, fare e diffondere musica, diventò dichiaratamente per il compositore un momento di sintesi dialettica tra arte e vita, vista come la sola maniera per «realizzarsi compiutamente» (Nono 1963, 144). Nei concetti di impegno e responsabilità sembra riaffiorare in Nono – apertamente negli anni centrali della sua “lotta” artistica, in modo più sotterraneo negli Ottanta – l’eco dell’insegnamento di Piscator, il cui volume Das politische Theater (del 1929, edito in Italia da Einaudi nel 1960) fu una lettura determinante per Nono nei primi anni Cinquanta, allorché il compositore prendeva gradualmente coscienza del fatto che «la sintesi di arte e politica significa suprema responsabilità, significa mettere al servizio delle supreme mete umane tutti i propri mezzi e dunque anche l’arte» (Piscator, Il teatro politico, Torino 1960, p. 48).

Ma, all’interno di una parabola compositiva quasi quarantennale che lo portò dalla serialità alla musica elettronica ai live electronics, per impegno bisogna intendere anche una visione “responsabile” della ricerca implicita in ogni nuova composizione e nella messa a punto di un linguaggio innovativo la cui rivoluzione è nel risultato sonoro. In questa prospettiva va ridimensionata, o addirittura respinta, l’idea di una presunta fase “a-politica” attribuita al Nono degli anni Ottanta: l’arte vissuta come responsabilità e impegno soggettivo (che lega l’autore, l’esecutore e l’ascoltatore) è una costante che accomuna l’intero itinerario artistico del compositore e che esula da messaggi politici, palesi o latenti.

Per la comunicazione dei propri messaggi sonori, Nono percepì come profondamente inadeguati tanto i tradizionali luoghi di produzione e diffusione musicale, quanto le procedure esecutive ad essi sottesi. Nel corso degli anni Sessanta-Settanta il lavoro si proiettò sempre più verso una dimensione collettiva; le fabbriche divennero sale da concerto; la ricerca di un nuovo teatro musicale fu equiparata tout court a una condanna delle tradizionali consuetudini di ascolto e fruizione degli spettacoli. Più che le istituzioni, furono piuttosto le forme e le consuetudini sedimentate in quelle istituzioni (culturali, concertistiche, sociali) ad essere considerate superate e discusse costantemente. Lo stesso concetto di “impegno” arrivò ad essere declinato anche nell’uso dello spazio. Nei progetti e nelle opere sceniche compiute – da Intolleranza 1960 alla «tragedia dell’ascolto» Prometeo (1984, rev. 1985) – divenne sempre più radicale la volontà di abbattere la tradizionale separazione tra scena e pubblico, vista da Nono come retaggio di una rappresentazione rituale e “antidemocratica” con «i fedeli che assistono e l’officiante che celebra» (Nono 1962, p. 122). Innovativa, in questo caso, non era l’idea in sé, già largamente dibattuta e sperimentata soprattutto in campo non musicale dalle avanguardie russe, nelle sperimentazioni di Gropius o nel teatro di Piscator (laddove in campo musicale vanno quantomeno ricordati alcuni esperimenti quali Passaggio di Luciano Berio, 1962).

Innovativa è, piuttosto, la dimensione sonora e visiva che Nono intendeva proiettare in uno spazio senza barriere, fisiche o ideali, di fruizione artistica. Lo “spazio” immaginato dal compositore – raggiunto negli anni Ottanta con la trasformazione, elaborazione e proiezione in tempo reale del suono consentita dai live electronics – era inteso come un ambiente in cui i rapporti spazio-temporali potessero infrangersi in una dimensione totale sia sul piano acustico (con la moltiplicazione e spazializzazione delle sorgenti sonore), sia su quello visivo (attraverso l’eliminazione della separazione tra scena e platea).

A metà degli anni Settanta, dopo la seconda importante tappa teatrale raggiunta con Al gran sole carico d’amore, intervenne nella vita di Nono una profonda crisi creativa amplificata dal doppio lutto che, a distanza di pochi mesi, lo colpì con la morte del padre (ottobre 1975) e della madre (gennaio 1976). Pochi anni dopo quegli eventi, così Nono rievocò quel particolare momento insieme ai cambiamenti umani e artistici che ne derivarono: «Subito dopo Al gran sole è venuto il silenzio, un silenzio inesprimibile: non avevo cioè i mezzi adatti ad esprimermi. Contemporaneamente è iniziato il mio rapporto di amicizia con Massimo Cacciari che pure conoscevo dal 1965. Ho sentito una necessità di studio non solo sul mio linguaggio musicale ma anche di analisi delle mie categorie mentali e ho ripreso a comporre con …..sofferte onde serene…, un lavoro che mi ha impegnato moltissimo» (Nono 1979-80, p. 245). Alle precedenti conquiste – quali la funzione prioritaria dell’interprete nel processo creativo, il ruolo delle tecnologie, ecc. – si affiancò una nuova tensione verso un’interiorizzazione del messaggio musicale e del concetto di impegno. Le principali caratteristiche dello stile che inaugura gli anni Ottanta – resosi manifesto a partire dal quartetto d’archi Fragmente-Stille, an Diotima, 1979-80 – sono il silenzio, la pausa, la giustapposizione di frammenti in cui pianissimi al limite dell’impercettibile si alternano a esplosioni sonore, il valore strutturale dello spazio.

Sebbene questi elementi abbiano portato alcuni critici ed esegeti a parlare di una “svolta” (memorabile l’articolo che nel 1988 Massimo Mila dedicò a quest’ultima fase noniana, intitolato appunto Nono, la svolta), o di repentine discontinuità nel percorso artistico noniano, è nondimeno da rilevare che tutti questi elementi erano già presenti in nuce (e altrimenti declinati) in diverse opere degli anni Cinquanta: silenzi e sonorità sulla soglia dell’inaudibile erano per esempio in Polifonica-Monodia-Ritmica, così come i chiaroscuri dinamici e le sonorità lacerate proprie di alcune opere degli anni Ottanta erano in Due espressioni per orchestra (1953), Il canto sospeso, La terra e la compagna (1957) o nei Cori di Didone. In aperta contraddizione con interpretazioni tecniche o stilistiche che procedono per decenni, nella totalità dell’arco creativo di Nono è possibile rintracciare il filo di uno sviluppo continuo, di un’incessante elaborazione di elementi messi al servizio di un’idea sonora immaginifica, confinante talvolta con l’utopia.

I mutamenti politici e sociali, la consapevolezza della progressiva perdita di un soggetto collettivo e dell’illusorietà di una rivoluzione sociale si palesano nelle scelte testuali delle opere degli anni Ottanta, in cui risulta evidente l’influsso dell’amico filosofo Massimo Cacciari. Friedrich Hölderlin, Rainer Maria Rilke, Robert Musil, la mistica ebraica, Walter Benjamin, Edmond Jabès, Giordano Bruno, Friedrich Nietzsche, il pensiero della tragedia e della mitologia greca: questi gli autori o testi rielaborati da Cacciari per Das atmende Klarsein (1981), Quando stanno morendo. Diario polacco n. 2 (1981), Guai ai gelidi mostri (1983) e per l’atipica opera Prometeo. A questi nuovi stimoli letterari si affiancarono le nuove risorse tecniche offerte dagli strumenti di trasformazione del suono in tempo reale (live electronics), sperimentate e approfondite in Germania presso l’Experimentalstudio der Heinrich-Strobel-Stiftung di Friburgo. Nono cominciò a frequentare questo nuovo laboratorio elettronico dal 1980 a seguito del suo addio allo Studio di Fonologia della RAI, ormai tecnologicamente vetusto, sancito dopo la messa a punto di Con Luigi Dallapiccola (1979), omaggio a una delle sue guide spirituali di gioventù e prima opera in cui i suoni – dismesso l’uso di nastri magnetici – sono trasformati in tempo reale.

Il pensiero sotteso alle creazioni dell’ultimo decennio – caratterizzate da un procedere per “tentativi” o “scelte”, e dalle costanti trasformazioni degli eventi sonori in sede di esecuzione – ricorda l’immagine dello scultore leonardiano, che «nel fare la sua opera fa per forza di braccia e di percussione a consumare il marmo, od altra pietra soverchia, ch’eccede la figura che dentro a quella si rinchiude» (Leonardo, Trattato della pittura, § «Differenza tra la pittura e la scoltura»). Questo procedere per sottrazione modellando il suono in tempo reale, spesso arricchito da nuove possibilità nate come reazioni ad errori tecnici, è evidente nel cammino che conduce al Prometeo, edificato sulle tappe preliminari di diversi brani intesi come “studi”: Das atmende Klarsein, Io, frammento dal Prometeo (1981), Quando stanno morendo. Diario polacco n. 2, Guai ai gelidi mostri. Sebbene Prometeo venga annoverato tra le opere teatrali di Nono, in esso si compie un’azione di totale scarnificazione dell’elemento scenico, del tutto dismesso, o narrativo: il “teatro”, l’azione, è nel suono, inteso come entità mobile, avulso da qualsivoglia apparato visivo e drammaturgicamente proiettato in uno spazio risonante edificato anche grazie alla struttura di legno in forma di “arca”, espressamente concepita da Renzo Piano per lo spazio della chiesa veneziana di S. Lorenzo (che ne ospitò la prima esecuzione assoluta nel 1984). Idealmente, il Prometeo può essere visto come l’approdo dei tentativi teatrali intrapresi all’indomani di Intolleranza 1960, proiettati verso un orizzonte sonoro in cui la vista lascia gradualmente il campo al puro ascolto. Sul piano dei contenuti testuali l’opera mirava invece non a una rilettura mitologica della figura di Prometeo quanto all’affermazione della sua portata “rivoluzionaria”, vista nella sua incessante ricerca di nuovi ordini che sovvertissero i precedenti: «in una parola, [del]la continuità prometeica senza fine» (Nono 1987, p. 559).

La ricerca di realtà sonore inaudite, tali da provocare non solo una differente maniera di “vivere” il suono (da parte di interpreti e fruitori) ma da richiedere anche diverse configurazioni degli spazi da concerto, è alla base dell’ultima produzione di Nono che, lontana dall’essere apolitica o disimpegnata, proietta l’ideale di un’arte tanto umana quanto impegnata nelle sfere interiori dell’“indicibile” (temi tra i più cari all’ultimo Nono, insieme a quello dell’“utopia”). All’indomani del Prometeo, conclusasi la collaborazione con Cacciari e tra soggiorni in Germania sempre più frequenti, Nono scrisse le sue opere orchestrali più visionarie: A Carlo Scarpa architetto, ai suoi infiniti possibili (1984), per orchestra a microintervalli; Caminantes… Ayacucho (1986-87) e «No hay caminos. Hay que caminar»… Andrei Tarkowskij (1987). Queste pagine per grande organico furono affiancate da brani con organico ridotto e trattamento del suono live electronics – tra queste i due omaggi a Boulez e Cacciari, A Pierre, dell’azzurro silenzio, inquietum (1985) e Risonanze erranti. Liederzyklus a Massimo Cacciari (1986) – e da brani solistici con o senza la trasformazione del suono in tempo reale: Post-Prae-Ludium per Donau (1988), La lontananza nostalgica utopica futura. Madrigale per più «caminantes» con Gidon Kremer (1988-89), scritta “sul” grande violinista citato nel titolo, e «Hay que caminar» sognando (1989), brano che chiude il catalogo noniano. Ciascuna di queste opere venne realizzata pressoché costantemente con singoli interpreti di fiducia – Roberto Fabbriciani, Ciro Scarponi, Giancarlo Schiaffini, Susanne Otto, Stefano Scodanibbio, Hans Peter Haller ecc. – spesso vicini al compositore anche nelle fasi preparatorie dell’opera in lunghe sedute di lavoro e studio presso il laboratorio elettronico di Friburgo, depositari di una volontà d’autore sempre più refrattaria ai limiti e margini di una scrittura musicale difficilmente codificabile in modo tradizionale.

Nei suoi ultimi anni di vita Nono intensificò i suoi rapporti con la Germania vivendone dall’interno le fasi che precedettero la caduta del Muro di Berlino. Altri viaggi decisivi per la genesi di alcune opere furono condotti in Spagna; fu proprio a Toledo, nel 1985, che lesse sul muro di un convento la parafrasi di un verso di Machado – «Caminantes: no hay caminos, hay que caminar» – fonte di ispirazione per il ciclo dei Caminantes che chiude il suo catalogo. Nel 1986-87 soggiornò per un lungo periodo a Berlino grazie a una borsa del DAAD (Deutscher Akademischer Austauschdienst, servizio tedesco per lo scambio accademico). Nel 1987-88 divenne membro del prestigioso istituto di ricerca Wissenschaftskolleg zu Berlin. Nel marzo 1990 vinse il Großer Kunstpreis Berlin, importante onorificenza annualmente conferita da una delle sei sezioni dell’Akademie der Künste a personalità di spicco in campo artistico. Ormai gravemente ammalato di una disfunzione epatica, Nono si spense a Venezia due mesi dopo, l’8 maggio 1990, nella sua dimora natale alle Zattere.

(Angela Ida De Benedictis, versione ampliata della voce Luigi Nono, Dizionario Biografico degli Italiani, Treccani.it L’enciclopedia italiana, Volume 78, 2013 / © Treccani. Per gentile concessione dell’editore e dell’autrice).
