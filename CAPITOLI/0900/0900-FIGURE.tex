%!TEX TS-program = xelatex
%!TEX encoding = UTF-8 Unicode
% !TEX root = ../metm.tex

\chapter{LE FIGURE DELLA MUSICA}

\begin{flushright}
		\textit{We think we're writing something to amuse, but \\
            we're actually saying something we desperately need to share. \\
            The real mystery is this strange need. \\
            Why can't we just hide it and shut up? \\
            Why do we have to blab? \\
            Why do human beings need to confess? \\
            Maybe if you don't have that secret confession, \\
            you don't have a poem - don't even have a story. \\
            Don't have a writer.} \\
            - Ted Hughes\footnote{Ted Hughes, \emph{The Art of Poetry No. 71}, \emph{The Paris Review}, n. 134 - 1995\\
						Pensiamo di scrivere qualcosa per intrattenere, ma | in realtà stiamo
						dicendo qualcosa che abbiamo disperatamente bisogno di condividere. |
						Il vero mistero è questo strano bisogno. | Perché non possiamo
						semplicemente nasconderlo e stare zitti? | Perché dobbiamo spifferare? |
						Perché gli esseri umani hanno bisogno di confessare? | Forse se non
						si ha quella confessione segreta, |	non si ha una poesia - non si
						ha nemmeno una storia. | Non si ha uno scrittore.}
\end{flushright}

\section{Compositore}

\emph{Chi è compositore oggi? Una persona che viene pagata per scrivere musica?
Una che produce dischi per i quali ha decine di sostenitori? Una persona che
insegna composizione? Una che è diplomata in composizione?}

La figura del compositore di musica è, oggi, un buon argomento di discussione e
\emph{deve} esserlo in ogni ambito di studi che possa presentare orizzonti compositivi.

Composizione è tecnica. Tecnica è regole. \emph{Composizione è regole?} Indispensabile
riflessione storica su questo punto è quella di Giorgio Nottoli\index{Nottoli, Giorgio} (1997) in
\emph{a proposito di musica contemporanea} pubblicato su \emph{Fisica nella Musica}
di Alberto Frova. In quel testo Nottoli\index{Nottoli, Giorgio} compone un ragionamento sulla composizione,
sull'essere compositore investendo inevitabilemente di responsabilità anche la
controparte, il fruitore, l'essere destinatario di quella che John Cage\index{Cage, John} definisce
\emph{lettera ad uno sconosciuto}: la composizone di un brano musicale.

Il testo si articola attorno ad un centro logico formato dal triangolo
\emph{regola-libertà-processo} che racchiude e protegge il cuore della composizione:
l'\emph{idea musicale}. Senza \emph{idea}, non c'è processo, né regole né
libertà d'azione. L'artista, il compositore, manipola l'\emph{idea} attraverso
questi tre canali d'intervento e ne ricava la sua \emph{composizione}.

La parabola narrativa usata da Nottoli\index{Nottoli, Giorgio} nell'articolo conduce al ruolo del fruitore,
ai sui diritti, ma anche ai suoi doveri (che dal 1997 sono purtroppo solo aumentati)
di studio, documentazione, discussione, per avvalersi di quegli strumenti culturali di analisi e
comprensione di un brano di musica d'oggi. Oggi. Perché il vero dramma
dell'essere compositori di musica oggi o, come scrive Nottoli\index{Nottoli, Giorgio}, compositori di
\emph{musica d'arte contemporanea}, è proprio quello della consapevolezza
dell'assenza di interlocutori o, più didascalicamente, di pubblico.

Se si vuole comprendere il \emph{messaggio codificato} insito in una musica e
si riconosce di non avere i requisiti minimi per la decodifica, ci si deve
impegnare per apprendere e conquistare tutte quelle le chiavi di accesso
attraverso le quali individuare i \emph{parametri} compositivi.
I \emph{come} ed i \emph{cosa} che collegano il fruitore all'ascolto, al
contesto storico del brano ed al senso che esso custidisce. Il
rapporto tra composizione e fruizione è quindi un rapporto alla pari, il
compositore e l'ascoltatore si servono di mezzi culturali condivisi per comprendersi.
Sempre che l'ascolto sia il fine ultimo, che si consideri nella musica la funzione
sociale di portare senso nell'ascolto stesso, dal quale si può anche dover
apprendere quindi, che ci stanno mancando chiavi di lettura, che bisogna studiare per
poi riascoltare.

I \emph{parametri} su cui opera il compositore d'oggi sono piuttosto sconosciuti
al fruitore d'oggi, ma come fa notare John Cage\index{Cage, John}, erano lontani anche dall'armonia
e dal carattere intervallare del contrappunto e della musica
dodecafonica\footnote{1948 - John Cage\index{Cage, John}, \emph{Confessioni di un compositore}}.
Entrambi i compositori fanno riferimento ai parametri fisici del suono, alle loro
relative sensazioni percettive, evidenziando quanto la musica sia inevitabilmente
condizionata dalla funzione che le releghiamo, una cornice in grado di oscurarne
il contenuto artistico ed ogni parametro codificato, in grado di portare in evidenza l'ormai sola ed unica
figura comprensibile e prodominante, nella cultura occidentale, della linea melodica.
Il problema culturale può essere affrontato partendo da infiniti spunti,
ma non è questo il luogo di una riflessione socio-antropologica. Potremmo
però soffermarci su alcune parole di John Cage\index{Cage, John} del 1948:

\begin{quote}
  La prima cosa che si nota a New York è la quantità incredibile di eventi. A
  Seattle, ricordo, c'era una mostra di pittura moderna che durava un mese ed
  era la sola; noi ci andavamo spesso, riflettevamo, ne parlavamo, la sentivamo
  sul serio. Suonavamo musica e ci rimaneva anche il tempo per qualche passatempo.
  Nulla di simile a New York. Ci sono talmente tanti concerti di musica, mostre
  di pittura, feste, eventi teatrali, chiamate telefoniche, una tale sequela di attività,
  che c'è da meravigliarsi come qualcuno riesca a mantenere la testa a posto\footnote{1948 -
	John Cage\index{Cage, John}, \emph{Confessioni di un compositore}}.
\end{quote}

% necessità e bisogno, fagioli.

Come qualcuno riesca ancora ad avere necessità di ascoltare. Di pensare. Vorrei
fare un ultimo ragionamento attorno alla composizione. Vorrei darle una
visione più complessa e tridimensionale.

Durante una conferenza di Michelangelo Lupone\index{Lupone, Michelangelo} a
Salerno nel 2016 lo sentii parlare per la prima volta del suo triangolo magico
di relazioni, quello composto da \emph{progetto-strumento-opera}. La visione
poetica di Lupone\index{Lupone, Michelangelo} rappresenta quanto di più monolitico prodotto negli ultimi
decenni: un legame tra tecnologia, suono, oggetto che lo evoca, spazio acustico, spazio
architettonico, percezione acustica, percezione visiva e percezione tattile.
La speculazione e la ricerca sullo strumento, sul mezzo, non è certo una novità
nel comporre musica. La sperimentazione e la ricerca sono però atteggiamenti che hanno
beneficiato enormemente degli impulsi nervosi e nevrotici che la musica elettronica
ha dato fin dai primi gemiti prodotti con le macchine. Il pensiero elettronico, o
elettroacustico, ha cambiato in maniera irrevocabile l'approccio alla composizone.
Si è insinuato nelle tecniche per cambiarle per sempre.

Dopo aver ascoltato il \emph{Prometeo} (1984) di Luigi Nono\index{Nono, Luigi} a Parma (2017), mi sono chiesto a lungo
dove fosse finita quella poetica,
quel modo di vedere e sentire la musica. La mia indagine si è conclusa circa un anno dopo, quando ho
analizzato \emph{Canto di madre} (1998) di Michelangelo Lupone\index{Lupone, Michelangelo}.
Quella poetica, con le giuste proporzioni socio-economiche,
non è mai sparita, ed è custodita proprio in questi triangoli di prassi compositiva: \emph{progetto-strumento-opera}
e \emph{regola-libertà-processo}. L'idea musicale e la poetica musicale contemporanea
sono vive, pulsano di ricerca, passione e desiderio di condivisione di un pensiero
racchiuso tra queste sei coordinate spaziali \emph{regola-libertà-processo-progetto-strumento-opera}.
Sei. In geometria tridimensionale con sei lati si descrive il primo solido regolare. Uno scrigno.
Una struttura ossea attorno ad un cuore pulsante.

\begin{quote}
  Dopo diciotto mesi di studio della filosofia e del misticismo orientali e del
  cristianesimo medievale, iniziai a leggere gli scritti di Jung sull'integrazione
  della personalità. Ci sono due componenti principali in ogni personalità:
  la mente cosciente e quella inconscia, e queste, nella maggior parte di noi,
  sono divise e disperse in infiniti modi e direzioni. La funzione della musica,
  come quella di ogni altra salutare attività, è quella di aiutare a riportare a
  una unità queste parti separate. La musica fa questo fornendo un momento in cui,
  essendo smarrita la consapevolezza dello spazio e del tempo, viene integrata la
  molteplicità degli elementi che costituisce un individuo ed egli è uno. Questo succede
  soltanto se, di fronte alla musica, non ci si lascia andare alla prigrizia e alla distrazione.
  Le occupazioni di molte persone oggi non solo non sono salutari, ma rendono malati coloro
  che le praticano, perché sviluppano una parte dell'individuo a detrimento dell'altra.
  Il malessere che ne risulta è in primo luogo psicologico, per questo si prendono periodi
  di vancanza dal lavoro per rimuoverlo. Ma poi la malattia attacca tutto l'organismo.
  [\ldots] Se uno fa musica, come direbbero in Oriente, \emph{disinteressatamente}, cioè
  senza alcun interesse per i soldi o la fama, ma soltanto per l'amore di farlo,
  compie un'attività di integrazione e troverà momenti nella vita che sono completi e
  appaganti. A volte è la composizione a farlo, a volte è suonare uno strumento, a
  volte è solo l'ascolto\footnote{1948 - John Cage\index{Cage, John}, \emph{Confessioni di un compositore}}.
\end{quote}

Il pensiero di Cage è lontano decenni da noi, dal nostro rapporto con l'arte e la
musica. Abbiamo una vaga idea di dove siamo ora con l'ascolto e con l'attenzione? La musica
elettronica ha lungamente trainato il pensiero musicale prima di essere messa a servizio,
funzionale alla macchina mercantile.

\begin{quote}
  La musica elettronica fu possibile solo quando la musica cessò di esistere come
  linguaggio costituito e come metafora linguistica, da quando il compositore
  cominciò a inventare e a elaborare "fonemi" (nel senso già indicato da Debussy, per esempio)
  e non a manipolare "parole" belle e fatte; da quando potè riprendere coscienza
  del fatto che le note non sono il materiale della musica ma solamente segni convenzionali
  dietro i quali si celano fenomeni concreti e che agire musicalmente significa organizzare la
  percezione e non le note\footnote{1961 - Luciano Berio\index{Berio, Luciano}, \emph{Musica sperimentale e musica radicale.}}.
\end{quote}

Non lo erano le note, il materiale della musica, non lo devono essere nemmeno i
nuovi mezzi di produzione sonora.

\begin{quote}
  Se questi ultimi hanno qualcosa da insegnare [al] di là della loro pratica utilizzazione,
  è proprio la incostituzionalità e la deficienza di un'assunzione "linguistica" della musica
  (senza per questo voler negare all'opera compiuta una sua struttra pensabile e
	riferibile).\footnote{1961 - Luciano Berio\index{Berio, Luciano}, \emph{Musica sperimentale e musica radicale.}}.
\end{quote}

Mi servo delle parole di Berio\index{Berio, Luciano} per cercare di capire in che momento questa coscienza
è involuta trascinandosi dietro la tecnologia, la scienza del suono. Qual è stato
il traino involutivo che ha trasformato un'opportunità di liberazione e consapevolezza
in un materiale linguistico.

\begin{quote}
  "Prima" dell'opera compiuta non esistono perciò materiali, ma situazioni di fatto
  naturali e culturali che noi assumiamo e trasformiamo di continuo, sulla base delle
  quali noi giungiamo a stabilire un certo campo d'azione possibile. Che io poi dia
  corpo alla mia opera valendomi di un'orchestra, di un pianoforte, di suoni
  prodotti elettricamente, registrati al microfono o tutte queste cose assieme,
  significherà semplicemente che l'idea di quell'opera implicava quei mezzi
  e non altri, e che mi verranno posti determianti problemi e determinate
  soluzioni piuttosto che altre: quello del compositore è pure sempre un
	mestiere!\footnote{1961 - Luciano Berio\index{Berio, Luciano}, \emph{Musica sperimentale e musica radicale.}}.
\end{quote}

Il pensiero di Berio\index{Berio, Luciano} completa il quadro poietico e poetico, fa da sfondo, è la storia
del pensiero emerso con Nottoli\index{Nottoli, Giorgio} e Lupone\index{Lupone, Michelangelo}, si concentra sull'idea musicale, sugli
inevitabili rapporti tra opera, struttura, progetto, mezzi, e quindi strumenti, un
campo d'azione con le sue regole e le sue libertà soggettive. È di nuovo un forte
legame con la storia alla base del quale si può ripartire con un'idea di \emph{scuola}
di composizione.

\begin{quote}
  Quello che più conta, infine, è di saper educare noi stessi e gli altri a
  considerare l'arte come una formazione, non come un
	funzionamento\footnote{1961 - Luciano Berio\index{Berio, Luciano}, \emph{Musica sperimentale e musica radicale.}}.
\end{quote}

%------------------------- APPROFONDIMENTO
		\begin{tabular}{L{.969\textwidth}}%
		\toprule
			\textbf{Composizione}\\
		\midrule
			dal lat. \emph{compositio -onis}, der. di \emph{componere} comporre.

			\begin{compactitem}
        \item L’atto, l’operazione, il lavoro del comporre, cioè del mettere
          ordinatamente e organicamente insieme; e anche il risultato di tale
          operazione: \emph{iniziare la c. di un mosaico; attendere alla c. di
          un poema, di una sinfonia; tinta ottenuta con una sapiente c. di colori;
          l’armonia risulta dalla c. di varî suoni}. Usato assol., nelle frasi
          studiare c., insegnare c. e sim., s’intende l’arte del comporre musica
          e la sua didattica. Nelle arti figurative e in fotografia, il modo in
          cui sono distribuiti, organizzati e messi in luce i vari elementi
          figurativi, con riguardo soprattutto all’unità stilistica; in
          architettura, il modo e il criterio con cui sono disposte e organizzate
          le diverse parti di un edificio o di più edifici tra loro
          (rispettivam., \emph{c. architettonica} e \emph{c. urbanistica}).
        \item L’insieme degli elementi di cui una cosa è composta, con riguardo
          alla natura o alla qualità di ciascuno o al loro reciproco rapporto:
          \emph{non si conosce ancora la c. della giuria; la c. dello sciroppo è
          descritta sull’etichetta}. Con sign. più specifico, \emph{c. chimica},
          il nome degli elementi e i rapporti in cui essi entrano a far parte
          della molecola di un composto: \emph{la c. dell’acido solforico}.
        \item La cosa stessa composta: \emph{questa bibita è una c. di diversi ingredienti;}
          spec. di opera d’arte: \emph{c. musicale, c. per canto e pianoforte;
          c. drammatica, poetica; mi ha letto una sua breve c. in versi;
          ha dipinto una c. allegorica;} meno com., lavoro scritto assegnato
          agli studenti, componimento.\footnote{\url{http://www.treccani.it/vocabolario/composizione/}}
      \end{compactitem} \\
    \bottomrule
		\end{tabular}
%------------------------- APPROFONDIMENTO

\section{Interprete}
\section{R.I.M.}
\section{Regista}
