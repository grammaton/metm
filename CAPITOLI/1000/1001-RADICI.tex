%%%%%%%%%%%%%%%%%%%%%%%%%%%%%%%%%%%%%%%%%%%%%%%%%%%%%%%%%%%%%%%%%%%%%% LE RADICI
%%%%%%%%%%%%%%%%%%%%%%%%%%%%%%%%%%%%%%%%%%%%%%%%%%%%%%%%%%%%%%%%%%%%%%%%%%%%%%%%
\section{Radici}

L'\emph{International Exposition of Electricity}, tenutasi nel 1881, ha mostrato
cose come l'illuminazione elettrica con Edison e altri che presentano le loro
lampadine a incandescenza.

Uno dei reperti era un "telefono stereo". Nel 1881 Alexander Graham Bell aveva un modello funzionante del suo telefono, ma era il francese Clément Ader la cui invenzione fu esposta [2]. Forse la ragione è stata la decisione del governo britannico di non sprecare denaro pubblico sponsorizzando l'evento. Ogni sera veniva suonata musica alla Grande Opera e la gente poteva ascoltare per alcuni minuti al telefono a circa due chilometri di distanza al Palais de Industrie, dove si teneva l'esposizione. Una serie di dieci microfoni al carbonio è stata posizionata lungo la larghezza del palco. Ciascuno di questi era collegato a otto stazioni di ascolto all'estremità opposta. Si potrebbe pensare che, dato che il telefono era ancora agli inizi, per i membri del pubblico semplicemente ascoltare la performance sarebbe stato sufficiente, ma Ader ha aggiunto l'esperienza di "stereo". Ogni persona che ascoltava aveva due auricolari, uno per ciascun orecchio. Ogni auricolare ha presentato la performance da una posizione separata sul palco (ad es. Microfoni 1 e 6, 2 e 7, e così via) consentendo così all'ascoltatore di valutare la posizione laterale dell'esecutore. Fortunatamente, l'esperimento ebbe successo e nacque "stereo". Diversi anni dopo l'Expo di Parigi del 1881, l'invenzione fu usata commercialmente in Francia e in Inghilterra. Il telefono e la rete richiesta furono sviluppati, così come la distribuzione di energia elettrica, ma il prossimo traguardo per lo stereo non sarebbe arrivato fino a 50 anni dopo.

La percezione dei caratteri spazio-temporali dei suoni, in particolare della
loro direzione di provenienza e della loro relazione con lo spazio che
attraversano, definiscono i tratti essenziali della stereofonia, in relazione
all'udito, in virtù dell’audizione biauricolare (o binaurale). Come sottolineato
nell'enciclopedia Treccani, tra le varie definizioni di stereofonia \emph{il
termine è inoltre usato per indicare la parte dell’acustica fisiologica
che si occupa di tale fenomeno}. L'ascolto biauricolare del sistema uditivo
conferisce alla percezione umana il potere localizzatore, cioè la capacità,
dovuta al lavoro congiunto dei due sistemi auricolari separati ed al sistema
nervoso centrale, di determinare la direzione di provenienza di un suono. In tal
senso, esiste in acustica fisiologica la definizione di monofonia in qualità di
condizione anomala del sistema percettivo, caratterizzata dalla mancanza degli
elementi necessari a individuare i caratteri spaziali dei suoni stessi, come per
esempio quella ottenuta con un solo orecchio.

L'ultimo esercizio linguistico è legato alla definizione dell'aggettivo
\emph{stereofònico} e ci permette di aprire il termine alle tecniche ed alle
tecnologie che hanno reso possibile la trasmissione, la riproduzione,
e la diffusione della stereofonia. \emph{Stereofònico}, quindi, che si riferisce
alla stereofonia, alla percezione della stereofonia, alla descrizione delle
tecniche di registrazione e riproduzione in grado di operare stereofonia. Un
aggettivo che dovrebbe essere usato solo nella
descrizione di tecniche di registrazione e di diffusione sonora atte alla
riproduzione dei suoni in modo che l’ascoltatore abbia l’impressione di trovarsi
nello spazio sonoro originale, dai quali ne deriva una musica stereofonica ed
una discografia stereofonica in grado di grarantire l’ambiente sonoro originale,
solido e tridimensionale attraverso l’effetto stereofonico.

\begin{quotation}
An observer in the room is listening with two ears, so that echoes reach him
with the directional significance which he associates with the music performed
in such room. He therefore discount these echoes and psychologically focuses
his attention on the source of the sound. When the music is reproduced through
a single channel the echoes arrive from the same direction as the direct sound
so that confusion results. [\ldots] Human ability to determine the direction
from which sound arrives is due to binaural hearing, the brain being able to
detect differences between sound received by the two ears from the same source
and thus to determine angular directions from which various sounds
arrive\footnote{Un osservatore nella stanza sta ascoltando con due orecchie, in
modo che gli echi lo raggiungano con il significato direzionale che associa alla
musica eseguita in quella stanza. Pertanto, non tiene conto di questi echi e
focalizza psicologicamente la sua attenzione sulla fonte del suono. Quando la
musica viene riprodotta attraverso un singolo canale, gli echi arrivano dalla
stessa direzione del suono diretto, in modo da creare confusione. [...] La
capacità umana di determinare la direzione da cui proviene il suono è dovuta
all'udito binaurale, il cervello è in grado di rilevare le differenze tra il
suono ricevuto dalle due orecchie dalla stessa fonte e quindi di determinare le
direzioni angolari da cui provengono i vari suoni.}. [\cite{ab58}]
\end{quotation}

Con queste parole Blumlein nel 1931 descrive i fondamenti di almeno due grandi
argomenti: quali erano le conoscenze  dell'epoca su come percepiamo i suoni
acustici e come abbiamo riprodotto i suoni fino a quel momento.

La binauralità dell'ascolto umano è la prima affermazione di Blumlein:
“\emph{un osservatore nella stanza sta ascoltando con due orecchie}”. Come
questa condizione di ascolto si evolva nel tempo è la peculiarità della
stereofonia.

% Non è correlato al numero di fonti, nemmeno al numero di microfoni
% e altoparlanti necessari per riprodurre tale condizione. La tecnica e lo scopo
% della tecnica prescelta si concentreranno per risolvere più argomenti possibili
% per soddisfare tale condizione.

%%%%%%%%%%%%%%%%%%%%%%%%%%%%%%%%%%%%%%%%%%%%%%%%%%%%%%%%%%%%%%%%%%%%%%%%%%%%%%%%%%%%%%%%%%%%%%%%%% dalle esposizioni
Dovremmo indicare quindi qualitativamente una condizione di ascolto, dal latino
\emph{auscultare}, prestare attenzione a qualcosa in quanto oggetto o motivo di informazione, con specifiche caratteristiche di solidità spaziale, dimensionalità osservabili nella forma sonora riconoscibile, informativa.

\emph{Una voce in una piccola stanza riverberante è una condizione d'ascolto che
rispetti queste qualità?}

Prima di approfondire questioni di propagazione e percezione del suono, val la pena
dedicare un tempo alla letteratura specializzata.

\begin{quote}
When recording music considerable trouble is experienced with the unpleasant
effects produced by echoes wich in the normal way would not be noticed by anyone
listening in the room in which the performance is taking place. An observer in
the room is listening with two ears, so that echoes reach him with the directional
significance which he associates with the music performed in such room. He,
therefore, discounts these echoes and psychologically focuses his attention on
the source of the sound\footnote{Quando si registra musica acustica, si riscontrano
notevoli problemi a causa degli effetti indesiderati prodotti dalle riflessioni
acustiche dell'ambiente, che nell'ascolto normale non vengono notati dagli
ascoltatori nella stanza in cui si svolge l'esibizione. L'ascoltatore, nella stanza,
ascolta attraverso le due orecchie, le riflessioni lo raggiungano con il significato direzionale che associa alla musica eseguita in quella stanza.
Pertanto, elimina dal messaggio le informazioni delle riflessioni e focalizza
psicologicamente la sua attenzione sulla sorgente sonora.}.
\end{quote}

Richiesta di brevetto numero 394325 del 14 dicembre 1931, accettazione del 14
giugno 1933. Alan Dower Blumlein.

La risposta alla domanda \emph{Una voce in una piccola stanza riverberante è una
condizione d'ascolto che rispetti queste qualità?} è, in funzione di quanto
appena letto, chiaramente affermativa.

Anche con un solo oggetto sonoro, una sola voce, in una piccola stanza, siamo in
presenza di un fenomeno acustico stereofonico. Almeno così dice Blumlein, papà
della stereofonia, nel brevetto in cui ne rende i concetti fondamentali, solidi,
stabili nel tempo e nello spazio delle parole, nel brevetto tecnologico che stabilisce
il \emph{point break} dell'elettroacustica, per il resto dell'umanità.
