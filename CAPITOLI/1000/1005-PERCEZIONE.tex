% %%%%%%%%%%%%%%%%%%%%%%%%%%%%%%%%%%%%%%%%%%%%%%%%%%%%%%%%%%%%%%%%% SECTION FIVE
% %%%%%%%%%%%%%%%%%%%%%%%%%%%%%%%%%%%%%%%%%%%%%%%%%%%%%%%%%%%%%%%%%%%%%%%%%%%%%%
\section{Ripresa stereofonica e percezione psicoacustica}

La localizzazione del suono è la capacità di un ascoltatore di identificare la
posizione o l'origine di un suono rilevato in direzione e distanza.

Il sistema uditivo utilizza diversi meccanismi per la localizzazione della
sorgente sonora, motivo per cui la percezione della direzione di provenienza del
suono è data dall’elaborazione, nel nostro cervello, dei differenti segnali
nervosi inviati dalle due orecchie che compongono il nostro apparato uditivo.
Tale elaborazione ci porta a definire la localizzazione del suono.

Un suono che arrivi non frontalmente, quindi con un angolo di incidenza
diverso da zero, giunge alle orecchie in modo differente, e precisamente:

\begin{compactitem}
\item con tempi diversi
\item con intensità diversa
\end{compactitem}

la correlazione della differenza di tempo e di intensità fa sì che la
direzione di provenienza del suono possa venire localizzata dal cervello ed in
funzione di queste differenze si innescano meccanismi subordinati che legati a
informazioni spettrali e di correlazione tra i due segnali.

Storia [modifica]
Il termine "binaurale" significa letteralmente "ascoltare con due orecchie", e fu introdotto nel 1859 per indicare la pratica di ascoltare lo stesso suono attraverso entrambe le orecchie o due suoni discreti, uno attraverso ciascun orecchio. Fu solo nel 1916 che Carl Stumpf (1848-1936), filosofo e psicologo tedesco, si distinse tra ascolto dicotico, che si riferisce alla stimolazione di ciascun orecchio con uno stimolo diverso e all'ascolto diotico, la stimolazione simultanea di entrambe le orecchie con stesso stimolo. [27]

Più tardi, risulterebbe evidente che l'udito binaurale, sia dicotico che diotico, è il mezzo con cui si verifica la localizzazione del suono. [27] [28] [pagina necessaria]

La considerazione scientifica dell'udito binaurale iniziò prima che il fenomeno fosse così chiamato, con le speculazioni pubblicate nel 1792 da William Charles Wells (1757–1817) basate sulla sua ricerca sulla visione binoculare. [29] Giovanni Battista Venturi (1746–1822) condusse e descrisse esperimenti in cui le persone cercavano di localizzare un suono usando entrambe le orecchie o un orecchio bloccato con un dito. Questo lavoro non è stato seguito, ed è stato recuperato solo dopo che altri avevano capito come funziona la localizzazione del suono umano. [27] [29] Lord Rayleigh (1842-1919) avrebbe fatto questi stessi esperimenti e sarebbe arrivato ai risultati, senza sapere che Venturi li aveva fatti per la prima volta, quasi settantacinque anni dopo [29].

Charles Wheatstone (1802–1875) lavorò sull'ottica e sulla miscelazione dei colori e esplorò anche l'udito. Ha inventato un dispositivo che ha chiamato un "microfono" che comportava una piastra di metallo su ciascun orecchio, ciascuno collegato a aste di metallo; ha usato questo dispositivo per amplificare il suono. Fece anche esperimenti tenendo le forcelle accordanti su entrambe le orecchie contemporaneamente o separatamente, cercando di capire come funziona il senso dell'udito, che pubblicò nel 1827. [29] Anche Ernst Heinrich Weber (1795–1878) e August Seebeck (1805–1849) e William Charles Wells tentarono di confrontare e confrontare ciò che sarebbe diventato noto come udito binaurale con i principi dell'integrazione binoculare in generale. [29]

Comprendere come le differenze nei segnali sonori tra due orecchie contribuiscano all'elaborazione uditiva in modo tale da consentire la localizzazione e la direzione del suono sono state considerevolmente avanzate dopo l'invenzione dello stetofono di Somerville Scott Alison nel 1859, che ha coniato il termine "binaurale". Alsion basò lo stetofono sullo stetoscopio, che era stato inventato da René Théophile Hyacinthe Laennec (1781–1826); lo stetofono aveva due "pickup" separati, che permettevano all'utente di ascoltare e confrontare i suoni derivati ​​da due posizioni discrete. [29] [29]

Il suono è il risultato percettivo delle vibrazioni meccaniche che viaggiano attraverso un mezzo come l'aria o l'acqua. Attraverso i meccanismi di compressione e rarefazione, le onde sonore viaggiano attraverso l'aria, rimbalzano dalla pinna e dal concha dell'orecchio esterno ed entrano nel canale uditivo. Le onde sonore vibrano la membrana timpanica (timpano), facendo vibrare le tre ossa dell'orecchio medio, che quindi invia l'energia attraverso la finestra ovale e nella coclea dove viene trasformata in un segnale chimico dalle cellule ciliate nell'organo di Corti, che sinapsi su fibre gangliari a spirale che viaggiano attraverso il nervo cocleare nel cervello.

Nei vertebrati, è noto che le differenze di tempo inter-auricolare sono calcolate nel nucleo olivario superiore del tronco encefalico. Secondo Jeffress, [1] questo calcolo si basa su linee di ritardo: neuroni nell'oliva superiore che accettano innervazione da ciascun orecchio con diverse lunghezze assone di collegamento. Alcune cellule sono più direttamente collegate a un orecchio rispetto all'altro, quindi sono specifiche per una particolare differenza di tempo inter-auricolare. Questa teoria è equivalente alla procedura matematica di correlazione incrociata. Tuttavia, poiché la teoria di Jeffress non è in grado di spiegare l'effetto di precedenza, in cui solo il primo di più suoni identici viene utilizzato per determinare la posizione dei suoni (evitando così la confusione causata da echi), non può essere utilizzato interamente per spiegare la risposta . Inoltre, una serie di recenti osservazioni fisiologiche fatte nel mesencefalo e nel tronco encefalico di piccoli mammiferi hanno sollevato notevoli dubbi sulla validità delle idee originali di Jeffress. [2]

I neuroni sensibili alle differenze di livello inter-auricolare (ILD) sono eccitati dalla stimolazione di un orecchio e inibiti dalla stimolazione dell'altro orecchio, in modo tale che l'entità della risposta della cellula dipenda dai punti di forza relativi dei due input, che a loro volta dipendono sulle intensità del suono alle orecchie.

Nel nucleo del mesencefalo uditivo, il collicolo inferiore (IC), molti neuroni sensibili all'ILD hanno funzioni di risposta che diminuiscono drasticamente dal picco massimo a zero in funzione dell'ILD. Tuttavia, ci sono anche molti neuroni con funzioni di risposta molto più superficiali che non diminuiscono a zero.

Il cono di confusione

La maggior parte dei mammiferi è abile nel risolvere la posizione di una sorgente sonora usando differenze di tempo interaurali e differenze di livello interaurale. Tuttavia, tali differenze temporali o di livello non esistono per i suoni originati lungo la circonferenza delle sezioni circolari coniche, in cui l'asse del cono si trova lungo la linea tra le due orecchie.

Di conseguenza, le onde sonore che originano in qualsiasi punto lungo una data altezza inclinata della circonferenza avranno coordinate percettive ambigue. Vale a dire, l'ascoltatore non sarà in grado di determinare se il suono proviene dalla parte posteriore, anteriore, superiore, inferiore o da qualsiasi altra parte lungo la circonferenza alla base di un cono a una data distanza dall'orecchio. Naturalmente, l'importanza di queste ambiguità è minuscola per le fonti sonore molto vicine o molto lontane dal soggetto, ma sono queste distanze intermedie che sono più importanti in termini di idoneità.

Queste ambiguità possono essere rimosse inclinando la testa, il che può introdurre uno spostamento sia nell'ampiezza che nella fase delle onde sonore che arrivano ad ogni orecchio. Ciò traduce l'orientamento verticale dell'asse interaurale in senso orizzontale, sfruttando così il meccanismo di localizzazione sul piano orizzontale. Inoltre, anche senza alcuna alternanza nell'angolo dell'asse interaurale (cioè senza inclinare la testa) il sistema uditivo può capitalizzare su schemi di interferenza generati da pinne, busto e persino il riposizionamento temporaneo di una mano come estensione della pinna (ad es. stringendo una mano attorno all'orecchio).

Come con altri stimoli sensoriali, anche la disambiguazione percettiva si realizza attraverso l'integrazione di più input sensoriali, in particolare segnali visivi. Avendo localizzato un suono all'interno della circonferenza di un cerchio ad una certa distanza percepita, gli indizi visivi servono a fissare la posizione del suono. Inoltre, una conoscenza preliminare della posizione dell'agente che genera il suono aiuterà a risolvere la posizione corrente.

Buona localizzazione da parte del sistema uditivo umano [modifica]
La localizzazione del suono è il processo per determinare la posizione di una sorgente sonora. Obiettivamente, l'obiettivo principale della localizzazione del suono è simulare un campo sonoro specifico, comprese le fonti acustiche, l'ascoltatore, i media e gli ambienti di propagazione del suono. Il cervello utilizza sottili differenze di intensità, spettrali e segnali di temporizzazione per permetterci di localizzare le fonti sonore. [3] [4] In questa sezione, per comprendere più a fondo il meccanismo uditivo umano, discuteremo brevemente della teoria della localizzazione dell'orecchio umano.

Introduzione generale [modifica]
La localizzazione può essere descritta in termini di posizione tridimensionale: l'azimut o l'angolo orizzontale, l'elevazione o l'angolo verticale e la distanza (per i suoni statici) o la velocità (per i suoni in movimento). [5]

L'azimut di un suono è segnalato dalla differenza nei tempi di arrivo tra le orecchie, dall'ampiezza relativa dei suoni ad alta frequenza (l'effetto ombra) e dalle riflessioni spettrali asimmetriche da varie parti del nostro corpo, inclusi busto, spalle, e pinnae. [5]

I segnali di distanza sono la perdita di ampiezza, la perdita di alte frequenze e il rapporto tra il segnale diretto e il segnale riverberato. [5]

A seconda di dove si trova la sorgente, la nostra testa funge da barriera per cambiare il timbro, l'intensità e le qualità spettrali del suono, aiutando il cervello ad orientarsi da dove il suono emanava. [4] Queste minime differenze tra le due orecchie sono note come segnali interaurali. [4]

Le frequenze più basse, con lunghezze d'onda più lunghe, diffondono il suono intorno alla testa costringendo il cervello a focalizzarsi solo sui segnali di fasatura dalla sorgente. [4]

Helmut Haas ha scoperto che possiamo discernere la fonte sonora nonostante ulteriori riflessi a 10 decibel più forti del fronte d'onda originale, usando il primo fronte d'onda in arrivo. [4] Questo principio è noto come effetto Haas, una versione specifica dell'effetto precedenza. [4] Haas misurato fino a una differenza di 1 millisecondo nei tempi tra il suono originale e il suono riflesso aumentava la spaziosità, consentendo al cervello di discernere la vera posizione del suono originale. Il sistema nervoso combina tutte le prime riflessioni in un singolo insieme percettivo permettendo al cervello di elaborare più suoni diversi contemporaneamente. [6] Il sistema nervoso combinerà riflessi che si trovano a circa 35 millisecondi l'uno dall'altro e che hanno un'intensità simile. [6]

Teoria duplex [modifica]
Per determinare la direzione di ingresso laterale (sinistra, anteriore, destra), il sistema uditivo analizza le seguenti informazioni sul segnale acustico:

Teoria duplex [modifica]
Nel 1907, Lord Rayleigh utilizzava diapason per generare l'eccitazione monofonica e studiava la teoria della localizzazione del suono laterale su un modello di testa umana senza padiglione auricolare. Ha presentato per la prima volta la teoria della localizzazione del suono basata sulla differenza interaurale dell'indizio, nota come teoria duplex. [7] Le orecchie umane sono su diversi lati della testa, quindi hanno coordinate diverse nello spazio. Come mostrato in fig. 2, poiché le distanze tra la sorgente acustica e le orecchie sono diverse, vi sono differenze di tempo e di intensità tra i segnali sonori di due orecchie. Chiamiamo questo tipo di differenze rispettivamente come Differenza di tempo interaurale (ITD) e Differenza di intensità interaurale (IID).

TD e IID [modifica]

Differenza di tempo interaurale (ITD) tra orecchio sinistro (in alto) e orecchio destro (in basso).
[sorgente sonora: 100 ms rumore bianco da destra]

Differenza di livello interaurale (ILD) tra orecchio sinistro (sinistro) e orecchio destro (destro).
[fonte sonora: una spazzata da destra]
Dalla fig.2 possiamo vedere che, indipendentemente dalla fonte B1 o dalla fonte B2, ci sarà un ritardo di propagazione tra due orecchie, che genererà il DTI. Allo stesso tempo, la testa e le orecchie umane possono avere un effetto ombra sui segnali ad alta frequenza, che genereranno IID.

Differenza di tempo interaurale (ITD) Il suono proveniente dal lato destro raggiunge l'orecchio destro prima dell'orecchio sinistro. Il sistema uditivo valuta le differenze di tempo interaurali da: (a) Ritardi di fase alle basse frequenze e (b) ritardi di gruppo alle alte frequenze.
Esperimenti di massa dimostrano che il DTI si riferisce alla frequenza del segnale f. Supponiamo che la posizione angolare della sorgente acustica sia θ, il raggio della testa sia r e la velocità acustica sia c, la funzione di ITD sia data da: [8]
io
T
D
=
{
300 ×
r
× sin⁡θ
/
c
,
Se
f≤
4000Hz
200 ×
r
× sin⁡θ
/
c
,
Se
f>
 4000Hz
 
{\ displaystyle ITD = {\ begin {case} 300 \ times {\ text {r}} \ times \ sin \ theta / {\ text {c}}, & {\ text {if}} f \ leq {\ text {4000Hz}} \\ 200 \ times {\ text {r}} \ times \ sin \ theta / {\ text {c}} e {\ text {if}} f> {\ text {4000Hz}} \ end {casi}}}. In forma chiusa sopra, abbiamo assunto che lo 0 gradi sia in avanti davanti alla testa e in senso antiorario sia positivo.
Differenza di intensità interaurale (IID) o Differenza di livello interaurale (ILD) Il suono proveniente dal lato destro ha un livello più elevato nell'orecchio destro rispetto all'orecchio sinistro, poiché la testa ombreggia l'orecchio sinistro. Queste differenze di livello dipendono fortemente dalla frequenza e aumentano con l'aumentare della frequenza. Ricerche teoriche di massa dimostrano che l'IID si riferisce alla frequenza del segnale f e alla posizione angolare della sorgente acustica θ. La funzione di IID è data da: [8]
io
io
D
=
1.0
+
(
f
/
1000
)
0.8
×
peccato
⁡
θ
{\ displaystyle IID = 1.0 + (f / 1000) ^ {0.8} \ times \ sin \ theta}
Per frequenze inferiori a 1000 Hz, vengono valutati principalmente i DTI (ritardi di fase), per frequenze superiori a 1500 Hz principalmente vengono valutati IID. Tra 1000 Hz e 1500 Hz c'è una zona di transizione, in cui entrambi i meccanismi svolgono un ruolo.
La precisione della localizzazione è di 1 grado per le fonti di fronte all'ascoltatore e di 15 gradi per le fonti ai lati. Gli umani possono discernere differenze di tempo interaurali di 10 microsecondi o meno. [9] [10]

Valutazione per basse frequenze [modifica]
Per frequenze inferiori a 800 Hz, le dimensioni della testa (distanza dell'orecchio 21,5 cm, corrispondente a un ritardo temporale interaurale di 625 µs) sono inferiori alla metà della lunghezza d'onda delle onde sonore. Quindi il sistema uditivo può determinare i ritardi di fase tra le due orecchie senza confusione. Le differenze di livello interaurali sono molto basse in questa gamma di frequenza, specialmente al di sotto di circa 200 Hz, quindi una valutazione precisa della direzione di ingresso è quasi impossibile sulla base delle sole differenze di livello. Poiché la frequenza scende al di sotto di 80 Hz diventa difficile o impossibile utilizzare la differenza di tempo o la differenza di livello per determinare la sorgente laterale di un suono, perché la differenza di fase tra le orecchie diventa troppo piccola per una valutazione direzionale. [11]

Valutazione per le alte frequenze [modifica]
Per frequenze superiori a 1600 Hz le dimensioni della testa sono maggiori della lunghezza delle onde sonore. Una determinazione inequivocabile della direzione di ingresso basata sulla sola fase interaurale non è possibile a queste frequenze. Tuttavia, le differenze di livello interaurale diventano maggiori e queste differenze di livello vengono valutate dal sistema uditivo. Inoltre, i ritardi di gruppo tra le orecchie possono essere valutati ed è più pronunciato a frequenze più alte; vale a dire, se c'è un insorgenza del suono, il ritardo di questo esordio tra le orecchie può essere usato per determinare la direzione di ingresso della sorgente sonora corrispondente. Questo meccanismo diventa particolarmente importante negli ambienti riverberanti. Dopo un inizio del suono c'è un breve lasso di tempo in cui il suono diretto raggiunge le orecchie, ma non ancora il suono riflesso. Il sistema uditivo utilizza questo breve lasso di tempo per valutare la direzione della sorgente sonora e mantiene tale direzione rilevata fintanto che riflessioni e riverbero impediscono una stima della direzione non ambigua. [12] I meccanismi sopra descritti non possono essere utilizzati per distinguere tra una sorgente sonora davanti all'ascoltatore o dietro l'ascoltatore; pertanto è necessario valutare ulteriori segnali. [13]

Teoria dell'effetto di filtro di Pinna [modifica]

fig.4 HRTF
Motivazioni [modifica]
La teoria duplex sottolinea chiaramente che ITD e IID svolgono ruoli significativi nella localizzazione del suono, ma possono affrontare solo problemi di localizzazione laterale. Ad esempio, in base alla teoria duplex, se due sorgenti acustiche sono posizionate simmetricamente sulla parte anteriore destra e posteriore destra della testa umana, genereranno ITD e IID uguali, che viene chiamato effetto modello del cono. Tuttavia, le orecchie umane possono effettivamente distinguere questo insieme di fonti. Oltre a ciò, nel senso dell'udito naturale, solo un orecchio, che non significa DTI o IID, può distinguere le fonti con un'alta precisione. A causa degli svantaggi della teoria duplex, i ricercatori hanno proposto la teoria dell'effetto del filtro pinna. [14] La forma del pinna umano è molto speciale. È concavo con pieghe complesse e asimmetrico, non importa in orizzontale o in verticale. Le onde riflesse e le onde dirette genereranno uno spettro di frequenza sul timpano, che è correlato alle fonti acustiche. Quindi i nervi uditivi localizzano le fonti secondo questo spettro di frequenza. Pertanto, una teoria corrispondente è stata proposta e chiamata come teoria dell'effetto filtro pinna. [15]

Modello matematico [modifica]
Questi indizi dello spettro generati dall'effetto filtro pinna possono essere presentati come una funzione di trasferimento correlata alla testa (HRTF). Le espressioni del dominio del tempo corrispondenti sono chiamate come risposta all'impulso (HRIR). La terapia ormonale sostitutiva è anche chiamata come funzione di trasferimento dal campo libero a un punto specifico nel condotto uditivo. Di solito riconosciamo gli HRTF come sistemi LTI: [8]

H
L
=
H
L
(
r
,
θ
,
φ
,
ω
,
α
)
=
P
L
(
r
,
θ
,
φ
,
ω
,
α
)
/
P
0
(
r
,
ω
)
{\ displaystyle H_ {L} = H_ {L} (r, \ theta, \ varphi, \ omega, \ alpha) = P_ {L} (r, \ theta, \ varphi, \ omega, \ alpha) / P_ { 0} (r, \ omega)}

H
R
=
H
R
(
r
,
θ
,
φ
,
ω
,
α
)
=
P
R
(
r
,
θ
,
φ
,
ω
,
α
)
/
P
0
(
r
,
ω
)
{\ displaystyle H_ {R} = H_ {R} (r, \ theta, \ varphi, \ omega, \ alpha) = P_ {R} (r, \ theta, \ varphi, \ omega, \ alpha) / P_ { 0} (r, \ omega)},

dove L e R rappresentano rispettivamente l'orecchio sinistro e l'orecchio destro.
P
L
P_ {L} e
P
R
P_R rappresenta l'ampiezza della pressione sonora alle entrate del canale uditivo sinistro e destro.
P
0
P_ {0} è l'ampiezza della pressione sonora al centro della coordinata principale quando l'ascoltatore non esiste. In generale, HRTF
H
L
H_ {L} e
H
R
H_ {R} sono funzioni della posizione angolare della sorgente
θ
\ theta, angolo di elevazione
φ
\ varphi, distanza tra sorgente e centro della testa
r
r, la velocità angolare
ω
\ omega e la dimensione equivalente della testa
α
\ alpha.

Altri spunti per la localizzazione dello spazio 3D [modifica]
Indizi monoaurali [modifica]
L'orecchio esterno umano, cioè le strutture della pinna e del condotto uditivo esterno, formano filtri selettivi in ​​direzione. A seconda della direzione di ingresso del suono nel piano mediano, diventano attive diverse risonanze del filtro. Queste risonanze tracciano schemi specifici della direzione nelle risposte in frequenza delle orecchie, che possono essere valutate dal sistema uditivo per la localizzazione del suono verticale. Insieme ad altri riflessi selettivi in ​​direzione della testa, delle spalle e del busto, formano le funzioni di trasferimento dell'orecchio esterno. Questi schemi nelle risposte in frequenza dell'orecchio sono altamente individuali, a seconda della forma e delle dimensioni dell'orecchio esterno. Se il suono viene presentato attraverso le cuffie ed è stato registrato tramite un'altra testa con superfici dell'orecchio esterno di forma diversa, i motivi direzionali differiscono da quelli dell'ascoltatore e compaiono problemi quando si cerca di valutare le direzioni nel piano mediano con queste orecchie estranee. Di conseguenza, possono apparire permutazioni fronte-retro o localizzazione all'interno della testa quando si ascoltano registrazioni di teste fittizie o altrimenti indicate come registrazioni binaurali. È stato dimostrato che i soggetti umani possono localizzare in modo monofonico il suono ad alta frequenza ma non il suono a bassa frequenza. La localizzazione binaurale, tuttavia, era possibile con frequenze più basse. Ciò è probabilmente dovuto al fatto che la pinna è abbastanza piccola da interagire solo con onde sonore ad alta frequenza. [17] Sembra che le persone possano localizzare accuratamente solo l'elevazione di suoni complessi che includono frequenze superiori a 7000 Hz e deve essere presente un pinna. [18]

Indizi binaurali dinamici [modifica]
Quando la testa è ferma, i segnali binaurali per la localizzazione del suono laterale (differenza di tempo interaurale e differenza di livello interaurale) non forniscono informazioni sulla posizione di un suono sul piano mediano. ITD e ILD identici possono essere prodotti da suoni a livello degli occhi o a qualsiasi altezza, purché la direzione laterale sia costante. Tuttavia, se la testa viene ruotata, ITD e ILD cambiano in modo dinamico e tali modifiche sono diverse per i suoni a quote diverse. Ad esempio, se una fonte sonora all'altezza degli occhi è dritta e la testa gira a sinistra, il suono diventa più forte (e arriva prima) all'orecchio destro che a sinistra. Ma se la sorgente sonora è direttamente ambientale, non ci saranno cambiamenti in ITD e ILD mentre la testa gira. Elevazioni intermedie produrranno gradi intermedi di cambiamento e se la presentazione di segnali binaurali alle due orecchie durante il movimento della testa viene invertita, il suono verrà udito dietro l'ascoltatore. [13] [19] Hans Wallach [20] ha alterato artificialmente i segnali binaurali di un suono durante i movimenti della testa. Sebbene il suono fosse obiettivamente posizionato all'altezza degli occhi, i cambiamenti dinamici a ITD e ILD mentre la testa ruotava erano quelli che sarebbero stati prodotti se la sorgente sonora fosse stata elevata. In questa situazione, il suono è stato ascoltato all'elevazione sintetizzata. Il fatto che le fonti sonore rimangano oggettivamente all'altezza degli occhi ha impedito ai segnali monofonici di specificare l'elevazione, dimostrando che fu il cambiamento dinamico nei segnali binaurali durante il movimento della testa che permise al suono di essere correttamente localizzato nella dimensione verticale. I movimenti della testa non devono essere prodotti attivamente; un'accurata localizzazione verticale avveniva in una configurazione simile quando la rotazione della testa veniva prodotta passivamente, facendo sedere il soggetto bendato su una sedia rotante. Finché i cambiamenti dinamici nei segnali binaurali hanno accompagnato una rotazione della testa percepita, è stata percepita l'elevazione sintetizzata. [13]

Distanza della sorgente sonora [modifica]
[citazione necessaria]

Il sistema uditivo umano ha solo limitate possibilità di determinare la distanza di una sorgente sonora. Nel range ravvicinato ci sono alcune indicazioni per la determinazione della distanza, come differenze di livello estreme (ad es. Quando sussurra in un orecchio) o risonanze pinna specifiche (la parte visibile dell'orecchio) nel range ravvicinato.

Il sistema uditivo utilizza questi indizi per stimare la distanza da una sorgente sonora:

Rapporto diretto / riflesso: in ambienti chiusi, due tipi di suono arrivano all'ascoltatore: il suono diretto arriva alle orecchie dell'ascoltatore senza essere riflesso su un muro. Il suono riflesso è stato riflesso almeno una volta su un muro prima di arrivare all'ascoltatore. Il rapporto tra suono diretto e suono riflesso può fornire un'indicazione sulla distanza della sorgente sonora.
Loudness: le fonti sonore distanti hanno un volume più basso di quelle vicine. Questo aspetto può essere valutato soprattutto per fonti sonore ben note.
Spettro sonoro: le alte frequenze vengono smorzate più rapidamente dall'aria rispetto alle basse frequenze. Pertanto, una sorgente sonora distante suona più ovattata di una vicina, perché le alte frequenze sono attenuate. Per il suono con uno spettro noto (ad esempio il parlato), la distanza può essere stimata approssimativamente con l'aiuto del suono percepito.
ITDG: Il gap di ritardo temporale iniziale descrive la differenza di tempo tra l'arrivo dell'onda diretta e la prima forte riflessione all'ascoltatore. Le fonti vicine creano un ITDG relativamente grande, con le prime riflessioni che hanno un percorso più lungo da prendere, forse molte volte più a lungo. Quando la sorgente è lontana, le onde sonore dirette e riflesse hanno percorsi simili.
Movimento: simile al sistema visivo, esiste anche il fenomeno della parallasse del movimento nella percezione acustica. Per un ascoltatore in movimento le fonti sonore vicine passano più velocemente delle fonti sonore distanti.
Differenza di livello: sorgenti sonore molto vicine causano un livello diverso tra le orecchie.
Elaborazione del segnale [modifica]
L'elaborazione del suono del sistema uditivo umano viene eseguita nelle cosiddette bande critiche. La gamma dell'udito è segmentata in 24 bande critiche, ciascuna con una larghezza di 1 corteccia o 100 Mel. Per un'analisi direzionale i segnali all'interno della banda critica vengono analizzati insieme.

Il sistema uditivo è in grado di estrarre il suono di una sorgente sonora desiderata da disturbi interferenti. Ciò consente all'ascoltatore di concentrarsi su un solo oratore se anche altri oratori stanno parlando (l'effetto cocktail party). Con l'aiuto dell'effetto cocktail party, il suono proveniente da direzioni interferenti viene percepito attenuato rispetto al suono proveniente dalla direzione desiderata. Il sistema uditivo può aumentare il rapporto segnale-rumore fino a 15 dB, il che significa che il suono interferente viene percepito come attenuato a metà (o meno) del suo volume reale. [Citazione necessaria]

Localizzazione in ambienti chiusi [modifica]
Nelle stanze chiuse non solo il suono diretto proveniente da una sorgente sonora arriva alle orecchie dell'ascoltatore, ma anche il suono che è stato riflesso alle pareti. Il sistema uditivo analizza solo il suono diretto, [12] che arriva per primo, per localizzazione del suono, ma non il suono riflesso, che arriva più tardi (legge del primo fronte d'onda). Quindi una corretta localizzazione rimane possibile anche in un ambiente ecologico. Questa cancellazione dell'eco si verifica nel Nucleo dorsale del Lemnisco laterale (DNLL). [Citazione necessaria]

Al fine di determinare i periodi di tempo in cui prevale il suono diretto e che possono essere utilizzati per la valutazione direzionale, il sistema uditivo analizza i cambiamenti di intensità in diverse bande critiche e anche la stabilità della direzione percepita. Se c'è un forte attacco del volume in diverse bande critiche e se la direzione percepita è stabile, questo attacco è probabilmente causato dal suono diretto di una sorgente sonora, che sta entrando di nuovo o che sta cambiando le sue caratteristiche del segnale. Questo breve periodo di tempo viene utilizzato dal sistema uditivo per l'analisi direzionale e sonora di questo suono. Quando i riflessi arrivano un po 'più tardi, non aumentano il volume interno delle bande critiche in un modo così forte, ma i segnali direzionali diventano instabili, perché c'è un mix di suoni in diverse direzioni di riflessione. Di conseguenza, il sistema uditivo non avvia alcuna nuova analisi direzionale.

Questa prima direzione rilevata dal suono diretto viene presa come direzione della sorgente sonora trovata, fino a quando altri forti attacchi di volume, combinati con informazioni direzionali stabili, indicano che è possibile una nuova analisi direzionale. (vedi effetto Franssen)







\subsection{Differenza di tempo}

La grandezza che misura la prima di queste due differenze è nota come
ITD (interaural time difference) ed è calcolata, in prima approssimazione,
dalla formula (vedi fig. 18):

\begin{equation}
\Delta t = \frac{d sin\theta}{c}
\end{equation}

dove

$\Delta t = ITD$, $d$ è il diametro della testa umana (convenzionalmente $0,18mt$),
$\theta$ è angolo di incidenza (in radianti) e $c$ è la velocità del suono.

Nella realtà, questa è una formula semplificata, in quanto non tiene conto del
percorso extra necessario al suono per girare intorno alla testa, ma comunque
valida per un calcolo della ITD in prima approssimazione. Assumendo
convenzionalmente come sferica la forma della testa umana, è possibile ricavare
una formula più precisa per il calcolo della ITD.

\begin{equation}
\Delta t = \frac{r (\theta + sin\theta)}{c}
\end{equation}

dove $r$ è il raggio ($0,09mt$)

Applicando questa formula, è possibile ricavare il valore massimo della ITD,
ossia il valore di differenza interaurale per un suono che arrivi con angolo
di incidenza pari a $90°$ ($\pi/2$)

\begin{equation}
ITDmax = \frac{0.09 (\pi/2 + sin(\pi/2))}{344} = 0.000673
\end{equation}

che dà un risultato pari a $673μs$.

\subsection{Differenza di intensità}

La differenza di intensità con cui un suono con una certa angolazione arriva
alle due orecchie è data dall’effetto di mascheramento dovuto alla testa,
l’effetto è differente per i suoni acuti
per i suoni gravi, a causa della capacità superiore dei suoni gravi a “girare”
intorno agli ostacoli (diffrazione acustica).

La grandezza di questa differenza di intensità è nota come IID (interaural
intensity difference), e come abbiamo evidenziato, varia col variare della
frequenza, generando così una famiglia di curve.

\subsection{Ripresa e ascolto stereofonico}

Mettendo in relazione i concetti di percezione con le differenze di intensità
(IID) e di tempo (ITD), possiamo esaminare la relazione che intercorre tra le
diverse tecniche di ripresa stereofonica. Nella fig. 22 vediamo la tecnica di
ripresa con coppia cardioide coincidente: il suono arriva ai due microfoni
senza differenze temporali, essendo le due capsule equidistanti dalla fonte
sonora, e la differenza tra i due canali è data quindi dalla differenza di
intensità, in quanto la caratteristica cardioide fa sì che l’intensità del suono
sia differente a seconda che il suono arrivi in asse o fuori asse rispetto al
microfono, e massimamente per le frequenze più acute.

Nella fig. 23 una coppia di microfoni omnidirezionali spaziati fra di loro genera
una stereofonia dovuta al ritardo con cui il suono arriva ai due microfoni,
mentre la curva polare omnidirezionale minimizza le differenze di intensità, a
meno che la coppia sia distanziata in modo significativo, aggiungendo in quel
caso anche una sensibile differenza di intensità.

Nella tecnica di registrazione ORTF, con coppia cardioide quasi-coincidente
angolata di 110° e spaziata di circa 17 cm, abbiamo una prima approssimazione
di una tecnica che tenga conto sia delle ITD che delle IID.

La spaziatura di 17 cm, infatti, è molto vicina alla spaziatura tra le orecchie
di una testa reale, mentre l’angolatura dei microfoni simula la “risposta in frequenza”
del nostro orecchio esterno, assegnando maggiore sensibilità ai suoni frontali.
Infine, un ulteriore metodo di registrazione consiste nella tecnica “binaurale”,
ossia in una tecnica basata sull’uso di una testa artificiale, all’interno della
quale è inserita, in corrispondenza delle orecchie, una coppia di microfoni.

Per comprendere appieno il principio di tale tecnica, occorre considerare
l’esistenza di un ulteriore fattore che, unitamente alla ITD e alla IID
contribuisce alla localizzazione della provenienza del suono. Questo fattore è
dato dalla conformazione del nostro orecchio esterno (“pinna”), della nostra testa
(“head”) e del busto (“torso”). Tale conformazione, che appartiene a ciascun
individuo e che contribuisce all’affinamento delle capacità dell’apparato uditivo,
fa sì che, ad esempio, vengano distinti i suoni di provenienza anteriore da
quelli di provenienza posteriore, come quelli provenienti dall’alto o dal basso,
a parità di angolazione.

La tecnica binaurale avvicina molto l’ascolto di musica riprodotta all’ascolto
naturale, ma presenta il principale inconveniente che è valida solo se l’ascolto
avviene tramite cuffia. L’ascolto tramite altoparlanti, infatti, presenta un
fenomeno denominato “cross- talk”: mentre con una cuffia i segnali dei due canali
sono perfettamente separati e raggiungono solo ed esclusivamente le rispettive
orecchie, il suono proveniente dagli altoparlanti, raggiunge entrambe le orecchie,
impedendo una corretta ricostruzione delle ITD e delle IID.

L’effetto di cross-talk può essere eliminato attraverso appositi circuiti che
implementano questa funzione (in questo caso si parla di registrazioni “transaurali”),
ma queste tecniche decadono nel momento in cui la testa non sia perfettamente
immobile sul piano orizzontale.
