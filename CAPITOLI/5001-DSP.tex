%!TEX TS-program = xelatex
%!TEX encoding = UTF-8 Unicode
% !TEX root = ../metm.tex

\chapter{DIGITAL SIGNAL PROCESSING}

\section{Segnale}

Partendo dall'etimologia latina del termine \emph{signale}, sostantivo di
\emph{signalis} derivato da \emph{signum}, \emph{segno}, definiamo genericamente
un'indicazione sensoriale convenzionale, stabilita d'intesa, per dare una comunicazione,
un avvertimento, un ordine. Esempi ne sono le espressioni: \emph{dare il segnale},
\emph{attendere il segnale} oppure le innumerevoli indicazioni dei segnali stradali.

Una comunicazione, quindi, di cui se ne condividano le modalità e se ne
comprendano i contenuti.

Oltre la portata sensoriale, direttamente percepita dai sensi, definiamo
una comunicazione a distanza \emph{tele-comunicazione}, dall'etimologia greca
del termine \emph{têle}, \emph{lontano}.

Nelle telecomunicazioni definiamo segnale la variazione in funzione del tempo
di una grandezza fisica utilizzata per convogliare informazioni o dello stato
fisico di un sistema.

Il segnale in questione può essere

\begin{description}
  \item[segnale analogico], segnale a tempo continuo e ad ampiezza continua in
       cui la grandezza caratteristica che trasporta
       informazioni può assumere in ogni istante un qualsivoglia valore
       all'interno di un intervallo continuo
  \item[segnale digitale] segnale a tempo discreto e ad ampiezza quantizzata
\end{description}

Un segnale può anche essere periodico o non periodico, si dice periodico quando
una parte di questo si ripete nel tempo ugualmente. L'intervallo di tempo in cui
si ripete la parte è detto periodo.

Nelle telecomunicazioni, dal punto di vista del tipo di informazione trasportata
fino all'utente si può distinguere essenzialmente tra:

\begin{description}
\item[segnale audio]
\item[segnale video]
\item[segnale dati]
\end{description}

ciascuno con caratteristiche diverse in termine di banda di trasmissione richiesta.

Dal punto di vista della tipologia fisica del segnale si ha:

\begin{description}
\item[segnale elettrico]
\item[segnale elettromagnetico]
\item[segnale acustico]
\end{description}

% In elettronica un segnale viene dunque studiato attraverso un modello matematico
% o funzione in cui il tempo (o il suo inverso, la frequenza) è considerato
% variabile indipendente.
%
% La teoria dei segnali studia la rappresentazione dei segnali in modo da poter poi
% manipolarli e trattarli matematicamente.

\section{Teorema del campionamento}

Il teorema del campionamento di Nyquist-Shannon o semplicemente teorema del campionamento,
il cui nome si deve a Harry Nyquist e Claude Shannon, definisce la minima frequenza,
detta frequenza di Nyquist (o anche cadenza di Nyquist), necessaria per campionare
un segnale analogico senza perdere informazioni, e per poter quindi ricostruire
il segnale analogico tempo continuo originario. In particolare, il teorema afferma
che, data una funzione la cui trasformata di Fourier sia nulla al di fuori di un
certo intervallo di frequenze (ovvero un segnale a banda limitata), nella sua
conversione analogico-digitale la minima frequenza di campionamento necessaria per
evitare aliasing e perdita di informazione nella ricostruzione del segnale analogico
originario (ovvero nella riconversione digitale-analogica) deve essere maggiore
del doppio della sua frequenza massima.
Il teorema, comparso per la prima volta nel 1949 in un articolo di C. E. Shannon,
dovrebbe chiamarsi Whittaker-Nyquist-Kotelnikov-Shannon (WNKS), secondo l'ordine
cronologico di chi ne dimostrò versioni via via più generalizzate.

Il campionamento è il primo passo del processo di conversione analogico-digitale
di un segnale. Consiste nel prelievo di campioni (in inglese \emph{samples}) da
un segnale analogico e continuo nel tempo ad intervallo di tempo regolare.
Tale intervallo può essere indicato con un valore di valore $\Delta t$ è denominato
\emph{intervallo di campionamento}. La \emph{frequenza di campionamento} è descrivibile quindi
con il reciproco di $\Delta t$: $fc = \frac{1}{\Delta t}$\footnote{fc ed fs sono equivalenti, indicando rispettivmente frequenza di camionamento e sampling frequency}.
Il risultato del campionamento è un segnale analogico in tempo discreto, che viene
misurato, quantizzato, codificato e reso accessibile digitalmente.

Il teorema di Nyquist-Shannon (o teorema del campionamento dei segnali) stabilisce che,
dato un segnale analogico definito nel tempo $s(t)$ la cui banda di frequenze sia limitata da una frequenza massima $fmax$
il segnale $s(t)$ può essere descritto da campioni presi a frequenza $fc > 2fmax$.
