%!TEX TS-program = xelatex
%!TEX encoding = UTF-8 Unicode
% !TEX root = ../metm.tex

\chapter{DIGITAL SIGNAL PROCESSING}

\section{Segnale}

Partendo dall'etimologia latina del termine \emph{signale}, sostantivo di
\emph{signalis} derivato da \emph{signum}, \emph{segno}, definiamo genericamente
un'indicazione sensoriale convenzionale, stabilita d'intesa, per dare una
comunicazione, un avvertimento, un ordine. Esempi ne sono le espressioni:
\emph{dare il segnale}, \emph{attendere il segnale} oppure le innumerevoli
indicazioni dei segnali stradali.

Una comunicazione, quindi, di cui se ne condividano le modalità e se ne
comprendano i contenuti.

Oltre la portata sensoriale, direttamente percepita dai sensi, definiamo
una comunicazione a distanza \emph{tele-comunicazione}, dall'etimologia greca
del termine \emph{têle}, \emph{lontano}.

Nelle telecomunicazioni definiamo segnale la variazione in funzione del tempo
di una grandezza fisica utilizzata per convogliare informazioni o dello stato
fisico di un sistema.

Il segnale in questione può essere

\begin{description}
  \item[segnale analogico], segnale a tempo continuo e ad ampiezza continua in
       cui la grandezza caratteristica che trasporta
       informazioni può assumere in ogni istante un qualsivoglia valore
       all'interno di un intervallo continuo
  \item[segnale digitale] segnale a tempo discreto e ad ampiezza quantizzata
\end{description}

Un segnale può anche essere periodico o non periodico, si dice periodico quando
una parte di questo si ripete nel tempo ugualmente. L'intervallo di tempo in cui
si ripete la parte è detto periodo.

Nelle telecomunicazioni, dal punto di vista del tipo di informazione trasportata
fino all'utente si può distinguere essenzialmente tra:

\begin{description}
\item[segnale audio]
\item[segnale video]
\item[segnale dati]
\end{description}

ciascuno con caratteristiche diverse in termine di banda di trasmissione richiesta.

Dal punto di vista della tipologia fisica del segnale si ha:

\begin{description}
\item[segnale elettrico]
\item[segnale elettromagnetico]
\item[segnale acustico]
\end{description}

% In elettronica un segnale viene dunque studiato attraverso un modello matematico
% o funzione in cui il tempo (o il suo inverso, la frequenza) è considerato
% variabile indipendente.
%
% La teoria dei segnali studia la rappresentazione dei segnali in modo da poter poi
% manipolarli e trattarli matematicamente.

\section{Teorema del campionamento}

Il teorema del campionamento di Nyquist-Shannon o semplicemente teorema del
campionamento, il cui nome si deve a Harry Nyquist e Claude Shannon, definisce
la minima frequenza, detta frequenza di Nyquist (o anche cadenza di Nyquist),
necessaria per campionare un segnale analogico senza perdere informazioni, e per
poter quindi ricostruire il segnale analogico tempo continuo originario. In
particolare, il teorema afferma che, data una funzione la cui trasformata di
Fourier sia nulla al di fuori di un certo intervallo di frequenze (ovvero un
segnale a banda limitata), nella sua conversione analogico-digitale la minima
frequenza di campionamento necessaria per evitare aliasing e perdita di
informazione nella ricostruzione del segnale analogico originario (ovvero nella
riconversione digitale-analogica) deve essere maggiore del doppio della sua
frequenza massima. Il teorema, comparso per la prima volta nel 1949 in un
articolo di C. E. Shannon, dovrebbe chiamarsi Whittaker-Nyquist-Kotelnikov-Shannon
(WNKS), secondo l'ordine cronologico di chi ne dimostrò versioni via via più
generalizzate.

Il campionamento è il primo passo del processo di conversione analogico-digitale
di un segnale. Consiste nel prelievo di campioni (in inglese \emph{samples}) da
un segnale analogico e continuo nel tempo ad intervallo di tempo regolare.
Tale intervallo può essere indicato con un valore di valore $\Delta t$ è denominato
\emph{intervallo di campionamento}. La \emph{frequenza di campionamento} è
descrivibile quindi con il reciproco di
$\Delta t$: $fc = \frac{1}{\Delta t}$\footnote{fc ed fs sono equivalenti,
indicando rispettivmente frequenza di camionamento e sampling frequency}.
Il risultato del campionamento è un segnale analogico in tempo discreto,
che viene misurato, quantizzato, codificato e reso accessibile digitalmente.

Il teorema di Nyquist-Shannon (o teorema del campionamento dei segnali) stabilisce
che, dato un segnale analogico definito nel tempo $s(t)$ la cui banda di frequenze
sia limitata da una frequenza massima $fmax$ il segnale $s(t)$ può essere descritto
da campioni presi a frequenza $fc > 2fmax$.

\section{Filtri Digitali}

% dal computer music tutorial
Gli esperimenti sull'implementazione digitale dei filtri risalgono ai primi anni 50.
Diverse fasi di sviluppo matematico e concettuale hanno permesso solo dagli anni 60
una vera teorizzazione e semplice implementazione. Solo dagli anni 80 sono stati
possibili implementazioni in tempo reale per un utilizzo musicale.

\begin{quote}
  Un filtro digitale è un processo di calcolo o un algoritmo attraverso il quale
  un segnale digitale o una sequenza di numeri (la sequenza di entrata) è
  trasformata in una seconda sequenza di numeri denominata segnale digitale di
  uscita.\footnote{Rabiner 1972.}
\end{quote}

In questo senso quindi ogni strumento digitale con un'entrata ed un'uscita è un
filtro digitale. Comunemente l'uso del termine \emph{filtro} descrive un dispositivo
in grado di amplificare o attenuare porzioni di spettro. Ma anche riverberi e linee
di ritardo sono filtri, questo dovrebbe suggerirci l'idea che un filtro può
modificare lo spettro di un segnale entrante, ma anche la sua struttura temporale,
sia su una scala molto piccola che su una molto ampia e che quindi struttura
spettrale e struttura temporale di un segnale sono in relazione diretta tra loro.

Un'equazione di un filtro digitale non necessariamente rivela le sue caratteristiche
audio. In letteratura un filtro è rappresentato attraverso la \tz.
La \tz~ mappa gli effetti dei ritardi del campione in un'immagine
bidimensionale del dominio della frequenza denominato \emph{piano complesso zeta}.
Su questo piano vengono rappresenntati \emph{poli} per i picchi di risonanza e
\emph{zeri} per i punti di ampiezza nulla dell'uscita del filtro. Un \emph{filtro
a due poli}, per esempio, indica due picchi di risonanza.

esempio due poli

La \tz è un concetto importante per lo sviluppo di un filtro in quanto rappresentazione
matematica di passaggio tra le caratteristiche del filtro ed i suoi parametri di
implementazione. Rimane però un'applicazione astratta con relazioni solo indirette
con i parametri fisici del filtro.

Nella rappresentazione software di un filtro si opera in processi applicati
ai campioni audio, in termini di operazioni matematiche e ritardi a questi applicati.
Generalmente quindi si descrive un filtro con diagrammi di flusso che ne dichiarino
il percorso, rispopste all'impulso e risposte in frequenza.

\subsection{Risposte all'impulso, in frequenza e in fase.}

Si può osservare l'effetto di un filtro sia nel dominio del tempo che in quello
della frequenza. Un'immagine \emph{prima} ed una \emph{dopo} l'applicazione, possono mostrare l'effetto del filtro. Senza dubbio alcune tipologie di segnali d'entrata
possono mostrare l'effetto del filtro più chiaramente di altre. Per testare un
filtro abbiamo bisogno di un segnale di entrata contenente tutte le frequenze necessarie.

Il \rb, che contiene tutte le frequenze ci indica come un filtro risponde nel dominio
della frequenza. 
