%!TEX TS-program = xelatex
%!TEX encoding = UTF-8 Unicode
% !TEX root = ../../metm.tex

\section{Teorema del campionamento}

Il teorema del campionamento di \emph{Nyquist-Shannon} o semplicemente teorema del
campionamento, il cui nome si deve a Harry Nyquist \index[names]{Nyquist, Harry} e
Claude Shannon \index[names]{Shannon, Claude}, definisce la minima frequenza, detta
frequenza di Nyquist (o anche cadenza di Nyquist), necessaria per campionare un
segnale analogico senza perdere informazioni, e per poter quindi ricostruire il
segnale analogico, tempo-continuo, originario.
In particolare il teorema afferma che,
%data una funzione la cui trasformata di
%Fourier sia nulla al di fuori di un certo intervallo di frequenze (ovvero un
%segnale a banda limitata), nella sua conversione analogico-digitale
la minima frequenza di campionamento necessaria per evitare difetti e perdita di
informazione nella ricostruzione del segnale analogico originario (ovvero nella
riconversione digitale-analogica) deve essere maggiore del doppio della sua
frequenza massima.
% Il teorema, comparso per la prima volta nel 1949 in un
% articolo di C. E. Shannon, dovrebbe chiamarsi Whittaker-Nyquist-Kotelnikov-Shannon
% (WNKS), secondo l'ordine cronologico di chi ne dimostrò versioni via via più
% generalizzate.

Il campionamento è il primo passo del processo di conversione analogico-digitale
di un segnale. Consiste nel prelievo di campioni (in inglese \emph{samples}) da
un segnale analogico e continuo nel tempo ad intervallo di tempo regolare.
Tale intervallo può essere indicato con un valore di valore $\Delta t$ è denominato
\emph{intervallo di campionamento}. La \emph{frequenza di campionamento} è
descrivibile quindi con il reciproco di
$ \Delta t $: $ fc = 1/\Delta t $\footnote{fc ed fs sono equivalenti,
indicando rispettivmente \emph{frequenza di camionamento} e s\emph{ampling frequency}}.
Il risultato del campionamento è un segnale digitale in tempo discreto,
misurato, quantizzato, codificato e reso accessibile digitalmente.

Il teorema di Nyquist-Shannon (o teorema del campionamento dei segnali) stabilisce
che, dato un segnale analogico definito nel tempo $s(t)$ la cui banda di frequenze
sia limitata da una frequenza massima $fmax$ il segnale $s(t)$ può essere descritto
da campioni presi a frequenza

\begin{equation}
\label{sampling}
fc > 2fmax
\end{equation}

\clearpage
