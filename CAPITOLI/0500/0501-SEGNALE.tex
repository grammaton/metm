%!TEX TS-program = xelatex
%!TEX encoding = UTF-8 Unicode
% !TEX root = ../../metm.tex

\section{Segnale}

Partendo dall'etimologia latina del termine \emph{signale}, sostantivo di
\emph{signalis} derivato da \emph{signum}, \emph{segno}, definiamo genericamente
un'indicazione sensoriale convenzionale, stabilita d'intesa, per dare una
comunicazione, un avvertimento, un ordine. Esempi ne sono le espressioni:
\emph{dare il segnale}, \emph{attendere il segnale} oppure le innumerevoli
indicazioni dei segnali stradali.

Una comunicazione, quindi, di cui se ne condividano le modalità e se ne
comprendano i contenuti.

Oltre la portata sensoriale, direttamente percepita dai sensi, definiamo
una comunicazione a distanza \emph{tele-comunicazione}, dall'etimologia greca
del termine \emph{têle}, \emph{lontano}.

Nelle telecomunicazioni definiamo segnale la variazione in funzione del tempo
di una grandezza fisica utilizzata per convogliare informazioni o dello stato
fisico di un sistema.

Il segnale in questione può essere

\begin{description}
  \item[segnale analogico], segnale a tempo continuo e ad ampiezza continua in
       cui la grandezza caratteristica che trasporta
       informazioni può assumere in ogni istante un qualsivoglia valore
       all'interno di un intervallo continuo
  \item[segnale digitale] segnale a tempo discreto e ad ampiezza quantizzata
\end{description}

Un segnale può anche essere periodico o non periodico, si dice periodico quando
una parte di questo si ripete nel tempo ugualmente. L'intervallo di tempo in cui
si ripete la parte è detto periodo.

Nelle telecomunicazioni, dal punto di vista del tipo di informazione trasportata
fino all'utente si può distinguere essenzialmente tra:

\begin{description}
\item[segnale audio]
\item[segnale video]
\item[segnale dati]
\end{description}

ciascuno con caratteristiche diverse in termine di banda di trasmissione richiesta.

Dal punto di vista della tipologia fisica del segnale si ha:

\begin{description}
\item[segnale elettrico]
\item[segnale elettromagnetico]
\item[segnale acustico]
\end{description}

% In elettronica un segnale viene dunque studiato attraverso un modello matematico
% o funzione in cui il tempo (o il suo inverso, la frequenza) è considerato
% variabile indipendente.
%
% La teoria dei segnali studia la rappresentazione dei segnali in modo da poter poi
% manipolarli e trattarli matematicamente.

\clearpage
