%!TEX TS-program = xelatex
%!TEX encoding = UTF-8 Unicode
% !TEX root = ../../metm.tex

\section{Segnale}

Possiamo stilare una lista sconfinata di esempi comuni di segnali della vita
quotidiana, dai quelli acustici di suonerie, cicalini di avvertimento, a quelli
complessi provenienti da strumenti musicali. Segnali prodotti da macchine che
rilevano variazioni di stato e che usiamo per tradurre campi fisici diversi:
di questa lista il sismografo per esempio crea un debole segnale elettrico di cui
abbiamo la chiara visione stampata su carta.

Partendo dall'etimologia latina del termine \emph{signale}, sostantivo di
\emph{signalis} derivato da \emph{signum}, \emph{segno}, definiamo genericamente
un'indicazione sensoriale convenzionale, stabilita d'intesa, per dare una
comunicazione, un avvertimento, un ordine. Esempi ne sono le espressioni:
\emph{dare il segnale}, \emph{attendere il segnale} oppure le innumerevoli
indicazioni dei segnali stradali.

Una comunicazione, quindi, di cui se ne condividano le modalità e se ne
comprendano i contenuti: il segnale si fa \emph{portatore di una informazione}
che giustifica l'esistenza e l'importanza del segnale stesso. Una lampadina ed un
semaforo, pur appartenenti alla categoria dei segnali luminosi, assumono
significati diversi.

Sulla base di queste considerazoni possiamo dire che un segnale \emph{è una
qualunque grandezza fisica variabile cui è associata unna informazione.} In molti
casi del segnale si sanno molte cose, in funzione di quanto siamo in grado di
annotare o registrare, su carta, su nastro magnetico, dentro un file digitale.

Oltre la portata sensoriale, ovvero direttamente percepita dai sensi, possiamo definire
una comunicazione, attraverso segnali a distanza, come \emph{tele-comunicazione},
dall'etimologia greca del termine \emph{têle}, \emph{lontano}. Da lontano
trasmettiamo e riceviamo il segnale radio, un campo elettromagnetico variabile.

Nelle telecomunicazioni definiamo segnale la variazione in funzione del tempo
di una grandezza fisica utilizzata per convogliare informazioni o dello stato
fisico di un sistema.

La funzione matematica di una o più variabili nel tempo diventa quindi lo
strumento ottimale di descrizione, rappresentazione ed elaborazione del segnale.

Prima di inoltrarsi nella descrizione delle tipologie di segnali, è necessario
precisare che il segnale come rappresentazione, seppur estremamente dettagliato,
veicolo di un'enormità di informazioni, è pur sempre una rappresentazione. Spesso
ci troviamo in condizioni di lavorare su segnali che sono rappresentazioni di
rappresentazioni. Un segnale acustico strumentale, uno squillo di tromba,
nella sua rappresentazione elettrica cede delle informazioni. Lo stesso segnale
elettrico può diventare a sua volta segnale digitale dello stesso fenomeno acustico,
in una rappresentazione della rappresentazione. Questa perdita di informazioni
deve costantemente rimanere sorvegliata e considerata parte delle nostre capacità
di trattamento dei segnali, nel rispetto della significativa differenza tra
l'informazione fisica ed una sua qualsiasi rappresentazione.

Ci sono molti esempi di segnali di segnali a cui siamo abituati e che non analizziamo
mai nell'ottica della perdita. Su un set cinematografico un'azione viene ripresa
da una cinepresa IMAX con pellicola da 70mm. Sono dati inutili, di circostanza.
Il meccanismo della ripresa in pellicola di un'azione filmica genera un segnale,
quello stampato sulla pellicola, discreto, perché fatto di un numero finito di
fotogrammi al secondo. Tra un fotogramma ed il successivo non ci sono informazioni.
La visione al proiettore cinematografico di quella pellicola crea un flusso
continuo ai nostri occhi, ed il segnale registrato acquisice movimento fluido su
una decina di metri di schermo. La stessa azione, la possiamo vedere a casa,
dopo qualche tempo, in un segnale discreto, di un discreto diverso da quello
originale, in una rappresentazione domestica, della rappresentazione cinematografica,
dell'azione iniziale.

\subsection{Tipologie di segnali}

Abbiamo descritto il segnale come veicolo di informazioni nel tempo. Le grandezze
che misuriamo nel tempo descritte dal segnale possono essere distinte in due
tipologie:

\begin{description}
  \item[segnali a tempo continuo] in cui la grandezza variabile nel tempo può
  assumere con continuità tutti i valori compresi entro un certo intervallo,
  eventualmente illimitato. La variabile temporale assumerà il nome di $t$ mentre
  il segnale sarà indicato con $x(t)$, $y(t)$.
  \item[segnali a tempo discreto] sono delle successioni, delle sequenze $x[n]$,
  $y[n]$ dove la variabile tempora $n$ è il numero intero rappresentante la posizione
  all'innterno della successione.
\end{description}

Le variazioni delle grandezze misurate sul \emph{codominio} della della funzione
che rappresenta il segnale può essere, secondo le due categorie sopra esposte, di
altrettante tipologie:

\begin{description}
  \item[segnali ad ampiezza continua] che possono assumere con continuità tutti
  i valori reali di un intervallo (anche illimitato) come nel caso di un segnale
  acustico, o in generale nei segnali osservati nei sistemi naturali.
  \item[segnali ad ampiezza discreta] aventi un numero finito di misurazioni,
  quindi appartenenti ad un insieme numerabile, eventualmente illimitato.
  Il comportamento della luce del semaforo, o in generale dei sistemi binari, può
  assumere solo due stati, acceso spento.
\end{description}

Le caratteristiche dei segnali sopra descritti definiscono le due categorie

\begin{description}
  \item[segnale analogico] a tempo continuo e ad ampiezza continua
  \item[segnale digitale] (o numerico) a tempo discreto e ad ampiezza quantizzata
\end{description}

Un segnale può anche essere periodico o non periodico, si dice periodico quando
una parte di questo si ripete nel tempo ugualmente. L'intervallo di tempo in cui
si ripete la parte è detto periodo.

Nelle telecomunicazioni, dal punto di vista del tipo di informazione trasportata
fino all'utente si può distinguere essenzialmente tra:

\begin{description}
\item[segnale audio]
\item[segnale video]
\item[segnale dati]
\end{description}

ciascuno con caratteristiche diverse in termine di banda di trasmissione richiesta.

Dal punto di vista della tipologia fisica del segnale si ha:

\begin{description}
\item[segnale elettrico]
\item[segnale elettromagnetico]
\item[segnale acustico]
\end{description}

% In elettronica un segnale viene dunque studiato attraverso un modello matematico
% o funzione in cui il tempo (o il suo inverso, la frequenza) è considerato
% variabile indipendente.
%
% La teoria dei segnali studia la rappresentazione dei segnali in modo da poter poi
% manipolarli e trattarli matematicamente.

\clearpage
