%!TEX TS-program = xelatex
%!TEX encoding = UTF-8 Unicode
% !TEX root = ../../metm.tex

\section{Segnale}

Possiamo stilare una lista sconfinata di esempi comuni di segnali utilizzati nella
vita quotidiana, da quelli acustici semplici di suonerie, cicalini di avvertimento,
a quelli complessi provenienti da strumenti musicali. Segnali luminosi, di
avvertimento, una luce lampeggiante, o di istruzione come un semaforo rosso, fino
ai segnali prodotti da macchine che in conseguenza e relazione a variazioni di
fisiche e che usiamo per tradurre campi fisici diversi come il sismografo per
esempio che crea un debole segnale elettrico di cui abbiamo la chiara visione
stampata su carta.

Partendo dall'etimologia latina del termine \emph{signale}, sostantivo di
\emph{signalis} derivato da \emph{signum}, \emph{segno}, definiamo genericamente
un'indicazione sensoriale convenzionale, stabilita d'intesa, per dare una
comunicazione, un avvertimento, un ordine. Esempi ne sono le espressioni:
\emph{dare il segnale}, \emph{attendere il segnale} oppure le innumerevoli
indicazioni dei segnali stradali.

Una comunicazione, quindi, di cui se ne condividano le modalità e se ne
comprendano i contenuti: il segnale si fa \emph{portatore di una informazione}
che giustifica l'esistenza e l'importanza del segnale stesso. Una lampadina ed un
semaforo, pur appartenenti alla categoria dei segnali luminosi, assumono
significati diversi.

Sulla base di queste considerazoni possiamo dire che un segnale \emph{è una
qualunque grandezza fisica variabile cui è associata unna informazione.} In molti
casi del segnale si sanno molte cose, in funzione di quanto siamo in grado di
annotare o registrare, su carta, su nastro magnetico, dentro un file digitale.

Oltre la portata sensoriale, ovvero direttamente percepita dai sensi, possiamo definire
una comunicazione attraverso segnali a distanza come nella \emph{tele-comunicazione},
dall'etimologia greca del termine \emph{têle}, \emph{lontano}. Da lontano
trasmettiamo e riceviamo il segnale radio, un campo elettromagnetico variabile.
Nelle telecomunicazioni definiamo segnale \emph{la variazione in funzione del tempo
di una grandezza fisica}, utilizzata per convogliare informazioni dello stato
fisico di un sistema. La funzione matematica di una o più variabili nel tempo
diventa quindi lo strumento ottimale di descrizione, rappresentazione ed
elaborazione del segnale.

Prima di inoltrarsi nella descrizione delle tipologie di segnali, è necessario
precisare che il segnale come rappresentazione, seppur estremamente dettagliato,
veicolo di un'enormità di informazioni, è pur sempre una rappresentazione. Il
ragionamento si fa più che necessario se consideriamo che spesso ci troviamo in
condizioni di lavorare su segnali che sono rappresentazioni di rappresentazioni.
Un segnale acustico strumentale, uno squillo di tromba, nella sua rappresentazione
elettrica cede delle informazioni. Lo stesso segnale elettrico può diventare a
sua volta segnale digitale dello stesso fenomeno acustico, in una rappresentazione
della rappresentazione. Questa perdita di informazioni deve costantemente rimanere
sorvegliata e considerata parte delle nostre capacità di trattamento dei segnali,
nel rispetto della significativa differenza tra l'informazione fisica ed una sua
qualsiasi rappresentazione.

Ci sono molti esempi di segnali di segnali a cui siamo abituati e che non analizziamo
mai nell'ottica della perdita. Su un set cinematografico un'azione viene ripresa
da una cinepresa \emph{IMAX} con pellicola da $70mm$ (dati riferiti all'ambito
di massima qualità disponbile). Il meccanismo della ripresa in pellicola di
un'azione genera un segnale, quello impresso sulla pellicola, fatto di un numero
finito di fotogrammi al secondo. Tra un fotogramma ed il successivo non ci sono
informazioni. La visione al proiettore cinematografico di quella pellicola crea
un flusso continuo ai nostri occhi, ed il segnale registrato acquisice movimento
(apparentemente) fluido su una decina di metri di schermo (una realtà continua,
diversa da quella \emph{originale}). La stessa azione la possiamo poi rivedere a
casa, dopo qualche tempo, in un segnale diverso da quello cinematografico che
è diverso dall'azione originale, in una rappresentazione domestica, della
rappresentazione cinematografica, dell'azione iniziale.

\subsection{Tipologie di segnali}

Abbiamo descritto il segnale come veicolo di informazioni nel tempo.

Le grandezze che misuriamo nel tempo descritte dal segnale possono essere
distinte in due tipologie:

\begin{description}
  \item[segnali a tempo continuo] in cui la grandezza variabile nel tempo può
  assumere con continuità tutti i valori compresi entro un certo intervallo,
  eventualmente illimitato. La variabile temporale assumerà il nome di $t$ mentre
  il segnale sarà indicato con $x(t)$, $y(t)$.
  \item[segnali a tempo discreto] sono delle successioni, delle sequenze $x[n]$,
  $y[n]$ dove la variabile tempora $n$ è il numero intero rappresentante la posizione
  all'interno della successione.
\end{description}

Le variazioni delle grandezze misurate sul \emph{codominio} della della funzione
che rappresenta il segnale può essere, secondo le due categorie sopra esposte, di
altrettante tipologie:

\begin{description}
  \item[segnali ad ampiezza continua] che possono assumere con continuità tutti
  i valori reali di un intervallo (anche illimitato) come nel caso di un segnale
  acustico, o in generale nei segnali osservati nei sistemi naturali.
  \item[segnali ad ampiezza discreta] aventi un numero finito di misurazioni,
  quindi appartenenti ad un insieme numerabile, eventualmente illimitato.
  Il comportamento della luce del semaforo, o in generale dei sistemi binari, può
  assumere solo due stati, acceso spento.
\end{description}

Le caratteristiche dei segnali sopra descritti definiscono quindi le due categorie:

\begin{description}
  \item[segnale analogico] a tempo continuo e ad ampiezza continua
  \item[segnale digitale] (o numerico) a tempo discreto e ad ampiezza quantizzata
\end{description}

La musica elettronica viene generalmente prodotta utilizzando un computer,
sintetizzando o elaborando segnali audio digitali. Queste sono sequenze di numeri,

\begin{equation}
  \label{digsig}
  ...,x[n-1],x[n],x[n+1],...
\end{equation}

dove l'indice $n$ definisce, per numeri interi, la posizione del campione (il singolo numero),
nel segnale digitale (la sequenza di numeri).

Un segnale può anche essere periodico o non periodico, si dice periodico quando
una parte di questo si ripete nel tempo uguale a se stessa. L'intervallo di tempo
in cui si ripete la parte è detto periodo. Un esempio di segnale audio digitale
periodico è la \emph{sinusoide}:

\begin{equation}
  \label{digsin}
x[n] = a cos(\omega n + \phi)
\end{equation}

dove $a$ è l'ampiezza, $\omega$ è la frequenza angolare e $\phi$ è la fase
iniziale. La fase è una funzione del numero di campione $n$, pari a $\omega n + \phi$
La fase iniziale è la fase nel campione zero (n = 0).

% La Figura 1.1 (parte a) mostra graficamente una sinusoide. L'asse orizzontale mostra i valori successivi di n e l'asse verticale mostra i valori corrispondenti di x [n]. Il grafico è disegnato in modo tale da enfatizzare la natura campionata del segnale. In alternativa, potremmo disegnarlo più semplicemente come una curva continua (parte b). Il disegno superiore è la rappresentazione più fedele della sinusoide (audio digitale), mentre quella inferiore può essere considerata una sua idealizzazione.
% I sinusoidi svolgono un ruolo chiave nell'elaborazione audio perché, se si sposta uno di essi a sinistra o a destra di un numero qualsiasi di campioni, ne si ottiene un altro. Ciò semplifica il calcolo dell'effetto di tutti i tipi di operazioni sui sinusoidi. Le nostre orecchie usano questa stessa proprietà speciale per aiutarci ad analizzare i suoni in arrivo, motivo per cui le sinusoidi e le combinazioni di sinusoidi possono essere utilizzate per ottenere molti effetti musicali.
% I segnali audio digitali non hanno alcuna relazione intrinseca con il tempo, ma per ascoltarli dobbiamo scegliere una frequenza di campionamento, di solito dato il nome della variabile R, che è il numero di campioni che rientrano in un secondo. Il tempo t è correlato

% Nelle telecomunicazioni, dal punto di vista del tipo di informazione trasportata
% fino all'utente si può distinguere essenzialmente tra:
%
% \begin{description}
% \item[segnale audio]
% \item[segnale video]
% \item[segnale dati]
% \end{description}
%
% ciascuno con caratteristiche diverse in termine di banda di trasmissione richiesta.
%
% Dal punto di vista della tipologia fisica del segnale si ha:
%
% \begin{description}
% \item[segnale elettrico]
% \item[segnale elettromagnetico]
% \item[segnale acustico]
% \end{description}

% In elettronica un segnale viene dunque studiato attraverso un modello matematico
% o funzione in cui il tempo (o il suo inverso, la frequenza) è considerato
% variabile indipendente.
%
% La teoria dei segnali studia la rappresentazione dei segnali in modo da poter poi
% manipolarli e trattarli matematicamente.

\clearpage
