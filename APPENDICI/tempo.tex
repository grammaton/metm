%!TEX TS-program = xelatex
%!TEX encoding = UTF-8 Unicode
% !TEX root = ../metp.tex

\chapter{Pensare il tempo}

\begin{flushright}
\textit{- Alice: How long is forever?\\
- White Rabbit: Sometimes, just one second.”}\\
Lewis Carroll, Alice in Wonderland
\end{flushright}

In questa sezione affronteremo un piccolo esercizio di programmazione che porterà
alla creazione di un semplice metronomo digitale. Lo scopo di tale esercizio è
quello di accedere alla programmazione temporizzata degli eventi sonori.

risorse necessarie:\\
\url{http://faustide.grame.fr/}

\startcontents[chapters]
\printcontents[chapters]{}{1}{}

\section{Scansioni ritmiche}

Prima di inoltrarsi nel pratico della creazione, c'è un minimo di riflessione da
fare sul come dividiamo il tempo e perché.

\begin{quote}
Da secoli, \emph{dividiamo} il tempo in giorni. La parola “tempo” deriva da una
radice indoeuropea, \emph{di} o \emph{dai}, che indica “dividere”. Da secoli,
dividiamo il giorno in ore. Per la maggior parte di questi secoli, però, le ore
erano più lunghe d'estate e più corte d'inverno, perché le 12 ore scandivano il
tempo fra alba e tramonto: l'ora prima era l'alba, e l'ora dodicesima il tramonto,
indipendentemente dalla stagione. [\ldots] È solo verso il XIII secolo che in Europa
la vita della gente comincia a essere regolata da orologi meccanici. Città e
villaggi costruiscono la loro chiesa, accanto alla chiesa il campanile, e sul
campanile un orologio che dà il ritmo alle funzioni collettive. Inizia l'era del
tempo regolato da orologi\footnote{Rovelli - l'ordine del tempo. pg 56-57}.
\end{quote}

L'idea di un tempo fluido, che non ticchetta, che non rintocca, era la consuetudine
prima degli orologi. Oggi è molto meno comune un pensiero di questo tipo. Siamo
legati ai ticchettii.
\begin{quote}
L'utilità degli orologi è indicare a tutti la stessa ora. Ma anche quest'idea è
più moderna di quanto possiamo immaginare. Per secoli, finché si viaggiava a
cavallo, a piedi o in carrozza, non c'era motivo di sincronizzare orologi da un
luogo all'altro. C'era un ottimo motivo per \emph{non} farlo: mezzodì è per
definizione quando il sole è più alto nel cielo. [\ldots] Nel XIX secolo arriva
il telegrafo, i treni diventano comuni e veloci, e il problema di sincronizzare
bene gli orologi da una città all'altra si fa importante\footnote{Rovelli - l'ordine del tempo}.
\end{quote}

Anche il metronomo ha la sua storia. Innanzitutto l'etimologia della parola,
che compare per la prima volta nel 1815 per opera di Johann Nepomuk M\"{a}lzel
mediante l'unione di due parole greche: \emph{metron}, misura e \emph{nomos}, regola.
Seppur sotto il nome di M\"{a}lzel, tanto da trovare indicazioni metronomiche
con la sigla \emph{MM} (Metronomo M\"{a}lzel), il dispositivo si basa su un
principio messo a punto da Dietrich Nikolaus Winkel, al quale M\"{a}lzel aggiunse
il tipico rintocco sonoro. Il musicista analogico sa che M\"{a}lzel e Winkel sono
rimasti a lungo due marchi importanti di metronomi meccanici.

Il metronomo meccanico si basa sui concetti dell'oscillazione pendolare:
è infatti costituito da pendolo capovolto, con un'asta graduata fra le 40 scansioni
e le 208 scansioni al minuto primo. Spostando un peso lungo l'asta graduata
selezioniamo la quantità di pulsazioni per minuto.

\section{Beat per minuto}

\emph{Quanti secondi ci sono in un minuto?} Sessanta.

\emph{Quanti minuti ci sono in un'ora?} Sessanta.

\emph{Quanti secondi ci sono in un'ora?} Tremilaseicento ($60*60$).

\emph{Quanto tempo passa tra i rintocchi di un metronomo a $108bpm$?}

L'indicazione metronomica $bpm$ \emph{(Beat per minute)} ci indica quanti rintocchi
vogliamo sentire in un minuto. Ad una quantità di $60bpm$ ascoltiamo un rintocco
sessanta volte per minuto, una scansione che corrisponde a quella dei secondi.
Il metronomo impostato a $60bpm$ produce un rintocco ogni secondo, il secondo è
la distanza che c'è tra un rintocco ed il successivo.

Il minuto è la nostra grandezza di riferimento. La nostra torta intera. Ad una
scansione di $60sec$ dividiamo, come sull'orologio, la torta in sessanta parti
uguali, e avanziamo tra una e l'altra di $1/60$ di minuto (ovvero $1sec$).
Ad una scansione di $108bpm$ (battiti per minuto) stiamo dividendo la torta in
centootto parti uguali ed avaziamo tra una parte e l'altra di $1/108$ di minuto.
Volendo individuare la distanza in secondi tra un rintocco ed il successivo non
ci resta che sostituire il minuto con l'equivalente in secondi, sessanta. La
distanza che separa un beat dal successivo ($\Delta(bpm)$) in una scansione di $108BMP$ può
essere calcolata quindi con $60/108=0,\overline{5}sec$.

\begin{equation}
\Delta(bpm) = 60/bpm (sec)
\end{equation}

Volendo aumentare il grado di precisione del calcolo precedente, ci basterà scendere
ad unità di misura più piccole del secondo, come per esempio il millesimo di secondo.

\emph{Quanti millesimi di secondo ci sono in un minuto?} Sessantamila.

\section{Metronomo}

Il metronomo digitale non può far affidamento su leggi meccaniche e masse in
movimento. Il tempo all'interno del mondo informatico si muove diversamente dagli
orologi. Per un computer non è troppo difficile portare il tempo anche in
presenza di tempi diversi, per esempio, un video multimediale può avere una scansione
di $23.976fps$ (\emph{frame per second}) e, all'interno dello stesso secondo, un suono
può avere $48000Hz$ di sample rate, ovvero essere descritto in un secondo per 48000 punti.

La frequenza di campionamento, \emph{sample rate (sr)}, nel magico mondo dei
segnali audio digitali, è la divisione del secondo per l'unità più piccola descrivibile,
il campione (\emph{sample}). Alla frequenza di campionamento di $48000Hz$ un campione
è grande $1/48000sec$. La torta, il secondo, è divisa in quarantottomila parti.

\emph{A $SR=48000$ quanti campioni ci sono in un minuto?} Due milioni e ottocentoottantamila.

Le frequenze di campionamento, seppur standardizzate, possono essere molteplici in
quanto definiscono quanto è accurata la descrizione del suono digitale nel tempo.

Dovremo riferirci ad una formula che generalizzi il valore della frequenza di campionamento
e sapere esattamente a che frequenza si esegue il calcolo, per ricavare il numero
di campioni che occorrono in ogni minuto, $spm$ (\emph{samples per minute}).

\begin{equation}
spm = SR*60
\end{equation}

Ora che il nostro orologio audio è definito nel numero di campioni che compongono
il minuto, è piuttosto semplice capire a che distanza, in campioni, avviene un
rintocco dal successivo alla velocità metronomica di $108BMP$ ad una frequenza
di campionamento di $48000Hz$, attraverso la formula $48000*60/108=26666,\overline{6}$.
È indispensabile chiarire immediatamente che per il computer non esiste, nella
descrizione del segnale audio, la parte decimale del campione, quindi di quella
dimensione terremo solo la parte intera: un rintocco ogni $26666$ campioni.

Ci sono ora i presupposti concettuali per iniziare a scrivere il codice di programma
che porterà al metronomo digitale.

%-----------------------------------------------------------
%-------------------------------larghezza massima del codice
\begin{lstlisting}
//-------------------------------------------- METRONOMO ---
//------------------------------- dichiarazioni iniziali ---
declare name "METRONOMO";
import("stdfaust.lib");
//------------------- numero di campioni per ogni minuto ---
SPM = (60*ma.SR);
\end{lstlisting}

Con le prime righe di codice abbiamo iniziato, e dato un nome, al progetto metronomo.
Il consueto \emph{import} della libreria standard ci permette di utilizzare
le funzioni standard di Faust. Con la funzione \emph{spm} otteniamo il numero di
campioni per minuto, chiendendo a Faust di verificare a che frequenza di campionamento
è impostato il computer prima di effettuare il calcolo, attraverso la funzione
matematica standard \emph{ma.SR}.

Aggiungiamo al codice precedente il selettore del valore metronomico attraverso
uno slider orizzontale

\begin{lstlisting}
//------------------------------- cursore selezione BPM ---
BPM = hslider("[01]BPM", 60, 40, 240, 0.1);
\end{lstlisting}

Ora che siamo in grado di introdurre il valore metronomico dovremo inserire
la formula che mette in relazione il numero di campioni per ogni minuto con la
quantità di BPM desiderata

\begin{lstlisting}
//-------------------------------- cursore selezione BPM ---
//-------------------- init-min--max-step-------------------
BPM = hslider("[01]BPM", 60, 40, 240, 0.1);
//------------------- numero di campioni per ogni minuto ---
SPM = (60*ma.SR);
//------------- formula di conversione da BPM a campioni ---
tempo = SPM/BPM : int; // solo la parte intera
\end{lstlisting}

Il sistema ora è in grado di fornire la distanza tra un rintocco ed il successivo
al variare dei $bpm$. Facciamolo suonare.

\begin{lstlisting}
//-------------------------------------------- METRONOMO ---
//------------------------------- dichiarazioni iniziali ---
declare name "METRONOMO";
import("stdfaust.lib");
//-------------------------------- cursore selezione BPM ---
//-------------------- init-min--max-step-------------------
BPM = hslider("[01]BPM", 60, 40, 240, 0.1);
//------------------- numero di campioni per ogni minuto ---
SPM = (60*ma.SR);
//------------- formula di conversione da BPM a campioni ---
tempo = SPM/BPM : int; // solo la parte intera
//--------------------------------- impulso temporizzato ---
metronomo = ba.pulsen(1, tempo); // (dimensione - distanza)
//------------------- METRONOMO SUI DUE CANALI DI USCITA ---
process =   metronomo <: _,_ ;
\end{lstlisting}

Il sistema funziona, agendo sul controllo BPM otteniamo quello la variazione di
distanza tra i rintocchi.

L'oggetto \emph{ba.pulsen} produce una pulsazione ad intervalli regolari.
La pulsazione può essere di grandezza variabile, così come la distanza, entrambe
le grandezze espresse in campioni. La riga

\begin{lstlisting}
//--------------------------------- impulso temporizzato ---
metronomo = ba.pulsen(1, tempo); // (dimensione - distanza)
\end{lstlisting}

genera un impulso grande un campione alla distanza in campioni espressa da \emph{tempo}
che sappiamo produrre il numero di campioni in relazione al numero di $bpm$ immesso.
L'impulso grande un campione è un clik unitario, ovvero grande la dimensione
più piccola possibile in un segnale audio ed è definito impulso.
Un impulso di questo tipo, se osservato nel dominio della frequenza contiene tutte
le frequenze dello spettro audio. In questo senso, un segnale contenente ogni
frequenza della banda audio, sprigionata in un impulso, è il segnale perfetto per
far risuonare un filtro, esattamente come farebbe una campana colpita da un martello.
A parità di martello e di energia, più la campana sarà stretta, più sarà lunga la risonanza.

\begin{lstlisting}
//---------------------------------------------- SINTESI ---
//------------------------------------- filtro risonante ---
fireson = fi.highpass(128, 1000) : fi.lowpass(8, 1000);
\end{lstlisting}

Il primo filtro è un filro passa alto del centoventottesimo ordine, intonato a
$1000Hz$ e si comporta come una campana intonata a quella frequenza. Il secondo
filtro è opposto, passa basso, serve a tenere sotto controllo la banda passante nel
versante superiore ed elimina il suono impulsivo iniziale.

Il metronomo è ora completo, di seguito il codice contenente tutti i passaggi
precedenti\footnote{Il codice completo può essere compilato online e scaricato
sul computer o direttamente sul proprio smartphone puntando il \emph{QRcode}.}.

%-----------------------------------------------------------
%-------------------------------larghezza massima del codice
\begin{lstlisting}
//-------------------------------------------- METRONOMO ---
//------------------------------- dichiarazioni iniziali ---
declare name "METRONOMO";
import("stdfaust.lib");
//-------------------------------- cursore selezione BPM ---
//-------------------- init-min--max-step-------------------
BPM = hslider("[01]BPM", 60, 40, 240, 0.1);
//------------------- numero di campioni per ogni minuto ---
SPM = (60*ma.SR);
//------------- formula di conversione da BPM a campioni ---
tempo = SPM/BPM : int; // solo la parte intera
//---------------------------------------------- SINTESI ---
//------------------------------------- filtro risonante ---
fireson = fi.highpass(128, 1000) : fi.lowpass(8, 1000);
//------------------------------------ impulso risonante ---
metronomo = ba.pulsen(1, tempo) : fireson;
//-------------------- METRONOMO SU DUE CANALI DI USCITA ---
process =   metronomo <: _,_ ;
\end{lstlisting}
