\begin{longtable}{r|L{0.71\textwidth}}
  \multicolumn{2}{c}{\textbf{CLASSE V LICEO MUSICALE • I Quadrimestre}} \\
  \multicolumn{2}{c}{\textbf{I modulo Titolo: Linguaggi di programmazione, interattività e multimedialità}} \\
  \endhead
  \hline
  \textbf{Obiettivi} &
  \textbf{Conoscenze:} \newline
    Acquisire i principali strumenti critici per l'analisi dell'audiovisivo.\newline
    Conoscere le principali tecniche di produzione e post-produzione audio e musicale in relazione all'audiovisivo.\newline
    Completare la conoscenza del percorso storico della musica elettroacustica, le aree territoriali, i compositori e le relazioni storiche tra essi.\newline
    Principi di interattività, installazione, realtà virtuale, multimedialità. \newline
  \textbf{Abilità:} \newline
    Lettura e comprensione di testi della letteratura specifica, manuali operativi e utente, di strumenti di lavoro, hardware e software.\newline
    Elaborazione e produzione di progetti compositivi, sia in forma di esercizio che di produzione autonoma, in lavori di gruppo con sistemi di condivisione dati e networking.\newline
    Accesso alla stesura di strutture musicali per gli audiovisivi e le opere interattive.\newline
    Elaborazione e produzione di un repertorio elettroacustico.\newline
  \textbf{Competenze:} \newline
    Completa padronanza della terminologia specifica, degli strumenti di studio, esercizio e lavoro applicato alle tecnologie musicali.\newline
    Completa autonomia critica ed analitica di opere elettracustiche e digitali.\newline
    Completa autonomia di produzione ed elaborazione personale spontanea e su commissione. \\
  \hline
  \textbf{Contenuti} &
    La tecnologia come strumento di esercizio e lavoro applicato alle
    discipline di studio. L'approfondimento teorico e la prova sperimentale
    come approccio d'indagine. Il repertorio musicale e la letteratura come
    guida tra le tematiche d'approfondimento. \newline
    Il computer come strumento e ambiente di lavoro polifunzionale.
    Il linguaggio di programmazione come strumento di pensiero. \newline
    L'elaborazione dei suoni al computer, elaborazione strutturata di un segnale,
    le tecniche di \emph{digital signal processing - dsp}.\newline
    La notazione musicale contemporanea, suono segno ed interpretazione.\newline
    Le superfici di controllo, le automazioni i sensori, strutturazione e
    programmazione di un proprio ambiente di lavoro personalizzato. \\
  \hline
  \textbf{Metodo} &
    Lezione frontale in laboratorio. \newline
    Ogni tematica è affrontata accedendovi dal repertorio musicale.
    Si ascolta, analizza e comprende il brano di repertorio ed attraverso
    la sua applicazione musicale si approfondiscono gli argomenti emergenti.
    Si veicola quindi lo studio teorico e tecnico attraverso la pratica
    musicale, senza perdere il collegamento con il repertorio storico.
    Ogni fenomeno uditivo, percettivo e tecnologico è radicato nella
    musica del suo tempo, acquisibile mediante l'analisi, l'interpretazione
    e la pratica dello strumento tecnologico musicale. \\
  \hline
  \textbf{Tempi} &
    Mesi da Settembre a Gennaio\\
  \hline
  \textbf{Materiali} &
    Dispense, Estratti di testi in letteratura, partiture del repertorio,
    contenuti multimediali. \\
  \hline
  \textbf{Tipologia della verifica} &
    Prove pratiche sulle attività svolte durante i laboratori. \newline
    Questionari di diversa tipologia. \newline
    Verifiche orali.
\end{longtable}%

\clearpage

\begin{longtable}{r|L{0.71\textwidth}}
  \multicolumn{2}{c}{\textbf{CLASSE V LICEO MUSICALE • II Quadrimestre}} \\
  \multicolumn{2}{c}{\textbf{II modulo Titolo: Musicista e Tecnologia, competenze trasversali.}} \\
  \endhead
  \hline
  \textbf{Obiettivi} &
    \textbf{Conoscenze:} \newline
      Acquisire i principali strumenti critici per l'analisi dell'audiovisivo.  \newline
      Conoscere le principali tecniche di produzione e post-produzione audio e musicale in relazione all'audiovisivo.  \newline
      Completare la conoscenza del percorso storico della musica elettroacustica, le aree territoriali, i compositori e le relazioni storiche tra essi.  \newline
      Conoscenza delle pratiche di interattività, installazione, realtà virtuale, multimedialità.  \newline
    \textbf{Abilità:} \newline
      Elaborazione e produzione di progetti compositivi, relazioni, articoli scientifici e documentazione operativa, partiture e schede tecniche. \newline
      Elaborazione e produzione di strutture musicali per gli audiovisivi e le opere interattive.\newline
      Elaborazione e produzione di un repertorio elettroacustico. \newline
    \textbf{Competenze:} \newline
      Completa padronanza della terminologia specifica, degli strumenti di studio, esercizio e lavoro applicato alle tecnologie musicali. \newline
      Completa autonomia critica ed analitica di opere elettracustiche e digitali. \newline
      Completa autonomia di produzione ed elaborazione personale spontanea e su commissione. \\
    \hline
    \textbf{Contenuti} &
      La tecnologia come strumento di esercizio e lavoro applicato alle
      discipline di studio. L'approfondimento teorico e la prova sperimentale
      come approccio d'indagine. Il repertorio musicale e la letteratura come
      guida tra le tematiche d'approfondimento. \newline
      Il computer come strumento e ambiente di lavoro polifunzionale. \newline
      Il linguaggio di programmazione come strumento di pensiero. \newline
      L'elaborazione dei suoni al computer, elaborazione strutturata di un segnale,
      le tecniche di \emph{digital signal processing - dsp}. \newline
      La notazione musicale contemporanea, suono segno ed interpretazione. \newline
      Studio, ricerca e produzione, acquisizione dei mezzi per sviluppare il
      proprio percorso musicale. \\
    \hline
    \textbf{Metodo} &
      Lezione frontale in laboratorio. \newline
      Ogni tematica è affrontata accedendovi dal repertorio musicale.
      Si ascolta, analizza e comprende il brano di repertorio ed attraverso
      la sua applicazione musicale si approfondiscono gli argomenti emergenti.
      Si veicola quindi lo studio teorico e tecnico attraverso la pratica
      musicale, senza perdere il collegamento con il repertorio storico.
      Ogni fenomeno uditivo, percettivo e tecnologico è radicato nella
      musica del suo tempo, acquisibile mediante l'analisi, l'interpretazione
      e la pratica dello strumento tecnologico musicale. \\
    \hline
    \textbf{Tempi} &
      Mesi da Febbraio a Giugno \\
    \hline
    \textbf{Materiali} &
      Dispense, Estratti di testi in letteratura, partiture del repertorio,
      contenuti multimediali. \\
    \hline
    \textbf{Tipologia della verifica} &
      Prove pratiche sulle attività svolte durante i laboratori. \newline
      Questionari di diversa tipologia. \newline
      Verifiche orali. \newline
      Presentazione della produzione personale acquisita durante l'anno scolastico. \newline
      Prova pratica interpretazione d'insieme in forma di concerto.
  \end{longtable}%

\clearpage

\begin{longtable}{R{0.21\textwidth}|L{0.71\textwidth}}
\multicolumn{2}{c}{\textbf{CRITERIO DI SUFFICIENZA}} \\
\endhead
\hline
\multicolumn{2}{c}{L’alunno conseguirà la valutazione sufficiente quando dimostrerà di possedere:} \\
\hline
\multicolumn{2}{c}{\textbf{Nel I Quadrimestre:}} \\
\hline
Conoscenze: &
Conoscenza completa delle discipline in oggetto, del lessico specifico, degli strumenti di lavoro. \\
\hline
Abilità: &
Capacità di gestione autonoma di attività tecniche e pratiche all'interno del laboratorio, individuali e collettive. \\
\hline
Competenze: &
Competenza e lessico appropriato nel rapporto con gli strumenti di lavoro e le tecnologie. \newline
Competenza dell'ambiente virtuale al computer, dell'acquisizione e trattamento digitale dei suoni. \newline
Autonomia critica, analitica e di giudizio, d'ascolto e relazione con la musica e i media. \\
\hline
\multicolumn{2}{c}{\textbf{Nel II Quadrimestre:}} \\
\hline
Conoscenze: &
Conoscenza completa delle discipline in oggetto, del lessico specifico, degli strumenti di lavoro. \\
\hline
Abilità: &
Capacità di gestione autonoma di attività tecniche e pratiche all'interno del laboratorio, individuali e collettive. \\
\hline
Competenze: &
Competenza e lessico appropriato nel rapporto con gli strumenti di lavoro e le tecnologie. \newline
Competenza dell'ambiente virtuale al computer, dell'acquisizione e trattamento digitale dei suoni. \newline
Autonomia critica, analitica e di giudizio, d'ascolto e relazione con la musica e i media.
\end{longtable}

\clearpage
